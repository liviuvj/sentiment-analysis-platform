\documentclass[a4paper,12pt,twoside]{memoir}

% Castellano
\usepackage[spanish,es-tabla]{babel}
\selectlanguage{spanish}
\usepackage[utf8]{inputenc}
\usepackage[T1]{fontenc}
\usepackage{lmodern} % Scalable font
\usepackage{microtype}
\usepackage{placeins}

%Para el float H de tablaSmallSinColores
\usepackage{float}

\RequirePackage{booktabs}
\RequirePackage[table]{xcolor}
\RequirePackage{xtab}
\RequirePackage{multirow}

% Links
\usepackage[hyphens]{url}

\usepackage[colorlinks]{hyperref}
\hypersetup{
	allcolors = {red}
}

% Ecuaciones
\usepackage{amsmath}

% Rutas de fichero / paquete
\newcommand{\ruta}[1]{{\sffamily #1}}

% Párrafos
\nonzeroparskip

% Huérfanas y viudas
\widowpenalty100000
\clubpenalty100000

% Evitar solapes en el header
\nouppercaseheads

% Imagenes
\usepackage{graphicx}
\newcommand{\imagen}[2]{
	\begin{figure}[!h]
		\centering
		\includegraphics[width=0.9\textwidth]{#1}
		\caption{#2}\label{fig:#1}
	\end{figure}
	\FloatBarrier
}

\newcommand{\imagenflotante}[2]{
	\begin{figure}%[!h]
		\centering
		\includegraphics[width=0.9\textwidth]{#1}
		\caption{#2}\label{fig:#1}
	\end{figure}
}

% Imágen con fuente
\newcommand{\imagenConFuente}[3]{
	\begin{figure}[!h]
		\centering
		\includegraphics[width=0.9\textwidth]{#1}
		\caption[#2]{#2. Fuente:~#3}\label{fig:#1}
	\end{figure}
	\FloatBarrier
}



% El comando \figura nos permite insertar figuras comodamente, y utilizando
% siempre el mismo formato. Los parametros son:
% 1 -> Porcentaje del ancho de página que ocupará la figura (de 0 a 1)
% 2 --> Fichero de la imagen
% 3 --> Texto a pie de imagen
% 4 --> Etiqueta (label) para referencias
% 5 --> Opciones que queramos pasarle al \includegraphics
% 6 --> Opciones de posicionamiento a pasarle a \begin{figure}
\newcommand{\figuraConPosicion}[6]{%
  \setlength{\anchoFloat}{#1\textwidth}%
  \addtolength{\anchoFloat}{-4\fboxsep}%
  \setlength{\anchoFigura}{\anchoFloat}%
  \begin{figure}[#6]
    \begin{center}%
      \Ovalbox{%
        \begin{minipage}{\anchoFloat}%
          \begin{center}%
            \includegraphics[width=\anchoFigura,#5]{#2}%
            \caption{#3}%
            \label{#4}%
          \end{center}%
        \end{minipage}
      }%
    \end{center}%
  \end{figure}%
}

%
% Comando para incluir imágenes en formato apaisado (sin marco).
\newcommand{\figuraApaisadaSinMarco}[5]{%
  \begin{figure}%
    \begin{center}%
    \includegraphics[angle=90,height=#1\textheight,#5]{#2}%
    \caption{#3}%
    \label{#4}%
    \end{center}%
  \end{figure}%
}
% Para las tablas
\newcommand{\otoprule}{\midrule [\heavyrulewidth]}
%
% Nuevo comando para tablas pequeñas (menos de una página).
\newcommand{\tablaSmall}[5]{%
 \begin{table}
  \begin{center}
   \rowcolors {2}{gray!35}{}
   \begin{tabular}{#2}
    \toprule
    #4
    \otoprule
    #5
    \bottomrule
   \end{tabular}
   \caption{#1}
   \label{tabla:#3}
  \end{center}
 \end{table}
}

%
% Nuevo comando para tablas pequeñas (menos de una página).
\newcommand{\tablaSmallSinColores}[5]{%
 \begin{table}[H]
  \begin{center}
   \begin{tabular}{#2}
    \toprule
    #4
    \otoprule
    #5
    \bottomrule
   \end{tabular}
   \caption{#1}
   \label{tabla:#3}
  \end{center}
 \end{table}
}

% Tablas pequeñas sin colores y con fuente.
\newcommand{\tablaSmallSinColoresConFuente}[6]{%
 \begin{table}[H]
  \begin{center}
   \begin{tabular}{#3}
    \toprule
    #5
    \otoprule
    #6
    \bottomrule
   \end{tabular}
   \caption[#1]{#1. Fuente:~#2}
   \label{tabla:#4}
  \end{center}
 \end{table}
}

\newcommand{\tablaApaisadaSmall}[5]{%
\begin{landscape}
  \begin{table}
   \begin{center}
    \rowcolors {2}{gray!35}{}
    \begin{tabular}{#2}
     \toprule
     #4
     \otoprule
     #5
     \bottomrule
    \end{tabular}
    \caption{#1}
    \label{tabla:#3}
   \end{center}
  \end{table}
\end{landscape}
}

%
% Nuevo comando para tablas grandes con cabecera y filas alternas coloreadas en gris.
\newcommand{\tabla}[6]{%
  \begin{center}
    \tablefirsthead{
      \toprule
      #5
      \otoprule
    }
    \tablehead{
      \multicolumn{#3}{l}{\small\sl continúa desde la página anterior}\\
      \toprule
      #5
      \otoprule
    }
    \tabletail{
      \hline
      \multicolumn{#3}{r}{\small\sl continúa en la página siguiente}\\
    }
    \tablelasttail{
      \hline
    }
    \bottomcaption{#1}
    \rowcolors {2}{gray!35}{}
    \begin{xtabular}{#2}
      #6
      \bottomrule
    \end{xtabular}
    \label{tabla:#4}
  \end{center}
}

%
% Nuevo comando para tablas grandes con cabecera.
\newcommand{\tablaSinColores}[6]{%
  \begin{center}
    \tablefirsthead{
      \toprule
      #5
      \otoprule
    }
    \tablehead{
      \multicolumn{#3}{l}{\small\sl continúa desde la página anterior}\\
      \toprule
      #5
      \otoprule
    }
    \tabletail{
      \hline
      \multicolumn{#3}{r}{\small\sl continúa en la página siguiente}\\
    }
    \tablelasttail{
      \hline
    }
    \bottomcaption{#1}
    \begin{xtabular}{#2}
      #6
      \bottomrule
    \end{xtabular}
    \label{tabla:#4}
  \end{center}
}

%
% Nuevo comando para tablas grandes sin cabecera.
\newcommand{\tablaSinCabecera}[5]{%
  \begin{center}
    \tablefirsthead{
      \toprule
    }
    \tablehead{
      \multicolumn{#3}{l}{\small\sl continúa desde la página anterior}\\
      \hline
    }
    \tabletail{
      \hline
      \multicolumn{#3}{r}{\small\sl continúa en la página siguiente}\\
    }
    \tablelasttail{
      \hline
    }
    \bottomcaption{#1}
  \begin{xtabular}{#2}
    #5
   \bottomrule
  \end{xtabular}
  \label{tabla:#4}
  \end{center}
}



\definecolor{cgoLight}{HTML}{EEEEEE}
\definecolor{cgoExtralight}{HTML}{FFFFFF}

%
% Nuevo comando para tablas grandes sin cabecera.
\newcommand{\tablaSinCabeceraConBandas}[5]{%
  \begin{center}
    \tablefirsthead{
      \toprule
    }
    \tablehead{
      \multicolumn{#3}{l}{\small\sl continúa desde la página anterior}\\
      \hline
    }
    \tabletail{
      \hline
      \multicolumn{#3}{r}{\small\sl continúa en la página siguiente}\\
    }
    \tablelasttail{
      \hline
    }
    \bottomcaption{#1}
    \rowcolors[]{1}{cgoExtralight}{cgoLight}

  \begin{xtabular}{#2}
    #5
   \bottomrule
  \end{xtabular}
  \label{tabla:#4}
  \end{center}
}


\graphicspath{ {./img/} }

% Capítulos
\chapterstyle{bianchi}
\newcommand{\capitulo}[2]{
	\setcounter{chapter}{#1}
	\setcounter{section}{0}
	\chapter*{#2}
	\addcontentsline{toc}{chapter}{#1. #2}
	\markboth{#2}{#2}
}

% Apéndices
\renewcommand{\appendixname}{Apéndice}
\renewcommand*\cftappendixname{\appendixname}

\newcommand{\apendice}[1]{
	%\renewcommand{\thechapter}{A}
	\chapter{#1}
}

\renewcommand*\cftappendixname{\appendixname\ }

% Formato de portada
\makeatletter
\usepackage{xcolor}
\newcommand{\tutor}[1]{\def\@tutor{#1}}
\newcommand{\course}[1]{\def\@course{#1}}
\definecolor{cpardoBox}{HTML}{E6E6FF}
\def\maketitle{
  \null
  \thispagestyle{empty}
  % Cabecera ----------------
\begin{center}%
	{\noindent\Huge Universidades de Burgos, León y Valladolid}\vspace{.5cm}%
	
	{\noindent\Large Máster universitario}\vspace{.5cm}%
	
	{\noindent\Huge \textbf{Inteligencia de Negocio y Big~Data en Entornos Seguros}}\vspace{.5cm}%
\end{center}%

\begin{center}%
	\includegraphics[height=3cm]{img/escudoUBU} \hspace{1cm}
	\includegraphics[height=3cm]{img/escudoUVA} \hspace{1cm}
	\includegraphics[height=3cm]{img/escudoULE} \vspace{1cm}%
\end{center}%

  \vfill
  % Título proyecto y escudo informática ----------------
  \colorbox{cpardoBox}{%
    \begin{minipage}{.9\textwidth}
      \vspace{.5cm}\Large
      \begin{center}
      \textbf{TFM del Máster Inteligencia de Negocio y Big Data en Entornos Seguros}\vspace{.6cm}\\
      \textbf{\LARGE\@title{}}
      \end{center}
      \vspace{.2cm}
    \end{minipage}

  }%
  \hfill
  \vfill
  % Datos de alumno, curso y tutores ------------------
  \begin{center}%
  {%
    \noindent\LARGE
    Presentado por \@author{}\\ 
    en Universidad de Burgos --- \@date{}\\
    Tutor: \@tutor{}\\
  }%
  \end{center}%
  \null
  \cleardoublepage
  }
\makeatother

\newcommand{\nombre}{Liviu Viorel Jula Vacar} %%% cambio de comando

% Datos de portada
\title{Herramienta \textit{open-source} para análisis de sentimientos en redes sociales}
\author{\nombre}
\tutor{Dr. Álvar Arnaiz González}
\date{\today}

\begin{document}

\maketitle


\newpage\null\thispagestyle{empty}\newpage


%%%%%%%%%%%%%%%%%%%%%%%%%%%%%%%%%%%%%%%%%%%%%%%%%%%%%%%%%%%%%%%%%%%%%%%%%%%%%%%%%%%%%%%%
\thispagestyle{empty}


\noindent
\begin{center}%
	{\noindent\Huge Universidades de Burgos, León y Valladolid}\vspace{.5cm}%
	
\begin{center}%
	\includegraphics[height=3cm]{img/escudoUBU} \hspace{1cm}
	\includegraphics[height=3cm]{img/escudoUVA} \hspace{1cm}
	\includegraphics[height=3cm]{img/escudoULE} \vspace{1cm}%
\end{center}%

	{\noindent\Large \textbf{Máster universitario en Inteligencia de Negocio y Big~Data en Entornos Seguros}}\vspace{.5cm}%
\end{center}%



\noindent D. Álvar Arnaiz González, profesor del departamento de Ingeniería Informática, Área de Lenguajes y Sistemas Informáticos.

\noindent Expone:

\noindent Que el alumno D. \nombre, con DNI dni, ha realizado el Trabajo final de Máster en Inteligencia de Negocio y Big Data en Entornos Seguros 
          titulado ``Herramienta \textit{open-source} para análisis de sentimientos en redes sociales sobre palabras clave o temas específicos''. 

\noindent Y que dicho trabajo ha sido realizado por el alumno bajo la dirección del que suscribe, en virtud de lo cual se autoriza su presentación y defensa.

\begin{center} %\large
En Burgos, {\large \today}
\end{center}

\vfill\vfill\vfill

% Author and supervisor
\begin{minipage}{0.9\textwidth}
\begin{center} %\large
Vº. Bº. del Tutor:\\[2cm]
D. Álvar Arnaiz González
\end{center}
\end{minipage}
\hfill

\vfill

% para casos con solo un tutor comentar lo anterior
% y descomentar lo siguiente
%Vº. Bº. del Tutor:\\[2cm]
%D. nombre tutor


\newpage\null\thispagestyle{empty}\newpage




\frontmatter

% Abstract en castellano
\renewcommand*\abstractname{Resumen}
\begin{abstract}
Las opiniones públicas que se realizan en Internet sobre diversos productos, servicios, marcas o lugares pueden influenciar el comportamiento de las personas. La gran mayoría de estas opiniones se difunden en redes sociales, foros de discusión y en los diferentes sitios web de reseñas.

El objetivo de este trabajo es emplear la información disponible públicamente para crear una herramienta que permita realizar un análisis de sentimientos sobre ciertas palabras clave o temas específicos de los que se quiera obtener información. Se realizará un proceso \textit{ETL} (\textit{Extract -- Transform -- Load}) para el procesamiento de los datos y se implementará un \textit{dashboard} que permita visualizar y explorar la información obtenida finalmente.

Para conseguir estos objetivos, se emplearán tecnologías \textit{open-source} para el desarrollo de la herramienta y técnicas de procesamiento de lenguaje natural para el análisis de sentimientos basado en aspectos (\textit{Aspect-Based Sentiment Analysis}, \textit{ABSA}).
\end{abstract}

\renewcommand*\abstractname{Descriptores}
\begin{abstract}
Aprendizaje automático, procesamiento de lenguaje natural, sentiment analysis, big data, ETL, \textit{dashboard}, visualización de datos, \textit{open-source}.
\end{abstract}

\clearpage

% Abstract en inglés
\renewcommand*\abstractname{Abstract}
\begin{abstract}
Public opinions made on the Internet about various products, services, brands or places can influence people's behavior. The vast majority of these opinions are shared on social media, discussion forums and on the different review websites.

The aim of this project is to make use of publicly available information to create a tool able to perform sentiment analysis on certain keywords or specific topics for which we want to obtain information. An ETL workflow will be used to process the data and a dashboard will be implemented to visualize and explore the final information obtained.

To achieve these objectives, open-source technologies will be used for the development of the tool and natural language processing techniques for aspect-based sentiment analysis.
\end{abstract}

\renewcommand*\abstractname{Keywords}
\begin{abstract}
Machine learning, natural language processing, sentiment analysis, big data, ETL, dashboard, data visualization, open-source.
\end{abstract}

\clearpage

% Indices
\tableofcontents

\clearpage

\listoffigures

\clearpage

\listoftables
\clearpage

\mainmatter

\part*{Memoria}
\addcontentsline{toc}{part}{Memoria}


\capitulo{1}{Introducción}

Cada día se genera una inmensa cantidad de datos en Internet, esto supone una fuente de información muy útil para numerosos casos de uso. Las redes sociales ofrecen de manera pública la gran mayoría de estos en forma de opiniones de personas, lo que invita a su estudio mediante el uso de técnicas como el análisis de sentimientos.

No obstante, para llegar a obtener conocimiento a partir de todos esos datos en bruto, es necesaria la capacidad de explotarlos de manera eficiente. Este proyecto tomará como objetivo la creación de una plataforma \textit{big data} de código abierto para el análisis de sentimientos. 

En los siguientes apartados se describe el planteamiento del proyecto, la estructura de la memoria y los materiales adjuntos a la misma.

\section{Estructura de la memoria}

La memoria está organizada de la siguiente manera:

\begin{itemize}
    \item \textbf{Introducción:} La estructuración de la memoria, los materiales adjuntos a la misma y el planteamiento del proyecto.
    \item \textbf{Objetivos del proyecto:} Descripción de los objetivos definidos inicialmente que se han intentado cumplir en su totalidad.
    \item \textbf{Conceptos teóricos:} Explicación de temas a tener en cuenta para el desarrollo del proyecto y de los modelos de \textit{Deep Learning} empleados.
    \item \textbf{Técnicas y herramientas:} Que se han utilizado para la realización del proyecto y facilitar algunas labores de programación, desarrollo o gestión.
    \item \textbf{Aspectos relevantes del desarrollo del proyecto:} Se entra en detalle en los puntos clave que han tenido mayor importancia para la correcta realización de este proyecto. En esta sección se expondrán los desafíos presentados y los pasos llevados a cabo para solventarlos.
    \item \textbf{Trabajos relacionados:} Aplicaciones existentes relacionadas con el proyecto y <<estado del arte>> sobre técnicas de procesamiento de lenguaje natural.
    \item \textbf{Conclusiones y líneas de trabajo futuras:} Deducciones y resultados obtenidos a lo largo del desarrollo de este Trabajo de Fin de Máster y posibles mejoras posteriores.
\end{itemize}

Junto a la memoria, se aportan también los siguientes anexos:

\begin{itemize}
    \item \textbf{Plan de proyecto software:} La planificación temporal del proyecto y estudio de la viabilidad económica y legal del mismo.
    \item \textbf{Especificación de requisitos:} Detalle de los requisitos técnicos, explicados los objetivos generales y el catálogo de requisitos.
    \item \textbf{Especificación de diseño:} Explicación de la arquitectura de la plataforma desarrollada y el diseño de datos.
    \item \textbf{Documentación técnica de programación:} La documentación técnica para la instalación, despliegue e integración de futuras funcionalidades.
    \item \textbf{Documentación de usuario:} Guía para utilizar el proyecto de manera segura y correcta.
\end{itemize}

\section{Materiales adjuntos}

Además de la memoria y de los anexos, se proporcionan también los siguientes materiales disponibles de manera \textit{online}:

\begin{itemize}
    \item Repositorio del proyecto en \textit{GitHub}.\\
        \url{https://github.com/liviuvj/sentiment-analysis-platform}
    
    \item Contenedor \textit{Docker} con las mejoras desarrolladas del conector para Twitter de  Airbyte.\\
        \url{https://hub.docker.com/r/liviuvj/airbyte-source-twitter/tags}

    \item Vídeos sobre el funcionamiento de la plataforma:
    \begin{itemize}
        \item Funcionamiento general de la plataforma.\\
            \url{...}
        \item Explicación detallada de la plataforma.\\
            \url{...}
    \end{itemize}

    \item Los datos extraídos de \textit{API} mediante la herramienta Airbyte.\\
        \url{...}

    \item El análisis realizado en \textit{Google Colaboratory}.\\
        \url{https://colab.research.google.com/drive/1d_obU9idFqjsDi7ezeFs1CORxTUAgh7V#offline=true&sandboxMode=true}

    \item Las particiones creadas a partir del conjuntos de datos utilizado.
    \begin{itemize}
        \item \textit{dataset\_movie}.\\
            \url{https://drive.google.com/file/d/1doLbhxFP5y4TRoMUpx6kgaIR3MUFVDVX/view}
        \item \textit{dataset\_got}.\\
            \url{...}
        \item \textit{dataset\_season8}.\\
            \url{https://drive.google.com/file/d/1tSA5bGkBGfgVWdltxLEEfd_NNm0qnUYs/view}
        \item \textit{dataset\_daenerys}.\\
            \url{https://drive.google.com/file/d/1hL3eh3K2lKtMNEkaG2JSgNSXjoNtHyTG/view}
        \item \textit{dataset\_jon}.\\
            \url{https://drive.google.com/file/d/1Uji4IajDAYlAj3yhQdQ9B1NcSFg-pqrG/view}
    \end{itemize}
    
\end{itemize}

\section{Planteamiento del proyecto}

El proyecto está formado por varios componentes, todos ellos de código abierto, que se encargan de diferentes tareas. Todos los componentes de la plataforma están completamente <<\textit{dockerizados}>>, es decir, empaquetados y aislados en contenedores \textit{Docker} independientes y desplegables en cualquier máquina.

Con estas decisiones de diseño se elimina la dependencia entre componentes, creando así una arquitectura modular para la plataforma. Por lo que resulta adaptable a distintos casos de uso o a la utilización de distintos componentes a los integrados actualmente para cada parte del proyecto, creando así una plataforma configurable según las necesidades que se planteen. En la \autoref{fig:platform-architecture-basic} se puede observar la arquitectura planteada del proyecto.

\imagen{platform-architecture-basic}{Vista básica de la arquitectura del proyecto}

El primero es \textit{Airbyte}, una herramienta que permite la extracción de datos desde distintos orígenes, como \textit{APIs}, bases de datos o archivos disponibles en la \textit{web}. Se ha utilizado para la ingestión de datos desde \textit{API} y desde \textit{Google Drive}. También se ha desarrollado y mejorado el conector base de Twitter que presenta la herramienta y se ha publicado en el repositorio oficial de la misma, quedando así disponible su uso para la comunidad.

El segundo es \textit{MongoDB}, una base de datos no relacional orientada a documentos. Se encarga de almacenar los datos en bruto de todas las fuentes de datos, actuando de esta manera como un \textit{Data Lake}\footnote{Un \textit{Data Lake} es un repositorio centralizado que almacena una gran cantidad y variedad de datos en su formato original. Esto permite a los usuarios acceder a los datos de forma flexible y rápida. Además de facilitar el descubrimiento de patrones, tendencias e \textit{insights}~\cite{awsDataLake}.} para la plataforma.

El tercero es \textit{Apache Spark}, un \textit{framework} de computación distribuida que permite procesar grandes volúmenes de datos de forma paralela y eficiente. Su labor en el proyecto se centra en el preprocesamiento de los datos extraídos, tanto su limpieza y filtrado como el enriquecimiento mediante <<metadatos>>.

El cuarto es \textit{HuggingFace Transformers}, una librería del lenguaje de programación Python que ofrece modelos preentrenados de procesamiento del lenguaje natural (\textit{NLP, Natural Language Processing}) basados en \textit{Deep Neural Networks}. Esta librería se utiliza para aplicar técnicas de \textit{NLP} sobre los datos preprocesados, desde la clasificación de sentimientos hasta la detección de entidades. 

El quinto es \textit{ClickHouse}, una base de datos columnar orientada al procesamiento analítico de datos en línea (\textit{OLAP, On-Line Analytical Processing}) que permite realizar consultas rápidas y complejas sobre los datos. Su labor es ejercer de \textit{Data Warehouse}\footnote{Un \textit{Data Warehouse} es un sistema de almacenamiento de datos que integra información de diversas fuentes y que facilita la ejecución de consultas y análisis complejos. La diferencia con una base de datos tradicional reside en que está diseñado para facilitar el procesamiento de datos analítico (\textit{OLAP}), no transaccional~\cite{sasDataWarehouse}.} del proyecto, puesto que almacenará los datos finales enriquecidos tras las etapas anteriores.

El sexto es \textit{Apache Superset}, una herramienta de visualización de datos que permite crear cuadros de mando interactivos y personalizados con gráficos, mapas o tablas. Su labor consistirá en permitir a los usuarios la creación y visualización de \textit{dashboards} para explotar la información obtenida hasta el momento, como exploración de datos, métricas resultantes y análisis de sentimientos.

El séptimo es \textit{Apache Airflow}, una plataforma de orquestación de flujos de trabajo que permite automatizar y programar tareas. Este componente se ha utilizado para gestionar las \textit{data pipelines} diseñadas y organizar las interacciones entre los demás componentes, como el despliegue dinámico de \textit{Spark} y \textit{Transformers} solamente cuando resulte necesario, permitiendo así la optimización de los recursos.

El octavo es una interfaz web como desarrollo propio y a medida para este proyecto. Tiene como objetivo servir de punto de acceso centralizado a las demás aplicaciones web que ofrecen las distintas herramientas empleadas.

El planteamiento de esta plataforma ofrece una solución \textit{big data} modular e integral para el análisis de sentimientos. La combinación de los componentes descritos ha sido elegida por formar una plataforma con tecnologías modernas, flexibles y altamente escalables que permitan explotar de manera eficiente los datos y obtener el valor esperado.

\capitulo{2}{Objetivos del proyecto}

Este apartado explica de forma precisa y concisa los objetivos que se persiguen con la realización del proyecto. Se realiza una distinción entre los objetivos de carácter general, los de carácter técnico (propios del proyecto) y también los personales.

\section{Objetivos generales}

\begin{itemize}
    \item Realizar extracciones de datos de dominio público.

    \item Llevar a cabo tareas de procesamiento, limpieza y asegurar la calidad de los datos.

    \item Utilizar técnicas de procesamiento de lenguaje natural para enriquecer los datos.

    \item Diseñar e implementar cuadros de mando interactivos sobre la información obtenida.

    \item Aplicar procesos de Extracción, Transformación y Carga (\textit{ETL}).

    \item Diseñar y desarrollar una plataforma \textit{open-source} para análisis de sentimiento sobre \textit{Big Data}.
\end{itemize}

\vspace{2cm}

\section{Objetivos técnicos}

\begin{itemize}
    \item Utilizar herramientas y tecnologías de código abierto.

    \item Emplear tecnologías distribuidas y escalables capaces de soportar \textit{Big Data}.

    \item Estudiar los \textit{frameworks} y herramientas más convenientes a utilizar en cada etapa del proyecto.

    \item Diseñar e implementar \textit{data pipelines} para realizar el proceso \textit{ETL}.

    \item Permitir la programación temporal de la ejecución de las \textit{data pipelines}.

    \item Automatizar los procesos mediante herramientas de orquestación.

    \item Diseñar e implementar los \textit{dashboard} para la toma de decisiones siguiendo las buenas prácticas recomendadas.

    \item Mantener la calidad de los datos extraídos y procesados.

    \item Diseñar la plataforma con una arquitectura modular que permita el cambio por otras tecnologías en las distintas etapas del proyecto.

    \item Mantener la plataforma segura con distinción de usuarios y gestión de roles y permisos.

    \item Diseñar el esquema de datos a utilizar y las interacciones entre los componentes de la plataforma.

    \item Permitir la gestión o monitorización de la plataforma mediante interfaces web.

    \item Realizar el despliegue del proyecto mediante contenedores \textit{Docker}.
\end{itemize}

\vspace{5cm}

\section{Objetivos personales}

\begin{itemize}
    \item Poner en práctica la mayoría de conocimientos obtenidos a lo largo del máster sobre la consutrcción de infraestructuras para \textit{Big Data}, metodologías distribuidas y escalables de procesamiento de datos y \textit{business intelligence}. 

    \item Aprender a utilizar tecnologías modernas sobre extracción, transformación y carga de datos.

    \item Diseñar la arquitectura completa de una plataforma integral para ofrecer soluciones \textit{Big Data}. 

    \item Utilizar modelos del estado del arte sobre técnicas de procesamiento de lenguaje natural.

    \item Diseñar \textit{dashboards} interactivos y llamativos.
\end{itemize}


\capitulo{3}{Conceptos teóricos}

En aquellos proyectos que necesiten para su comprensión y desarrollo de unos conceptos teóricos de una determinada materia o de un determinado dominio de conocimiento, debe existir un apartado que sintetice dichos conceptos.

\capitulo{4}{Técnicas y herramientas}

Esta parte de la memoria tiene como objetivo presentar las técnicas metodológicas y las herramientas de desarrollo que se han utilizado para llevar a cabo el proyecto. En el caso de algunas de estas herramientas se estudiarán diferentes alternativas, en las que se incluirán comparativas entre las distintas opciones y una justificación de la elección realizadas.



\section{Técnicas}

En este apartado se hara una breve descripción sobre las técnicas empleadas a lo largo del proyecto.

\subsection{SCRUM}
Es un proceso de desarrollo software enfocado hacia las metodologías ágiles. Consiste en segmentar un proyecto en varios requisitos que se han de cumplir y posteriormente subdividir estos en tareas. El desarrollo se realiza mediante \textit{sprints}, iteraciones incrementales de normalmente dos semanas de duración, en los que se planifican las tareas a realizar durante dicho periodo.

\subsection{Procesamiento de Lenguaje Natural}
El término \textit{NLP} (\textit{Natural Language Processing}) se refiere al conjunto de métodos dentro de la inteligencia artificial que trabajan con recursos textuales o sonoros. Se ponen en práctica metodologías de estadística, lingüística y \textit{machine learning} para permitir crear programas que puedan interpretar dicho tipo de información.

\subsection{Sentiment Analysis}
El análisis de sentimientos es una técnica en la que se busca identificar y extraer información subjetiva a partir de recursos textuales. Las principales maneras de realizar este tipo de análisis siguen dos rutas.

La primera, utilizando reglas y diccionarios de palabras a las que se les asigna distintas puntuaciones según el sentimiento asociado a cada palabra. La segunda, y la que mejores resultados proporciona actualmente, emplea técnicas de \textit{NLP} para extraer características de los datos y comprender el contexto de la información proporcionada. Esto permite realizar clasificaciones y predicciones más acertadas ya que el resultado no se limita simplemente a un subconjunto de palabras, sino al sentido que se les da a las mismas también.

\section{Herramientas}

Para llevar a cabo este proyecto, se ha utilizado el siguiente conjunto de herramientas.

\subsection{GitHub}

Para el \textit{hosting} del repositorio se ha utilizado \textit{GitHub}\footnote{\url{https://github.com//}}, puesto que ya se tenía experiencia en el uso de esta plataforma. Permite realizar la gestión del control de versiones a lo largo del desarrollo del software y simplifica el seguimiento de las tareas. Posee capacidades para creación de procesos de integración continua y despliegue continuo (\textit{CI/CD}), automatización de flujos de trabajo, seguimiento y gestión de proyectos.

\subsection{ZenHub}
Para facilitar el trabajo de la gestión del proyecto se ha utilizado \textit{ZenHub}\footnote{\url{https://www.zenhub.com/}}. Es una plataforma centrada en mejorar la productividad de los equipos de desarrollo, que permite llevar a cabo la planificación del proyecto, realizar un seguimiento del progreso y calcular métricas de productividad mediante gráficas. 

Se ha elegido esta herramienta ya que, además de permitir realizar toda la gestión del proyecto, cuenta con una extensión web desde la que se puede acceder al panel de control directamente desde el propio repositorio de GitHub. Por lo que todas las operaciones de planificación de tareas se llevan a cabo desde el mismo lugar y facilita el trabajo del desarrollador.

\subsection{Entorno de desarrollo integrado (IDE)}

\subsubsection{Herramientas consideradas:}

\begin{itemize}
    \item \textbf{Spyder:} Entorno de desarrollo \textit{open-source} especializado en la exploración de datos y el análisis científico.
    \item \textbf{Visual Studio:} Herramienta que permite realizar todas las tareas de programación, depuración, pruebas y desarrollo de soluciones para cualquier plataforma.
    \item \textbf{Visual Studio Code:} Versión más ligera y personalizable de Visual Studio.
\end{itemize}

\subsubsection{Herramienta elegida:}

\begin{itemize}
    \item \textbf{Visual Studio Code}\footnote{\url{https://code.visualstudio.com/}}
\end{itemize}

Es el IDE elegido para llevar a cabo el desarrollo de proyecto. Como ventajas principales, presenta un tamaño reducido de instalación respecto a las otras opciones y permite la configuración y ejecución de tareas, además de la capacidad para instalar y personalizar nuevas funcionalidades mediante sus extensiones.

\subsubsection{Extensiones utilizadas}

Se han escogido una serie de extensiones del \textit{Marketplace} que presenta la herramienta para facilitar la calidad de vida al trabajar con este IDE.

\begin{itemize}
    \item \textbf{Python:} Extensión principal para dar soporte al lenguaje de programación Python para el correcto desarrollo de código (\textit{linting}, formato de código, exploración de variables, depuración, etc.).
    \item \textbf{Python Docstring Generator:} Facilita y Asiste en la creación de comentarios tipo \textit{docstring} para funciones en Python.
    \item \textbf{Pylance:} Servidor de lenguae que añade soporte adicional a Python.
    \item \textbf{Trailing Whitespace:} Resalta y recorta los espacios en blanco sobrantes.
    \item \textbf{Visual Studio IntelliCode:} Emplea IA para añadir desarrollo predictivo y autocompletado de código.
    \item \textbf{Docker:} Facilita la creación y gestión de contenedores a través del IDE.
\end{itemize}

\subsection{Editor \LaTeX}

\subsubsection{Herramientas consideradas:}

\begin{itemize}
    \item \textbf{MiK\TeX{} + Texmaker:} Herramientas que realizan la traducción de \LaTeX{} a texto y permiten gestionar y editar este tipo de archivos, respectivamente.
    \item \textbf{Overleaf:} Plataforma en línea que facilita la gestión y edición de documentos con formato \LaTeX{}.
\end{itemize}

\subsubsection{Herramienta elegida:}

\begin{itemize}
    \item \textbf{Overleaf}
\end{itemize}

Overleaf es un editor en línea\footnote{\url{https://es.overleaf.com/}} de \LaTeX{}. Para utilizarlo no es necesario realizar la instalación de ningún componente, tiene documentación integrada para \LaTeX y permite la visualización de los cambios realizados en tiempo real, además de contar ya con los paquetes más utilizados.

También resulta más cómodo al tratarse de una plataforma \textit{online}, ya que tan solo hace falta disponer de un navegador y conexión a Internet para poder trabajar con ella desde cualquier equipo. Otra de las mejores funcionalidades que ofrece es la posibilidad de comprobar el histórico de los archivos modificados y realizar un \textit{rollback} de los mismos.

Se ha utilizado esta herramienta para elaborar la memoria y los anexos en \LaTeX.

\subsection{Joplin}
A lo largo de la duración del proyecto hará falta tomar notas de varios temas diversos. Para facilitar esta tarea, se ha utilizado \textit{Joplin}\footnote{\url{https://joplinapp.org/}}. Es una plataforma de código abierto que permite gestionar apuntes y notas en forma de \textit{notebooks}.

Entre las principales características que ofrece se encuentra la total privacidad de los datos, la sencilla interfaz que presenta, la facilidad de uso gracias al lenguaje \textit{Markdown} y la sincronización de contenido entre diversos equipos.

Se utilizará principalmente para dejar constancia de los temas comentados durante las reuniones y apuntar información relevante para el proyecto que se vaya encontrando a medida que se desarrolle este trabajo.

\subsection{Super Productivity}
La gestión del tiempo dedicado se ha llevado a cabo mediante la herramienta de código abierto \textit{Super Productivity}\footnote{\url{https://super-productivity.com/}}. Sus principales funciones consisten en realizar la planificación, seguimiento y gestión de tareas. Permite distribuir tareas a lo largo de diversos proyectos, la asignación de etiquetas personalizadas y tener constancia del tiempo estimado y dedicado para cada una. 

Presenta una interfaz sencilla de utilizar y amigable para el usuario que agiliza el trabajo gracias a la utilización de atajos de teclado. Otra de las características más importantes que tiene esta herramientas es la integración con varias plataformas para la importación de tareas. Por lo que la planificación realizada en GitHub y ZenHub se puede extraer a esta herramienta y realizar un mejor seguimiento del tiempo empleado en cada una de ellas.
\capitulo{5}{Aspectos relevantes del desarrollo del proyecto} \label{section:relevant_aspects}

Este apartado pretende recoger los aspectos más interesantes del desarrollo del proyecto, además de la experiencia práctica adquirida durante la
realización del mismo con las diversas tecnologías empleadas.

\section{Extracción de datos} \label{section:data_extraction}

Comenzando con la primera etapa del proceso \textit{ETL}, el objetivo de la extracción de datos consiste en investigar y explotar los posibles recursos disponibles para recoger toda la información necesaria para el proyecto. Consecuentemente, los requisitos fundamentales de esta etapa consistirán en localizar las fuentes de datos a utilizar y emplear las herramientas necesarias para extraer dichos datos.

\subsection{Fuentes de datos}

La información necesaria para conseguir el objetivo de este proyecto está formada por opiniones públicas de personas sobre algún tema o temas en concreto. La manera más sencilla de obtener estos datos es empleando recursos web como foros, \textit{blogs} y redes sociales. Más concretamente, se ha optado por investigar la disponibilidad de \textit{APIs} públicas de los principales sitios web donde las personas publican sus opiniones.

A continuación, se realiza un pequeño resumen de la información de la que se dispone actualmente sobre las \textit{APIs} de cada plataforma.

\subsubsection{Twitter}

En 2006 se abrió al público la \textit{API REST} \cite{twitterGettingStarted} de Twitter, que actualmente se encuentra ya en su versión \textbf{v2}, aunque coexiste a su vez con algunas partes de la misma aún en la versión \textbf{v1.1} y otras de pago (\textit{Premium v1.1} o \textit{Enterprise}). Está basada en \textit{GraphQL}\footnote{Lenguaje de consultas para \textit{API}s que facilita la gestión de datos y peticiones (\url{https://graphql.org/}).} y devuelve los resultados en formato \textit{JSON}.

Los permisos que se deben asignar son solo de lectura o escritura de contenido. Mientras que el número de peticiones varía en función del \textit{endpoint}, la ventana temporal de restricción se limita a tan solo 15 minutos \cite{twitterRateLimits}. 

Ofrece acceso de lectura, escritura, modificación y borrado de una amplia variedad de recursos, como puede verse en la \autoref{tabla:recursosTwitter}.

\tablaSmallSinColoresConFuente{Recursos disponibles a través de la \textit{API} de Twitter}{\cite{twitterAPIreference}}{@{}p{.3 \textwidth} p{.1 \textwidth} p{.5 \textwidth}@{}}{recursosTwitter}
{\multicolumn{1}{l}{\textbf{Recurso}} & \multicolumn{1}{l}{\textbf{Versión}} & \multicolumn{1}{l}{\textbf{Descripción}}\\}{
\textbf{\textit{Tweets}} & v2 & Operaciones \textit{CRUD}. \\
& v1.1 & \\
& \textit{Premium} & \\
& \textit{Enterprise} & \\
\specialrule{.05em}{.05em}{0em}
\textbf{\textit{Users}} & v2 & Gestión y búsqueda de usuarios \\
& v1.1 & y relaciones entre los mismos. \\
& \textit{Premium} & \\
& \textit{Enterprise} & \\
\specialrule{.05em}{.05em}{0em}
\textbf{\textit{Spaces}} & v2 & Búsqueda de espacios y participantes. \\
\specialrule{.05em}{.05em}{0em}
\textbf{\textit{Direct Messages}} & v1.1 & Envío y respuesta a mensajes directos. \\
\specialrule{.05em}{.05em}{0em}
\textbf{\textit{Lists}} & v2 & Gestión de listas de contactos. \\
& v1.1 & \\ 
\specialrule{.05em}{.05em}{0em}
\textbf{\textit{Trends}} & v1.1 & Identificar tendencias por zonas geográficas. \\
\specialrule{.05em}{.05em}{0em}
\textbf{\textit{Media}} & v1.1 & Cargar archivos multimedia. \\
\specialrule{.05em}{.1em}{.1em}
\textbf{\textit{Places}} & v1.1 & Búsqueda de lugares. \\
}



\subsubsection{Facebook}

La \textit{API} de Facebook originalmente utilizaba \textit{FQL} (\textit{Facebook Query Language}) como lenguaje de consulta, parecido a \textit{SQL}. Sin embargo, en 2010 comenzó la migración hacia \textit{Graph API} \cite{facebookGraphAPI}, actualmente en su versión \textbf{v16.0}. Se organiza en función de colecciones, nodos y campos. Un nodo es un objeto único que representa una clase del diccionario de datos en concreto, mientras que los campos son atributos del mismo y una colección comprende un conjunto de nodos. Toda esta información es presentada en formato \textit{JSON}.

Presenta una gran cantidad de permisos \cite{facebookPermissions} que son requeridos para realizar las acciones de gestión, algunos de los cuales es necesario que sean aprobados por Facebook para su uso. Además, el número de peticiones que se pueden realizar se limita a 200 por hora por cada usuario \cite{facebookRateLimits}.

La lista de nodos que presenta la \textit{API} es muy extensa, aunque en la \autoref{tabla:nodosFacebook} se muestran algunos de ellos que podrían resultar útiles para el desarrollo de este proyecto.

\tablaSmallSinColoresConFuente{Muestra de nodos disponibles en \textit{Graph API} de Facebook}{\cite{facebookAPIreference}}{l l}{nodosFacebook}
{\multicolumn{1}{l}{\textbf{Nodo}} & \multicolumn{1}{l}{\textbf{Descripción}}\\}{
\textbf{\textit{Comment}} & Comentarios de los objetos. \\
\textbf{\textit{Link}} & Enlaces compartidos. \\
\textbf{\textit{Group}} & Objeto único de tipo grupo. \\
\textbf{\textit{Likes}} & Lista de personas que han dado \textit{like} a un objeto. \\
\textbf{\textit{Page}} & Información sobre páginas. \\
\textbf{\textit{User}} & Representación de un usuario. \\
}

\subsubsection{Instagram}

Lanzada originalmente en 2014 y actualmente integrada junto a la \textit{Graph API} de Facebook. Dispone de dos versiones, una más básica enfocada solamente al consumo de contenido, y la normal, que permite realizar diversos tipos de acciones sobre la cuenta y llevar a cabo su gestión.

Se dispone de un conjunto de permisos requeridos bastante más reducido que para la \textit{API} de Facebook, aunque el límite de peticiones es el mismo ya que funciona sobre la propia \textit{Graph API}.

Al estar basada también en la misma tecnología, la estructura consta de los mismos elementos mencionados en el apartado anterior. La diferencia serían los nodos principales en los que se distribuye su contenido, como se observa en la \autoref{tabla:nodosInstagram}.

\tablaSmallSinColoresConFuente{Muestra de nodos disponibles en \textit{Graph API} de Instagram}{\cite{instagramAPIreference}}{l l}{nodosInstagram}
{\multicolumn{1}{l}{\textbf{Nodo}} & \multicolumn{1}{l}{\textbf{Descripción}}\\}{
\textbf{\textit{Comment}} & Comentarios de los objetos. \\
\textbf{\textit{Hashtag}} & Representa un \textit{hashtag}. \\
\textbf{\textit{Multimedia}} & Referencia una foto, vídeo, historia o álbum. \\
\textbf{\textit{User}} & La cuenta de un usuario. \\
\textbf{\textit{Page}} & Información sobre páginas. \\
}

\subsubsection{YouTube}

Introducida en el año 2013, actualmente en su versión \textbf{v3}, y permite la integración de funcionalidades de la plataforma, búsqueda de contenido y análisis demográficos. Cada recurso se representa como un objeto \textit{JSON} sobre el que se pueden ejecutar varias acciones.

Respecto a permisos requeridos, no son tan estrictos como por parte de Meta. No obstante, el número de peticiones se calcula en función de las <<unidades>> que consume cada tipo de petición, teniendo un total básico de 10\,000 al día \cite{youtubeRateLimits}.

Cada recurso se representa como un objeto de datos con identificador único. Entre ellos, los más representativos para la realización de este proyecto podrían ser los expuestos en la \autoref{tabla:recursosYouTube}.

\tablaSmallSinColoresConFuente{Recursos disponibles a través de la \textit{API} de YouTube}{\cite{youtubeAPIreference}}{l l}{recursosYouTube}
{\multicolumn{1}{l}{\textbf{Recurso}} & \multicolumn{1}{l}{\textbf{Descripción}}\\}{
\textbf{\textit{Caption}} & Representa los subtítulos de un vídeo. \\
\textbf{\textit{Comment}} & Comentarios de los objetos. \\
\textbf{\textit{Playlist}} & Colección de vídeos accesibles de forma secuencial. \\
\textbf{\textit{Search result}} & Información de una búsqueda que apunta a un objeto. \\
\textbf{\textit{Video}} & Objeto representativo para un vídeo. \\
}

\subsubsection{Reddit}

Esta \textit{API} fue lanzada en 2011, proporcionando acceso y gestión sobre todas las acciones disponibles desde su interfaz web. También la menos restrictiva de las estudiadas en esta sección, aunque no por ello menos trabajada.

Como ventaja respecto al resto, permite realizar hasta 60 peticiones por minuto \cite{redditRateLimits} a través de todos sus \textit{endpoints}.

Los recursos se representan como objetos tipo \textit{JSON}. En la \autoref{tabla:recursosReddit} se pueden observar las principales estructuras de datos.

\tablaSmallSinColoresConFuente{Recursos disponibles a través de la \textit{API} de Reddit}{\cite{redditAPIreference}}{l l}{recursosReddit}
{\multicolumn{1}{l}{\textbf{Recurso}} & \multicolumn{1}{l}{\textbf{Descripción}}\\}{
\textbf{\textit{Comment}} & Comentario de las demás estructuras de datos. \\
\textbf{\textit{Subreddit}} & Representación de un subforo. \\
\textbf{\textit{Message}} & Información sobre mensajes. \\
\textbf{\textit{Account}} & Datos de la cuenta de un usuario. \\
}

\subsection{Método de extracción}

Para realizar la extracción de los datos se comenzó a realizar un prototipo inicial empleando los \textit{API wrappers} mencionados anteriormente (véase la \autoref{section:api_wrappers}). No obstante, debido a las razones ya explicadas en dicho apartado, finalmente se ha optado por utilizar la herramienta Airbyte para realizar esta tarea.

Esta plataforma de código abierto se ha instalado en la máquina local mediante contenedores \textit{docker}, lo que ha facilitado en gran medida su despliegue ya que está compuesta por una arquitectura compleja con varios servicios interconectados entre sí. Cuenta con una interfaz web sencilla que permite realizar la gestión y configuración de fuentes de datos, destinos de datos y conexiones. En la \autoref{fig:airbyte_architecture} se puede observar una visión general de la arquitectura de esta herramienta.

También presenta una \textit{API} propia~\cite{airbyteAPI} desde la que es posible gestionar las configuraciones de dichos recursos sin necesidad de acceder a su interfaz web.

\imagenConFuente{airbyte_architecture}{Visión general de la arquitectura de \textit{Airbyte}}{\cite{airbyteOverview}}

Inicialmente se comenzó utilizando el conector básico para Twitter que ya presentaba esta herramienta. Tras comprobar su funcionamiento y las posibilidades de extracción de datos que ofrecía, resultó no ofrecer los datos suficientes que se esperaba.

Por ello, al tratarse de una herramienta \textit{open-source}, se procedió a realizar un desarrollo propio y modificar el código base de dicho conector. Se ampliaron así las posibilidades de parametrización y extracción de datos que ofrecía, mejorando así su facilidad de uso y extensibilidad de opciones. Dicho conector está disponible para su uso como una imagen \textit{Docker} que se puede agregar como un nuevo conector en Airbyte, disponible en el siguiente enlace: \url{https://hub.docker.com/r/liviuvj/airbyte-source-twitter/tags}.

\imagen{airbyte_twitter_connector_og}{Configuración básica del conector original de Airbyte para Twitter}

En la \autoref{fig:airbyte_twitter_connector_og} se pueden observar las opciones básicas de configuración del conector, mientras que en la \autoref{fig:airbyte_twitter_connector_custom} se puede comprobar la cantidad de opciones extendidas que se han implementado.

Además, se ha conservado el flujo de datos básico que ya presentaba el conector y se ha añadido un flujo de datos avanzado, en el que se permite la extracción de todas las opciones de configuración documentadas en la \textit{API} de Twitter para la ejecución de consultas y peticiones.

El código de dicho desarrollo se puede comprobar en el \textit{Pull Request} realizado al repositorio oficial de la herramienta Airbyte y su correspondiente \textit{issue} documentada, accediendo al siguiente enlace: \url{https://github.com/airbytehq/airbyte/pull/25534}.

\imagen{airbyte_twitter_connector_custom}{Configuración ampliada del conector mejorado de Airbyte para Twitter}

\subsection{Cambios críticos en las APIs investigadas}

Al comienzo del proyecto se decidió utilizar inicialmente la \textit{API} de Twitter para realizar la extracción de datos. Las razones tras esta decisión se basaron en que presenta una mayor facilidad de uso que el resto de las \textit{APIs} mencionadas, además de que la herramienta de extracción Airbyte ya disponía de un conector básico para ello. Consecuentemente, la siguiente fuente de datos que se había planteado utilizar para el proyecto sería la \textit{API} de Reddit, por presentar mayor facilidad de extracción de datos enfocados a temas concretos que Facebook o Instagram.

 Después de terminar las tareas de integración y mejora del conector de Airbyte para Twitter se dirigió el enfoque hacia las demás etapas del proyecto, dejando así por finalizada esta parte. Tras la inclusión, despliegue y configuración de la herramienta de orquestación de procesos Apache Airflow, se procedió a la realización de pruebas mediante la ejecución completa de la \textit{pipeline ETL} para comprobar el correcto funcionamiento del proyecto, surgiendo así problemas en la etapa de extracción de datos.

Al comprobar el origen de los errores, se descubrieron los cambios críticos que había sufrido recientemente la \textit{API} de Twitter~\cite{karissa2023, stokel2023}. El acceso básico sin coste que existía anteriormente permitía realizar hasta 500\,000 peticiones de manera mensual a una multitud de diversos \textit{endpoints}. Actualmente, el acceso de dicho plan (\textit{Free}) sin coste ha quedado altamente restringido, permitiendo tan solo realizar publicaciones en la propia cuenta del desarrollador. El siguiente plan (\textit{Basic}), que permite el acceso al \textit{endpoint} de búsqueda de \textit{tweets} presenta un coste de \$100 mensuales, limitando a 10\,000 el número de peticiones de lectura que se pueden realizar. El siguiente plan que se aproxima al número original de peticiones es el \textit{Pro}, con un coste de \$5\,000 mensuales y restringiendo el acceso a 300\,000 peticiones de búsqueda.

Al comprobar la siguiente opción investigada inicialmente a utilizar como segunda fuente de datos, se descubrió que la \textit{API} de Reddit también está sufriendo cambios muy restrictivos~\cite{watercutter2023}. Los usuarios y desarrolladores de la plataforma han organizado numerosas protestas~\cite{antonio2023} para intentar evitar estos hechos. El nuevo plan básico y sin coste ofrece entre 10 y 100 peticiones por minuto, dependiendo del tipo de cuenta y acceso concedido.

Estos cambios críticos en las \textit{APIs} fueron detectados \textbf{a lo largo del \textit{Sprint} 10}. Por lo que fue necesaria la toma de una decisión rápida sobre la manera de proseguir con esta parte del proyecto.

\subsection{Conjunto de datos de demostración}

Por las razones explicadas en la sección anterior, las mejoras implementadas en el conector de Airbyte para Twitter se han vuelto poco usables. También se ha descartado el desarrollo de un nuevo conector para Reddit que dependa de la baja cadencia de peticiones posibles a realizar.

Teniendo esto en cuenta y el poco margen temporal restante para la finalización del proyecto, se ha decidido investigar el uso de un posible conjunto de datos a modo de demostración de uso del proyecto, ya que resulta poco viable utilizar las \textit{APIs} investigadas.

El conjunto de datos seleccionado finalmente ha sido \textit{Game of Thrones S8 (Twitter)}~\cite{got8}, de la plataforma de \textit{data science} Kaggle\footnote{\url{https://www.kaggle.com}}. Se trata de un \textit{dataset} sobre la serie de televisión del mismo nombre, que el autor recolectó mediante un \textit{script} en el lenguaje R de manera diaria a lo largo de un mes durante el estreno de su octava temporada.

Este conjunto de datos en formato \textit{CSV} presenta más de 760\,000 registros y 88 atributos distintos, con información tanto sobre los usuarios, como las publicaciones realizadas por los mismos. Lo que resulta de gran utilidad ya que se puede reconstruir una estructura de datos similar a la del primer \textit{data pipeline} realizado mediante la \textit{API} de Twitter.

No obstante, dichos datos son del año 2019 mientras que los obtenidos mediante la \textit{API} de Twitter datan de este año 2023. Al momento de visualizar los datos habrá grandes discrepancias en las gráficas que incluyan un eje temporal (véase la \autoref{fig:date-data-diff}), por lo que estos datos extraídos de la \textit{API} de Twitter no van a dejarse disponibles para su uso, empleando solamente este conjunto de datos investigado a modo de demostración.

\imagen{date-data-diff}{Discrepancia en el eje temporal al utilizar los datos recientes de la \textit{API} de TWitter y los del conjunto de datos de demostración}

Para comprobar de mejor manera los datos disponibles y la usabilidad de los mismos se ha realizado un análisis exploratorio, al que se puede acceder a través del \textit{Jupyter Notebook}\footnote{\url{https://colab.research.google.com/drive/1d_obU9idFqjsDi7ezeFs1CORxTUAgh7V\#offline=true&sandboxMode=true}} que se deja disponible.

La primera parte de dicho análisis consiste en comprobar la posibilidad de crear una estructura de datos parecida a la diseñada originalmente para el flujo de datos de la \textit{API} de Twitter, consiguiendo completar dicho esquema de datos en una gran medida.

La segunda parte se centra en buscar palabras clave que puedan servir a modo de tópico o tema de interés sobre el que se pueda obtener más información mediante el análisis de sentimientos. Para ello, se comprueba el interés de los usuarios sobre diversos personajes y temas, de los que se han escogido los siguientes para formar particiones más pequeñas y manejables del conjuntos de datos total:

\begin{itemize}
    \item \textbf{\texttt{dataset\_movie}.} Partición compuesta de 6\,996 registros que contienen la palabra clave <<\textit{movie}>> en la publicación.
    \item \textbf{\texttt{dataset\_got}.} Partición compuesta de 414\,955 registros que contienen la palabra clave <<\textit{Game of Thrones}>> en la publicación.
    \item \textbf{\texttt{dataset\_season8}.} Partición compuesta de 33\,663 registros que contienen la palabra clave <<\textit{season 8}>> en la publicación.
    \item \textbf{\texttt{dataset\_daenerys}.} Partición compuesta de 8\,830 registros que contienen la palabra clave <<\textit{Daenerys}>> en la publicación.
    \item \textbf{\texttt{dataset\_jon}.} Partición compuesta de 12\,222 registros que contienen la palabra clave <<\textit{Jon}>> en la publicación.
\end{itemize}

\section{Carga de datos}

La carga de los datos se ha realizado en dos bases de datos distintas para diferentes propósitos. La primera se ha utilizado a modo de \textit{Data Lake} del proyecto para guardar los datos en bruto extraídos, mientras que la segunda se ha empleado como un \textit{Data Warehouse} para realizar la agregación de datos de distintos orígenes.

\subsection{MongoDB}

Una vez completada la extracción de los datos, resulta necesario persistirlos para su posterior uso. La herramienta seleccionada para realizar esta tarea es \textit{MongoDB}~\cite{mongodbArchitecture}.

Esta base de datos no relacional basada en documentos almacena los datos en un \textit{JSON} optimizado llamado \textit{BSON}. Teniendo en cuenta que los datos extraídos mediante las \textit{APIs} que se han detallado en el apartado anterior se encuentran en su totalidad en formato \textit{JSON}, su carga en esta base de datos resulta íntegra y directa, eliminando cualquier necesidad de transformación intermedia para ser ajustados a un esquema concreto.

\textit{MongoDB} facilita el desarrollo al ofrecer una alta flexibilidad de almacenamiento de documentos no estructurados con diferentes tipos de datos en una misma colección. Esta característica resulta de gran importancia para el caso de uso del proyecto, debido a que las consultas realizadas a los distintos servicios web no siempre van a poder encontrar toda la información solicitada en los parámetros de la petición.

Presenta también capacidades de alta disponibilidad y escalabilidad gracias a la replicación y particionamiento de los datos (\textit{sharding}), cumpliendo con las partes \textit{C} (Consistencia) y \textit{P} (Tolerante a particiones) del \textit{Teorema CAP}.

Para más información sobre los esquemas de datos utilizados, consultar el apartado correspondiente de la \autoref{section:mongodb_schema}.

\subsection{ClickHouse}

Tras la ejecución de los modelos de \textit{NLP} sobre los datos procesados, estos resultados son almacenados y agregados en el sistema \textit{OLAP} (\textit{On-Line Analytical Processing}) ClickHouse~\cite{clickhouseFeatures}. Este sistema gestor de bases de datos (SGBD) analítico y columnar permite realizar consultas rápidas sobre grandes volúmenes de datos.

Los datos con los que se trabaja en el proyecto son peticiones \textit{API} cuya respuesta contiene una serie de campos que vienen o no completos según la disponibilidad de los datos. En la mayoría de los casos, algunos de los campos presentan poca densidad de información, por lo que en gran parte de registros estos campos estarán vacíos. El modelo de datos columnar que emplea ClickHouse es perfecto para este tipo de casos en los que algunos campos puedan presentar poca densidad.

Al realizar las consultas a la base de datos, se filtran únicamente los campos o columnas necesarias para resolver la petición en lugar de iterar sobre todos los campos de cada fila como en una base de datos tradicional. Por consiguiente, ClickHouse ofrece un alto rendimiento para este tipo de casos ya que se consigue aumentar notablemente la velocidad de las consultas ejecutadas al seleccionar únicamente los campos que intervienen en ellas.

ClickHouse también presenta escalabilidad horizontal, permitiendo así adaptarse ante cargas de trabajo con \textit{Big Data}. Además cuenta con una alta flexibilidad, admitiendo datos en diversos formatos y empleando motores de almacenamiento con funciones específicas por cada tabla, según resulte necesario.

Otro punto ventajoso a la hora de trabajar con este SGBD es la capacidad de integración que presenta. Ofrece interfaz directa con la herramienta de visualización empleada Apache Superset y presenta una sintaxis \textit{SQL} nativa, por lo que resulta de gran facilidad a la hora de lanzar consultas contra los datos.

Para más información sobre los esquemas de datos utilizados, consultar el apartado correspondiente de la \autoref{section:clickhouse_schema}.


\section{Transformación de los datos}

Una vez finalizada la carga inicial de los datos en \textit{MongoDB}, se procede con su procesamiento. Para esta labor, se han empleado dos herramientas descritas en los siguientes apartados.

\subsection{Apache Spark}

Para la primera fase de la etapa de transformación de datos se ha utilizado \textit{Apache Spark}~\cite{apacheSpark} con el lenguaje de programación \textit{Scala}, ya que permite una gran velocidad de cómputo ideal al trabajar con grandes cantidades de datos. Además presenta las características perfectas para \textit{Big Data}, su arquitectura es escalable y el procesamiento se puede realizar de manera distribuida.

Las tareas de procesamiento contemplan la lectura de los datos en bruto y una posterior limpieza inicial de los mismos. De esta manera, se recogen únicamente los datos necesarios y se elimina la compleja estructura de datos de la que se han extraído. A continuación, se seleccionan los campos que serán exportados nuevamente a una colección de \textit{MongoDB} de datos limpios. Los esquemas de datos antes y después de este procesamiento se pueden comprobar en la \autoref{section:mongodb_schema}.

Estos datos procesados estarán ya listos para servir de entrada a los algoritmos de \textit{sentiment analysis} que serán ejecutados a continuación.

\subsection{HuggingFace Transformers}

Para la segunda fase de la etapa de transformación de datos se ha utilizado la librería \textit{HuggingFace Transformers}~\cite{huggingfaceTransformers} del lenguaje de programación \textit{Python}. Esta librería ha facilitado la ejecución de las técnicas de procesamiento de lenguaje natural gracias a su interfaz unificada para la carga y ejecución de \textit{LLM} (\textit{Large Language Models}) pre-entrenados.

Más concretamente, se han enriquecido los datos procesados gracias a la inferencia de varias tareas mediante variaciones del modelo \textit{BERT} ya explicado en la \autoref{section:theory_nlp}. A continuación, se describen las distintas tareas \textit{NLP} empleadas y sus respectivos modelos empleados:

\begin{itemize}
    \item \textbf{\textit{Sentiment Analysis}.} Consiste en identificar y extraer la opinión o actitud de una persona o entidad hacia un tema. Las posibles categorías de sentimiento son: positivo, neutro o negativo. El modelo empleado: \texttt{Twitter-roBERTa-base for Sentiment Analysis}~\cite{cardiffnlpSentiment,  camacho2022tweetnlp, loureiro2022timelms}.

    \item \textbf{\textit{Emotion Analysis}.} Consiste en reconocer y clasificar las emociones expresadas por una persona en un texto. Las posibles categorías de emoción son: optimismo, alegría, tristeza, enfado. El modelo empleado: \texttt{Twitter-roBERTa-base for Emotion Recognition}~\cite{cardiffnlpEmotion, camacho2022tweetnlp, loureiro2022timelms}

    \item \textbf{\textit{Topic Classification}.} Consiste en clasificar en una o más categorías a un enunciado según su contenido y contexto. El modelo empleado: \texttt{tweet-topic-21-multi}~\cite{cardiffnlpTopic, antypas2022twitter}. Las posibles categorías son las mostradas a continuación:
    \begin{verbatim}
    0: arts_&_culture           1: business_&_entrepreneurs
    2: celebrity_&_pop_culture  3: diaries_&_daily_life
    4: family                   5: fashion_&_style
    6: film_tv_&_video          7: fitness_&_health
    8: food_&_dining            9: gaming
    10: learning_&_educational  11: music
    12: news_&_social_concern   13: other_hobbies
    14: relationships           15: science_&_technology
    16: sports                  17: travel_&_adventure
    18: youth_&_student_life		
    \end{verbatim}

    \item \textbf{\textit{Named Entity Recognition}.} Consiste en localizar y etiquetar las entidades nombradas que aparecen en un texto, como personas, lugares, organizaciones, etc. El modelo empleado:\\\texttt{tner/twitter-roberta-base-dec2021-tweetner7-all}~\cite{tnerNER, ushio2022t, ushio2022named}. Las posibles categorías de entidad son las mostradas a continuación:
    \begin{verbatim}
    0: corporation      1: creative_work    2: event
    3: group            4: location         5: person
    6: product
    \end{verbatim}
    
\end{itemize}


\section{Visualización de los datos}

En esta sección se detallarán los aspectos relevantes de la herramienta seleccionada para la visualización de los datos. Apache Superset~\cite{apacheSuperset} permite a la exploración y visualización de datos, además de compartir información de forma interactiva y colaborativa.

Presenta capacidades de creación de usuarios y permisos, por lo que es posible designar roles específicos para cada usuario según las necesidades del mismo. De esta manera, se consigue implementar cierta seguridad en los datos, de modo que ciertos usuarios podrían interactuar solamente con ciertos \textit{dashboards} según el nivel de acceso que tengan, en caso de trabajar con datos sensibles.

Además, cuenta con una sección llamada <<\textit{SQL Lab}>> en la que es posible realizar consultas en formato \textit{SQL} nativo sobre los distintos conjuntos de datos cargados. Una vez ejecutadas las consultas, estas se pueden exportar a un \textit{dataset} virtual que se puede tomar como base para la creación de nuevas visualizaciones.

En este proyecto se ha desarrollado un cuadro de mando integral que comprende 5 vistas a distinto nivel de detalle sobre los datos recogidos. Para ello, se han empleado las visualizaciones pertinentes junto a una paleta de colores elegida intencionalmente para representar el análisis de sentimientos.

Las vistas del \textit{dashboard} permiten la obtención de \textit{insights} no solo en el ámbito general mediante métricas sobre los datos extraídos, sino también sobre las estadísticas representadas de las publicaciones realizadas y los usuarios de las mismas, junto a la información enriquecida añadida mediante las técnicas \textit{NLP}. 

Para utilizar los datos almacenados en el sistema \textit{OLAP} ClickHouse ha hecho falta la instalación de un conector adicional. Esta configuración, así como el cuadro de mando desarrollado y la importación de los \textit{datasets} desde ClickHouse, se realiza mediante un \textit{script} automático para facilitar el despliegue para el usuario. 

El diseño e implementación de las vistas para el \textit{dashboard} se pueden comprobar en mayor detalle en la \autoref{section:dashboard-design}.


\section{Orquestación de los procesos}

La arquitectura implementada en este proyecto resulta compleja, no solamente a nivel de integración de las herramientas empleadas, sino también de la comunicación realizada entre ellas y los procesos que intervienen entre cada una.

Debido a estas razones, surge la necesidad de emplear una herramienta capaz de gestionar todo el flujo de acciones que se lleva a cabo entre los distintos componentes del proyecto. Para lograr este objetivo, se ha seleccionado \textit{Apache Airflow}~\cite{apacheAirflow} como orquestador de procesos.

En las siguientes secciones se desarrollan los aspectos que mayor influencia han tenido para la integración de esta etapa.

\subsection{Configuración y despliegue}

La documentación de \textit{Apache Airflow} indica de la disponibilidad de un <<entorno \textit{dockerizado} listo para producción>>, aunque dicha afirmación no resulta del todo cierta. Para la correcta configuración y despliegue de esta herramienta ha resultado necesaria la creación de un \textit{script} que automatice de la manera más abstracta posible para el usuario la parametrización y despliegue iniciales de la plataforma.

Este \textit{script} se encarga de la declaración de las variables y conexiones a las demás herramientas necesarias para la ejecución de las \textit{data pipelines} creadas. También crea y configura la clave de encriptación \textit{Fernet}~\cite{fernetKey} y realiza la instalación de los \textit{plugins} empleados (\texttt{apache-airflow-providers-docker}~\cite{airflowDocker} y \texttt{apache-airflow-providers-airbyte}~\cite{airflowAirbyte}), que se han utilizado para, respectivamente, realizar la integración con el servicio \textit{Docker} y ejecutar la sincronización de datos en Airbyte.

\subsection{Acceso seguro al \textit{Docker socket}}

Teniendo en cuenta los objetivos del proyecto sobre la creación de una plataforma segura y autocontenida, se ha realizado el despliegue completo de la plataforma en contenedores \textit{Docker}.

Debido a esto, las etapas de procesamiento e inferencia de los datos se realizan en contenedores <<desechables>>, en el sentido de que solamente resulta necesario que estén activos durante el tiempo necesario para realizar sus operaciones. De esta manera, esta parte del flujo de datos se vuelve más dinámica y eficiente, utilizando únicamente los recursos necesarios durante el tiempo requerido y liberándolos nuevamente tras finalizar dichas tareas.

Para realizar dicho despliegue y borrado dinámico de contenedores, \textit{Apache Airflow} necesita acceso al \textit{Docker socket}. No obstante, conceder acceso a este \textit{socker} significa otorgar permisos de administrador local sobre la propia máquina en la que se despliega. Para solventar este problema y securizar el \textit{socket}, se ha desplegado un \textit{proxy}~\cite{dockerProxy} también \textit{dockerizado} y configurable que permite el paso de peticiones benignas y bloquea las malignas.

\subsection{Integración y flujos de datos}

La orquestación de los procesos y comunicación entre los numerosos componentes del proyecto ha requerido la creación de distintos flujos de datos utilizando diversas tareas y configuraciones.

Teniendo en cuenta la cantidad de fuentes de datos seleccionadas finalmente para el proyecto, se han diseñado dos \textit{data pipelines}. Un flujo de datos con un \textit{DAG} (\textit{Directed Acyclic Graph}) normal para los datos provenientes de la \textit{API} de Twitter y otro flujo de datos que genera de manera dinámica los \textit{DAGs} necesarios para todos los subconjuntos de datos del \textit{dataset} explicado en anteriores apartados. No obstante, en ambos flujos de datos han intervenido las mismas tareas y en el mismo orden de ejecución, como se puede observar en la \autoref{fig:airflow-data-pipeline}.

\imagen{airflow-data-pipeline}{Tareas de la \textit{data pipeline} sobre el \textit{dataset} de demostración}

\begin{itemize}
    \item La primera tarea (\texttt{sync\_data}) es un actuador que se encarga de ejecutar la sincronización de los datos en Airbyte. Para ello, es necesario establecer previamente la conexión con la herramienta de extracción de datos desde la configuración de Airflow y especificar el identificador de la conexión a ejecutar.

    \item La segunda tarea (\texttt{sensor\_sync\_data}) es un sensor que se encarga de recibir la señal con el estado de la sincronización cuando Airbyte finaliza la extracción de datos. Según el estado de la señal, finalización satisfactoria o no, termina la \textit{pipeline} o continúa con la siguiente tarea.

    \item La tercera tarea (\texttt{process\_data}) es un actuador que se encarga de realizar el procesamiento de los datos mediante Apache Spark. Los parámetros de configuración necesitan los datos de conexión a la base de datos MongoDB, junto al término de búsqueda empleado para las consultas de extracción y en qué colección se están almacenando estos datos en bruto. Esta tarea levanta un contenedor \textit{Docker} con Apache Spark para realizar sus operaciones, que posteriormente es eliminado para liberar los recursos ocupados. Durante la ejecución de esta fase se puede acceder a la interfaz web de monitorización que provee Spark.

    \item La cuarta tarea (\texttt{nlp\_inference}) es otro actuador que se encarga de realizar la inferencia de los modelos \textit{NLP} (\textit{Natural Language Processing}). Los parámetros de configuración necesitan los datos de conexión a las bases de datos MongoDB y ClickHouse, junto al término de búsqueda empleado para las consultas de extracción. Esta tarea levanta un contenedor \textit{Docker} con la librería \textit{HuggingFace Transformers} y emplea los modelos descritos en la sección anterior para realizar la inferencia de los datos. Tras su finalización, los datos son agregados en ClickHouse y el contenedor eliminado para liberar los recursos utilizados.
\end{itemize}


\section{Interfaz web de acceso centralizado}

Como se ha explicado en los apartados anteriores, la mayoría de los componentes utilizados para crear esta plataforma poseen una interfaz web que permite monitorizar y gestionar el servicio correspondiente.

La modularidad que ofrece esta arquitectura invita a posibles extensiones de funcionalidades o a la incorporación de distintas herramientas o nuevos servicios. Esto implica también la posibilidad de que haya más interfaces web a las que acceder según con qué parte de la plataforma se quiera trabajar, en caso de que se tenga los permisos suficientes para ello.

Por estas razones, se ha decidido crear un punto único de acceso centralizado a todas las interfaces de gestión y monitorización en forma de página web. De esta manera, se agiliza el acceso al resto de interfaces que ofrece la plataforma actualmente en caso de que se necesite trabajar de manera paralela en varios puntos de la misma.

El desarrollo de la página web (véase la \autoref{fig:web-app}) se ha realizado mediante el \textit{framework Flask} con el objetivo de que esta interfaz sea una aplicación web ligera con las funcionalidades básicas y necesarias. El diseño se ha realizado a medida, \textit{responsive} (adaptable a cualquier dispositivo) y teniendo en cuenta un estilo minimalista acorde a las necesidades de un punto web de acceso centralizado.

\imagen{web-app}{Aplicación web desarrollada como punto de acceso centralizado}

\capitulo{6}{Trabajos relacionados} \label{section:related_works}

En este apartado se describirán otras herramientas similares ya existentes que cumplen un propósito similar al planteado en este proyecto. También se escribirá sobre los principales artículos científicos que comprenden el \textit{state-of-the-art} relacionado con las técnicas de procesamiento de lenguaje natural utilizadas.

\section{Herramientas similares}

A continuación se detallan las características de las principales plataformas que existen actualmente que cumplen con funciones similares a las planteadas en este proyecto. Se han categorizado según supongan o no algún coste económico para su utilización.

\subsection{Herramientas de pago}

Comenzando con las opciones de pago, por ser más establecidas y conocidas que las gratuitas.

\subsubsection{Brand24}

Es una plataforma\footnote{\url{https://brand24.com/}} que monitoriza las menciones sobre la marca del cliente tanto en la web como en redes sociales. Utiliza técnicas NLP para analizar en tiempo real los datos de diversas fuentes como blogs, foros, redes sociales, vídeos...

Una de las ventajas competitivas que ofrece es su capacidad de mostrar la influencia que ha tenido cada mención. Como desventaja, cabe destacar el limitado número de menciones que permite monitorizar en sus servicios de suscripción. El rango de precios comprende desde los \$49 mensuales del paquete básico hasta los \$348 del paquete ejecutivo.

\subsubsection{MonkeyLearn}

Es un conjunto de herramientas de análisis de texto que permite crear modelos propios de \textit{machine learning} sobre los datos introducidos, empleando la propia interfaz gráfica de la plataforma.

Como ventaja principal, provee unos modelos ya entrenados que se pueden utilizar en la mayoría de las situaciones, pero permite también entrenarlos sobre los datos específicos que interesen al cliente. Como desventajas, se podrían incluir la manera de establecer la conexión con los datos, puesto que necesita acceso directo a la base de datos del cliente, además de requerir una suscripción mensual de \$299.

\subsubsection{Repustate}

Es una herramienta de análisis\footnote{\url{https://www.repustate.com/}} de sentimientos que analiza de manera sintáctica los datos introducidos para poder evaluar de mejor manera la intención de cada texto. También es capaz de analizar \textit{emojis} según el contexto en el que se utilicen y provee una API que da soporte a 23 idiomas distintos.

Las principales ventajas que ofrece son la gran cantidad de idiomas que soporta y la posibilidad de especificar distintos significados de palabras concretas para mejorar el análisis que realiza. Como principal desventaja, la utilización de este servicio requiere una suscripción mensual de \$199 para su plan \textit{Standard} o \$499 para el \textit{Premium}.

\subsection{Herramientas gratuitas}

A continuación, las opciones que no requieren realizar gasto económico alguno para utilizar sus funcionalidades básicas.

\subsubsection{Social Searcher}

Es una herramienta sencilla\footnote{\url{https://www.social-searcher.com/}} que ofrece búsqueda por palabras clave, etiquetas o usuarios y muestra unos análisis básicos sobre los resultados obtenidos. Muestra un \textit{dashboard} con varias pestañas en las que se realizan distintos tipos de análisis, además de gráficos diversos que categorizan las menciones en temas y clasifican las opiniones de los usuarios.

La principal ventaja de esta herramienta es que permite aprovechar sus servicios de manera gratuita y sin límite de consultas, aunque tenga también planes de pago. Como desventaja, las funcionalidades que ofrece la versión gratuita son bastante básicas. 

\subsubsection{Tweet Sentiment Viz}

Esta herramienta es la más básica\footnote{\url{https://www.csc2.ncsu.edu/faculty/healey/tweet_viz/tweet_app/}} de la lista. Muestra una serie de gráficos exploratorios (temas, mapas de calor, nubes de palabras, etc.) sobre los datos buscados en tiempo real en función de palabras clave.

Como principal ventaja, es que funciona bastante bien dentro de unos límites preestablecidos. Entre sus desventajas, esta herramienta analiza únicamente datos de la plataforma Twitter, además de emplear técnicas de bolsas de palabras. Por lo que tendrá dificultades a la hora de interpretar cualquier palabra utilizada que no esté dentro de dichos diccionarios.

\section{Artículos científicos}

A continuación, se detallan los artículos científicos que más relevancia han tenido en relación a los objetivos establecidos para este proyecto.

\subsection{BERT: Bidirectional Encoder Representations from Transformers}

Se trata de un modelo~\cite{devlin2019bert} que utiliza una red neuronal ya entrenada para generar \textit{word embeddings} que son utilizadas posteriormente como características en modelos \textit{NLP}.

\textit{BERT} se basa en \textit{transformers} (mecanismos de atención que <<aprenden>> correlaciones entre las palabras de un texto). Estos \textit{transformers} presentan dos componentes, un \textit{encoder} que procesa los datos de entrada y un \textit{decoder} que se encarga de realizar las predicciones correspondientes. Sin embargo, como el objetivo es construir un modelo de lenguaje, tan solo hace falta la primera parte de estos, el codificador.

Mientras que los modelos hasta el momento tomaban una dirección de lectura secuencial de los datos (bien de izquierda a derecha o bien al revés), el codificador del \textit{transformer} es capaz de leer cada palabra del texto a la vez. Esto permite al modelo analizar el contexto general en el que se presenta cada palabra y no teniendo en cuenta solamente una dirección. De esta manera, se considera un modelo <<bidireccional>>, aunque en realidad no tenga una dirección como tal.

Generalmente, los modelos de lenguaje se entrenan intentando predecir una secuencia de palabras dentro de un texto, lo que los convierte en unidireccionales. Por ello, \textit{BERT} emplea dos estrategias para mantener su habilidad bidireccional:

\begin{itemize}
    \item \textbf{\textit{Masked LM (MLM)}.} La primera estrategia que se utiliza es ocultar, mediante un \textit{token}, a forma de máscara aproximadamente un 15\% de las palabras del texto de entrada del codificador. Posteriormente, el modelo intentará predecir las palabras que faltan basándose en el contexto que las rodea.
    
    \item \textbf{\textit{Next Sentence Prediction (NSP)}.} La segunda estrategia consiste en entrenar el modelo mediante pares de frases. La mitad de los datos de entrada se divide de tal manera que la segunda frase de cada par es la que va a continuación de la primera frase en el texto original. Mientras que en la otra mitad de los datos, la segunda frase se escoge al azar del texto original. De esta manera, se asume que el modelo será capaz de distinguir correctamente qué frase tiene sentido a continuación de otra. Se utilizan una serie de \textit{tokens} para indicar el inicio y final de cada frase.
\end{itemize}

Ambas estrategias se ponen en práctica y se entrenan a la vez para conseguir minimizar la <<función de pérdida>> o \textit{loss function} del modelo.
\capitulo{7}{Conclusiones y Líneas de trabajo futuras}

En esta sección final se desarrollarán las conclusiones derivadas del desarrollo de este proyecto, así como los resultados obtenidos en comparación con los objetivos planteados inicialmente. También se incluye un listado de posibles mejoras a implementar de cara a líneas de trabajo futuras.

\section{Conclusiones} \label{section:conclusions}

Este proyecto ha sido el resultado de los conocimientos aprendidos hasta el momento y de los obtenidos a medida que se estaba desarrollando esta <<Plataforma \textit{Big Data} para \textit{Sentiment Analysis}>>.

A continuación, se destacan las conclusiones obtenidas tras la realización del proyecto:

\begin{itemize}
    \item En general, se han podido cumplir la gran mayoría de los objetivos propuestos inicialmente para el proyecto. Se ha desarrollado e integrado una plataforma modular, escalable y capaz de trabajar con \textit{Big Data} utilizando únicamente tecnologías de código abierto.

    \item Se han diseñado e implementado diversas \textit{data pipelines} mediante el proceso \textit{ETL} (\textit{Extract, Transform, Load}) que han necesitado de la correcta cooperación entre diversas tecnologías. Para ello, ha hecho falta la correcta integración y configuración de los diversos componentes que han intervenido en estos flujos de datos.

    \item Respecto a la herramienta de extracción de datos Airbyte, se ha conseguido mejorar y extender las funcionalidades que presentaba el conector base de Twitter. El desarrollo de la mejora de este conector se ha hecho público para la comunidad en el repositorio oficial de la herramienta.

    \item Para la utilización de las técnicas de procesamiento de lenguaje natural (\textit{NLP}) se han empleado modelos que forman parte del \textit{state-of-the-art} en este campo, más concretamente, las distintas variantes del modelo \textit{BERT}. La investigación realizada sobre dichos modelos ha ayudado a comprender mejor el funcionamiento de los modelos \textit{Transformer} y de las tendencias actuales del \textit{Deep Learning} en general.

    \item Se ha conseguido el despliegue de la plataforma completa mediante contenedores \textit{docker}. Esto la convierte en una solución integral para el análisis de sentimientos en \textit{Big Data}, además de una solución modular. Por consiguiente, se posibilita la agregación de nuevos componentes para extender las funcionalidades de la misma y el intercambio de unos componentes por otros, en caso de necesitar emplear tecnologías diferentes o adaptarse a distintos casos de uso.

    \item Se ha llevado a cabo el diseño e implementación de un \textit{dashboard} que comprende 5 vistas distintas en las que se muestra la información procesada a través de la \textit{pipeline ETL} a diferente nivel de detalle. El cuadro de mando creado permite la obtención de nuevos \textit{insights} mediante varias visualizaciones, que muestran los sentimientos de los usuarios respecto al tema inquirido.

    \item El desarrollo de esta plataforma empleando las tecnologías seleccionadas ha supuesto un grado de aprendizaje nada trivial. Cada herramienta presentaba numerosos conceptos necesarios para utilizarlas correctamente, además de la complejidad inherente que podía darse en cada una de ellas. No obstante, se ha obtenido una gran satisfacción por todos los conocimientos aprendidos, ya que resultarán útiles para futuros proyectos.

    \item Este proyecto ha resultado de gran complejidad por la cantidad de tecnologías empleadas en su desarrollo, ya que el número de componentes utilizados se podría haber reducido. Esta decisión se ha tomado con el objetivo de afianzar los conocimientos obtenidos hasta el momento sobre las distintas áreas que ha cubierto el Máster y, al mismo tiempo, aprender nuevas tecnologías modernas capaces de cubrir todo el proceso \textit{ETL}. Con los resultados finales obtenidos, este proceso de aprendizaje ha resultado satisfactorio.

    \item Finalmente, cabe destacar que la realización de este Trabajo de Fin de Máster se ha realizado compaginando tanto la jornada laboral de trabajo como el estudio de las asignaturas del propio Máster. Esto ha supuesto una gran carga de trabajo y limitaciones de tiempo, lo que ha necesitado de una capacidad de organización para la priorización de ciertas tareas y partes del proyecto con el objetivo de finalizar este trabajo final de manera satisfactoria.

\end{itemize}

\section{Líneas de trabajo futuras} \label{section:future_works}

A continuación, se listan una serie de posibles líneas de mejora o aspectos con los que se podría continuar el desarrollo de este proyecto:

\begin{itemize}
    \item Actualmente, la plataforma desarrollada se ha enfocado en trabajar con datos en \textit{batch}. Por ello, la ejecución del proceso \textit{ETL}, en especial la etapa de Extracción y la fase de inferencia mediante \textit{NLP} de la etapa de Transformación, conllevan cierta cantidad de tiempo para su finalización.
    
    Una de las posibles mejoras futuras para esta plataforma sería implementar la posibilidad de trabajar también con datos en flujo (\textit{streaming}), para los casos de uso en los que se quiera obtener la información de inmediato en lugar de esperar a la ejecución periódica de los \textit{data pipelines}.

    \item El funcionamiento desarrollado para la fase de inferencia mediante técnicas \textit{NLP} se realiza en las \textit{data pipelines} de manera consecutiva, ejecutando todas las tareas de procesamiento de lenguaje natural una a una sobre los datos. La herramienta Apache Airflow permite la ejecución de tareas en paralelo mediante su programación en \textit{DAGs} (\textit{Directed Acyclic Graph}). El código desarrollado para las \textit{data pipelines} se ha diseñado de tal manera que permita la inclusión de estos cambios para su generación de manera dinámica.
    
    Por lo que otra posible línea de mejora consistiría en realizar la ejecución de cada tarea \textit{NLP} en paralelo, necesitando posteriormente la agregación de los datos inferidos de cada tarea con sus respectivos registros de datos. Esto conllevaría un aumento notable en la velocidad de ejecución de las \textit{data pipelines}, debido a que esta fase es la que supone una mayor carga de trabajo en la plataforma.

    \item Como se ha mencionado en la sección anterior, la arquitectura de la plataforma está compuesta por numerosas tecnologías. Este diseño se podría reducir y emplear una menor cantidad de componentes en caso de que se quiera aligerar la carga de trabajo en el servidor de despliegue, por ejemplo, eliminando el uso de algunos componentes y unificando su funcionalidad en una sola tecnología o herramienta.

    \item En contraste con el punto anterior, se podría también sugerir añadir un componente más. Al trabajar con \textit{Big Data}, y más concretamente con opiniones de personas, resulta necesario mantener la seguridad de los datos. Para ello, se podría sugerir la implementación de una herramienta para llevar a cabo la <<gobernanza>> de los datos y asignar los permisos necesarios a cada usuario según el uso que se necesite dar a los datos.

    \item Finalmente, otro posible punto de mejora en el ámbito de la seguridad. La mayoría de las herramientas empleadas se han protegido con acceso restringido mediante credenciales básicos de autenticación. Sin embargo, y ya que las herramientas utilizadas lo permiten, se podría ir un paso más en esta línea y realizar la integración de estas tecnologías con un sistema \textit{LDAP} (Lightweight Directory Access Protocol), mejorando así la seguridad de autenticación en la plataforma.

\end{itemize}



%\renewcommand\chaptername{Anexo}
%\renewcommand\thechapter{\Roman{chapter}}
%\setcounter{chapter}{0}

% Añadir entrada en el índice: Anexos
\appendix
\addcontentsline{toc}{part}{Apéndices}
\part*{Apéndices}

\apendice{Plan de Proyecto Software}

\section{Introducción}

En las siguientes secciones se realizará un estudio de la planificación temporal seguida durante el desarrollo de este proyecto, además de la viabilidad tanto económica como legal que podría llegar a suponer este trabajo.

Debido a la naturaleza inherente del proyecto, al no tratarse de un \textit{software} típicamente tradicional sino más bien centrado hacia la investigación e implementación de modelos de \textit{machine learning}, no ha resultado sencillo llevar a cabo algunas de las buenas prácticas y conceptos normales de acuerdo a un ``Plan de Proyecto Software'' tradicional.

\section{Planificación temporal}

La planificación del proyecto se ha llevado a cabo mediante la metodología de desarrollo ágil \textit{Scrum}. A continuación se realiza un desglose de los distintos \textit{Sprints} llevados a cabo.

Inicialmente, se presentan las tareas correspondientes a cada iteración del trabajo y su duración inicial estimada. Posteriormente, se realiza una comparación entre el tiempo total estimado y el real gastado mediante la ilustración de gráficos \textit{burn-down}.

\subsection{\textit{Sprint} 0 (01/02/2023 - 15/02/2023)}

Este Sprint inicial será dedicado a la preparación del entorno de trabajo para el proyecto. Se elegirán las herramientas con las que se trabajará en algunas de las etapas del proyecto, se investigarán técnicas y librerías a utilizar, se realizarán unas pruebas concepto iniciales y se comenzará la labor de documentación.

\begin{itemize}
    \item \textbf{Gestión del \textit{Sprint} (4h).}  Se realizará el planteamiento de las tareas a llevar a cabo a lo largo de este sprint y se documentarán en el apartado \textbf{A.2 - Planificación temporal} del \textbf{Apéndice A - Plan de Proyecto Software} de los anexos del proyecto.

    \item \textbf{Elegir IDE (2h).} Para la realización de este proyecto será necesaria la utilización de diversos lenguajes de programación, por lo que la elección de un entorno de desarrollo integrado adecuado resultará de gran ayuda.

    \item \textbf{Estudiar guía \LaTeX (2h).} Como objetivo para la generación de la memoria del proyecto, se va estudiar una guía sobre \LaTeX con el fin de recordar los conocimientos necesarios para poder crear la documentación correspondiente.
 
    \item \textbf{Documentación de la memoria - Técnicas y herramientas (4h).} Comenzar con la documentación de la memoria del proyecto, con la sección ``Técnicas y herramientas''. De manera inicial, se documentará lo siguiente:

    \begin{itemize}
        \item \textbf{Técnicas}
        \begin{itemize}
            \item \textit{Scrum}
            \item \textit{Natural Language Processing}
            \item \textit{Sentiment Analysis}
        \end{itemize}
        \item \textbf{Herramientas}
        \begin{itemize}
            \item GitHub
            \item ZenHub
            \item Overleaf
            \item Joplin
            \item Super Productivity
        \end{itemize}
    \end{itemize}
    
    \item \textbf{Documentación de la memoria - Trabajos relacionados (4h).} La siguiente parte de la memoria que se va a redactar será la sección de ``Trabajos relacionados''. En este apartado se describirán otras herramientas similares ya existentes que cumplen un propósito similar al planteado en este proyecto.
    
    También se escribirá sobre los principales artículos científicos que comprenden el \textit{state-of-the-art} relacionado con las técnicas de procesamiento de lenguaje natural que serán utilizadas.

    \item \textbf{Investigar y probar recursos NLP ya existentes (8h).} Ya que inicialmente no se prevé el desarrollo de un algoritmo NLP propio, se investigará el \textit{state-of-the-art} sobre análisis de sentimientos y se comprobará si existen recursos ya implementados para utilizar en el proyecto.

\end{itemize}


\section{Estudio de viabilidad}

\subsection{Viabilidad económica}

\subsection{Viabilidad legal}



\apendice{Especificación de Requisitos}

\section{Introducción}

En las secciones siguientes se detallan lo objetivos propuestos para la realización de este proyecto, su descomposición y sus respectivos requisitos específicos.

\section{Objetivos generales}

\begin{itemize}
    \item Realizar extracciones de datos de dominio público.

    \item Llevar a cabo tareas de procesamiento, limpieza y asegurar la calidad de los datos.

    \item Utilizar técnicas de procesamiento de lenguaje natural para enriquecer los datos.

    \item Diseñar e implementar cuadros de mando interactivos sobre la información obtenida.

    \item Aplicar procesos de Extracción, Transformación y Carga (\textit{ETL}).

    \item Diseñar y desarrollar una plataforma \textit{open-source} para análisis de sentimiento sobre \textit{Big Data}.
\end{itemize}

\section{Catalogo de requisitos}

A continuación, se detallan los requisitos funcionales y no funcionales.

\subsection{Requisitos funcionales}

\begin{itemize}

    \item \textbf{RF-1 Realizar la extracción de datos.} Para la obtención de los datos será necesario:
    \begin{itemize}
        \item \textbf{RF-1.1} Configurar una fuente de datos.
        \item \textbf{RF-1.2} Configurar un destino de datos.
        \item \textbf{RF-1.3} Configurar una conexión de sincronización entre el origen y destino de los datos.
    \end{itemize}

    \item \textbf{RF-2 Realizar el procesamiento de los datos.} Para ello se utilizará la herramienta Apache Spark, que deberá permitir:
    \begin{itemize}
        \item \textbf{RF-2.1} Cargar los datos en bruto.
        \item \textbf{RF-2.2} Realizar las operaciones de procesamiento, limpieza y asegurar la calidad de los datos.
        \item \textbf{RF-2.3} Guardar los datos limpios.
    \end{itemize}

    \item \textbf{RF-3 Ejecutar tareas de procesamiento de lenguaje natural.} Para ello se utilizarán modelos \textit{NLP} (\textit{Natural Language Processing}) pre-entrenados que deberán ser capaces de:
    \begin{itemize}
        \item \textbf{RF-3.1} Cargar los datos limpios.
        \item \textbf{RF-3.2} Ejecutar las tareas \textit{NLP} y enriquecer los datos con análisis de sentimientos.
        \item \textbf{RF-3.3} Guardar los datos enriquecidos.
    \end{itemize}

    \item \textbf{RF-4 Diseñar e implementar \textit{dashboards} interactivos}. Para ello, la herramienta de visualización de datos deberá hacer posible lo siguiente:
    \begin{itemize}
        \item \textbf{RF-4.1} Cargar los datos enriquecidos.
        \item \textbf{RF-4.2} Crear visualizaciones a partir de la información obtenida.
        \item \textbf{RF-4.3} Componer cuadros de mando mediante los gráficos creados.
    \end{itemize}

\end{itemize}

\subsection{Requisitos no funcionales}

\begin{itemize}
    \item \textbf{RNF-1 Usabilidad.} Los componentes empleados en el proyecto deberán resultar sencillos de utilizar por el usuario final según su \textit{background} y rol. 

    \item \textbf{RNF-2 Escalabilidad.} La plataforma desarrollada deberá ser escalable para soportar grandes volúmenes de datos.
    
    \item \textbf{RNF-3 Modularidad.} Los componentes de la plataforma deberán poder intercambiarse con otros de funciones similares, según los casos de uso establecidos.
    
    \item \textbf{RNF-4 Configurabilidad} La configuración de las tecnologías utilizadas deberá resultar sencilla de modificar.
\end{itemize}


\section{Especificación de requisitos}

En esta sección se detalla el diagrama de casos de uso, las descripciones de los mismos y los requisitos funcionales que forman parte de cada caso de uso en concreto.

\imagen{use-cases}{Diagrama de casos de uso}


\vspace{2cm}

\subsection{Casos de uso}


\casoDeUso{CU-01}{Configurar una conexión de sincronización de datos}{RF-1, RF-1.3}
{La herramienta de extracción de datos debe permitir la configuración de conexiones entre origen y destino para ejecutar la sincronización de los datos y realizar su ingestión en la base de datos.}
{Se han creado y configurado la fuente de datos y el destino de datos de la conexión.}
{
\item El usuario abre la interfaz web de Airbyte.
\item El usuario especifica los parámetros de configuración de la conexión entre origen y destino de datos.
\item El usuario ejecuta la sincronización de los datos.
}
{Los datos extraídos se encuentran correctamente almacenados en el destino de datos.}
{
\item La configuración del origen de datos presenta fallos.
\item La configuración del destino de datos presenta fallos.
}
{Alta}


\casoDeUso{CU-02}{Realizar procesamiento de los datos}{RF-2, RF-2.2}
{La herramienta de procesamiento de datos debe permitir la ejecución de tareas de limpieza de los datos y asegurar la calidad de los mismos.}
{Los datos en bruto se encuentran cargados en la base de datos.}
{
\item El usuario programa las tareas de procesamiento de datos.
\item El usuario ejecuta las tareas de procesamiento.
\item El usuario recibe el \textit{feedback} con los registros procesados.
}
{Los datos en bruto se han procesado correctamente y almacenado nuevamente.}
{
\item La herramienta de procesamiento no puede establecer conexión con la base de datos.
\item Las tareas de procesamiento presentan algún fallo y la validación del esquema de datos falla.
}
{Alta}


\casoDeUso{CU-03}{Ejecutar tareas de procesamiento de lenguaje natural}{RF-3, RF-3.2}
{Durante la etapa de transformación del proceso \textit{ETL} se debe poder realizar la inferencia de modelos \textit{NLP} (\textit{Natural Language Processing}) sobre los datos procesados para enriquecerlos mediante análisis de sentimientos.}
{Los datos procesados se encuentran cargados en la base de datos.}
{
\item El usuario programa las tareas de procesamiento de lenguaje natural.
\item El usuario ejecuta las tareas de inferencia.
\item El usuario recibe el \textit{feedback} con los registros inferidos.
}
{Los datos procesados se han enriquecido con el análisis de sentimientos y se han almacenado nuevamente en la base de datos \textit{OLAP} (\textit{On-Line Analytical Processing}).}
{
\item La herramienta no puede establecer conexión con la base de datos.
\item Algún registro no se ha procesado correctamente y supera el límite de caracteres de entrada a los modelos \textit{NLP}.
}
{Alta}

\casoDeUso{CU-04}{Diseñar e implementar \textit{dashboards} interactivos}{RF-4, RF-4.3}
{Durante la etapa de transformación del proceso \textit{ETL} se debe poder realizar la inferencia de modelos \textit{NLP} (\textit{Natural Language Processing}) sobre los datos procesados para enriquecerlos mediante análisis de sentimientos.}
{Los datos procesados se encuentran cargados en la base de datos.}
{
\item El usuario programa las tareas de procesamiento de lenguaje natural.
\item El usuario ejecuta las tareas de inferencia.
\item El usuario recibe el \textit{feedback} con los registros inferidos.
}
{Los datos procesados se han enriquecido con el análisis de sentimientos y se han almacenado nuevamente en la base de datos \textit{OLAP} (\textit{On-Line Analytical Processing}).}
{
\item La herramienta no puede establecer conexión con la base de datos.
\item Algún registro no se ha procesado correctamente y supera el límite de caracteres de entrada a los modelos \textit{NLP}.
}
{Alta}


\apendice{Especificación de diseño}

\section{Introducción}

\section{Diseño de datos}

\subsubsection{Diseño de los cuadros de mando}

A continuación, se detalla la evolución seguida para el diseño de los cuadros de mando implementados en la herramienta \textit{Apache Superset}.

\imagen{dashboard-mockup-1}{Diseño inicial de la pestaña \textit{Raw data analysis}}

\imagen{dashboard-mockup-2}{Diseño inicial de la pestaña \textit{Overview analysis}}

\imagen{dashboard-mockup-3}{Diseño inicial de la pestaña \textit{Sentiment analysis}}

\section{Diseño procedimental}

\section{Diseño arquitectónico}



\apendice{Documentación técnica de programación}

\section{Introducción}

En esta sección se incluye la documentación técnica necesaria para entender la organización de directorios, la manera de instalar, configurar y ejecutar el proyecto, y los pasos a seguir para continuar con futuros desarrollos.


\section{Estructura de directorios}

La estructura del proyecto se ha organizado en la siguiente forma:

\begin{itemize}
    \item \textbf{\texttt{./}} Carpeta raíz del proyecto, contiene el directorio de prototipos, las carpetas relevantes para desplegar cada componente del proyecto, la documentación del proyecto, el fichero de requisitos y los de información (\textit{README}, licencia...).

    \item \textbf{\texttt{./docs}} Directorio principal de la documentación del proyecto, donde se encuentra la memoria y los anexos en formato \LaTeX{}.

    \begin{itemize}
        \item \textbf{\texttt{./docs/img}} Carpeta que contiene las imágenes utilizadas en la documentación.
        \item \textbf{\texttt{./docs/tex}} Carpeta que contiene los ficheros \textit{TEX} que conforman la documentación.
    \end{itemize}

    \item \textbf{\texttt{./prototypes}} Directorio que contiene los desarrollos realizados a lo largo del prototipado del proyecto. La estructura de carpetas es la misma que la expuesta en los puntos siguientes.

    \item \textbf{\texttt{./airflow}} Directorio que contiene los ficheros de despliegue y configuración de Apache Airbyte, así como los archivos \textit{dockerfile} y \textit{docker-compose} para crear los contenedores del proyecto y levantar el \textit{proxy} para securizar el \textit{Docker socket}.

    \begin{itemize}
        \item \textbf{\texttt{./airflow/config}} Carpeta que aparecerá una vez desplegado Apache Airflow en la que se pueden añadir configuraciones adicionales.

        \item \textbf{\texttt{./airflow/dags}} Carpeta que contiene los archivos con las definiciones de los \textit{DAGs} (\textit{Directed Acyclic Graphs}) que crean las \textit{data pipelines} implementadas y el orden de ejecución de las mismas.

        \item \textbf{\texttt{./airflow/logs}} Carpeta que aparecerá una vez desplegado Apache Airflow en la que se guardan los registros de todas las ejecuciones de las \textit{data pipelines} creadas.

        \item \textbf{\texttt{./airflow/plugins}} Carpeta que aparecerá una vez desplegado Apache Airflow en la que se pueden añadir \textit{plugins} adicionales.
    \end{itemize}

    \item \textbf{\texttt{./clickhouse}} Directorio que contiene los ficheros de despliegue y configuración de ClickHouse.

    \item \textbf{\texttt{./extraction}} Directorio que contiene el fichero de descarga y configuración de la herramienta de extracción Airbyte.

    \item \textbf{\texttt{./mongodb}} Directorio que contiene los ficheros de despliegue y configuración de MongoDB.

    \item \textbf{\texttt{./nlp}} Directorio que contiene los ficheros de despliegue y configuración de \textit{HuggingFace Transformers}.

    \begin{itemize}
        \item \textbf{\texttt{./nlp/nlp-tasks}} Carpeta que contiene los ficheros Python que se encargan de la configuración y ejecución de las tareas de procesamiento de lenguaje natural (\textit{NLP}).
    \end{itemize}

    \item \textbf{\texttt{./spark}} Directorio que contiene los ficheros de despliegue y configuración de Apache Spark.

    \begin{itemize}
        \item \textbf{\texttt{./spark/spark-tasks}} Carpeta que contiene los ficheros Scala que se encargan de la configuración y ejecución de las tareas de procesamiento y limpieza de datos.
    \end{itemize}

    \item \textbf{\texttt{./visualization}} Directorio que contiene los ficheros de despliegue y configuración de Apache Superset.

    \item \textbf{\texttt{./web}} Directorio principal de la aplicación web \textit{Flask} que se utiliza a modo de punto de acceso centralizado a las interfaces web de los componentes del proyecto..
    \begin{itemize}
        \item \textbf{./app} 
        \begin{itemize}
            \item \textbf{./app/static} Contiene los ficheros estáticos utilizados para la aplicación web.
                \begin{itemize}
                    \item \textbf{./app/static/css} Carpeta con los archivos de estilos \textit{CSS}.
                    \item \textbf{./app/static/js} Carpeta con los archivos de funciones \textit{JavaScript}.
                    \item \textbf{./app/static/media} Carpeta con las imágenes utilizadas en aplicación web.
                \end{itemize}
            \item \textbf{./app/templates} Contiene las plantillas \textit{HTML} de la aplicación web.
        \end{itemize}

    \end{itemize}

\end{itemize}


\section{Manual del programador}

En esta sección se describirán todos los elementos necesarios y la metodología a seguir para realizar futuros desarrollos.

\subsection{Configuración de Airbyte}

En este apartado se detallarán los métodos de configuración de la herramienta de extracción de datos.

\subsubsection{Configuración del origen de datos}

A continuación, se va a crear un ejemplo de fichero de configuración para una de las fuentes de datos seleccionada.

\begin{itemize}
    
    \item \textbf{Utilizando la API.}
    
    \begin{enumerate}
        
        \item \textbf{Consultar la definición específica de la fuente de datos a configurar.} Para ello es necesario consultar el \textit{endpoint} \verb|/v1/source_definitions/get|, lo que resultaría en una respuesta como la siguiente en el caso de escoger Twitter como fuente de datos (abreviada debido a su extensión real):

        \begin{verbatim}
        {
          "sourceDefinitionId": "...",
          "documentationUrl": "...",
          "connectionSpecification": {
            "type": "object",
            "title": "Twitter Spec",
            "$schema": "...",
            "required": [
              "api_key",
              "query"
            ],
            "properties": {...},
            ...
          },
          "jobInfo": {...}
          }
        }
        \end{verbatim}
        
        \item \textbf{Enviar la petición de creación de fuente de datos.} Completando el \textit{payload} de la petición hacia \verb|/v1/sources/create| con las propiedades correspondientes obtenidas del paso anterior. 
        
        \item \textbf{Comprobar la conexión con la fuente de datos.} Enviando una petición a \verb|/v1/sources/check_connection|.
    
    \end{enumerate}
    
    \item \textbf{Utilizando la interfaz gráfica.} Navegando a la sección \textit{Sources} y seleccionando el tipo de fuente a configurar, se muestra un formulario de edición que permite realizar las mismas operaciones ejecutadas anteriormente mediante la \textit{API}. En la \autoref{fig:airbyte_source_config} se puede observar dicho formulario.

    \imagen{airbyte_source_config}{Configuración del origen de datos en Airbyte: Twitter}
    
\end{itemize}

\subsubsection{Configuración del destino de los datos}

A continuación, se va a crear un ejemplo de fichero de configuración para uno de los destinos de datos seleccionados.

\begin{itemize}
    
    \item \textbf{Utilizando la API.}
    
    \begin{enumerate}
        
        \item \textbf{Consultar la definición específica del destino de datos a configurar.} Para ello es necesario consultar el \textit{endpoint} \\ \verb|/v1/destination_definition_specifications/get|, lo que resultaría en una respuesta como la siguiente en el caso de escoger un fichero \textit{CSV} local como destino de datos (abreviada debido a su extensión real):

        \begin{verbatim}
        {
            "destinationDefinitionId": "...",
            "documentationUrl": "...",
            "connectionSpecification": {
                "type": "object",
                "title": "CSV Destination Spec",
                "$schema": "...",
                "required": [
                    "destination_path"
                ],
                "properties": {...},
                ...
            },
            "jobInfo": {...},
            "supportedDestinationSyncModes": [
                "overwrite",
                "append"
            ]
        }
        \end{verbatim}
        
        \item \textbf{Enviar la petición de creación de destino de datos.} Completando el \textit{payload} de la petición hacia \verb|/v1/destinations/create| con las propiedades correspondientes obtenidas del paso anterior. 
        
        \item \textbf{Comprobar la conexión con el destino de datos.} Enviando una petición a \verb|/v1/destinations/check_connection|.
    
    \end{enumerate}
    
    \item \textbf{Utilizando la interfaz gráfica.} Navegando a la sección \textit{Destinations} y seleccionando el tipo de fuente a configurar, se muestra un formulario de edición que permite realizar las mismas operaciones ejecutadas anteriormente mediante la \textit{API}. En la \autoref{fig:airbyte_destination_config} se puede observar dicho formulario.

    \imagen{airbyte_destination_config}{Configuración del destino de datos en Airbyte: CSV Local}
    
\end{itemize}

\subsubsection{Configuración de la conexión entre fuente y destino}

A continuación, se va a crear un ejemplo de fichero de configuración para una de las conexiones de datos seleccionada.

\begin{itemize}
    
    \item \textbf{Utilizando la API.}
    
    \begin{enumerate}
        
        \item \textbf{Crear la conexión entre fuente y destino.} Para ello es necesario mandar una petición al \textit{endpoint} \verb|/v1/connections/create| con un \textit{payload} como el siguiente (abreviado debido a que es bastante extenso):

        \begin{verbatim}
        {
          "name": "Twitter-API <> Local-CSV",
          "namespaceDefinition": "destination",
          "sourceId": "...",
          "destinationId": "...",
          "syncCatalog": {
            "streams": [
              {
                "stream": {
                  "name": "...",
                  "supportedSyncModes": [
                    "full_refresh", "incremental"
                  ]
                },
                "config": {
                  "syncMode": "full_refresh",
                  "destinationSyncMode": "append",
                  "aliasName": "...",
                  "selected": true
                }
              }
            ]
          },
          "scheduleType": "manual",
          "status": "active",
          "geography": "auto",
          "notifySchemaChanges": true,
          "nonBreakingChangesPreference": "ignore"
        }
        \end{verbatim}
        
        \item \textbf{Ejecutar una sincronización manual de la conexión.} Mandando una petición hacia \verb|/v1/connections/sync|.
    
    \end{enumerate}
    
    \item \textbf{Utilizando la interfaz gráfica.} Navegando a la sección \textit{Connections} y seleccionando las fuentes y destinos de datos para los que configurar la conexión. Se muestra un formulario de edición que permite realizar las mismas operaciones ejecutadas anteriormente mediante la \textit{API}. En la \autoref{fig:airbyte_connection_config} se puede observar dicho formulario.

    \imagen{airbyte_connection_config}{Configuración de la conexión entre origen y destino de los datos en Airbyte}
    
\end{itemize}

\subsubsection{Creación de nuevos conectores}

La herramienta de extracción Airbyte permite la creación de nuevos conectores que utilicen \textit{APIs REST} de forma rápida y sencilla directamente desde la interfaz gráfica. Para ello, es necesario acceder a la siguiente sección de la herramienta: \url{http://<AIRBYTE-EXAMPLE.COM>/workspaces/<WORKSPACE_ID>/connector-builder/}.

En esta vista se pueden crear y configurar nuevos conectores para \textit{APIs} bien de manera visual (\autoref{fig:airbyte-connection-builder}) o mediante un esquema en formato \textit{YAML} (\autoref{fig:airbyte-connection-builder-yaml}).

\imagen{airbyte-connection-builder}{Interfaz gráfica de la vista para creación y configuración de nuevos conectores para Airbyte}

\imagen{airbyte-connection-builder-yaml}{Esquema \textit{YAML} de la vista para creación y configuración de nuevos conectores para Airbyte}

\subsection{Procesamiento de los datos}

El procesamiento de los datos se ha realizado mediante \textit{scripts} en lenguaje Scala para el sistema Apache Spark, por lo que será necesario tener conocmientos de estas dos herramientas para poder llevar a cabo mayores desarrollos en esta parte.

Se ha de configurar un único archivo de procesamiento a ejecutar por cada origen de datos creado, aunque se pueden incluir dependencias como funciones comunes a varios archivos. Los \textit{scripts} pueden seguir la estructura de ejemplo en los ficheros \texttt{transform\_twitter.scala} y \texttt{transform\_dataset.scala}.

La ejecución de ambos archivos deberá poder realizarse mediante la \texttt{spark-shell}, a la que se le han de pasar los parámetros los parámetros de conexión para las bases de datos de origen y destino de los datos. Además, de los parámetros necesarios para identificar y filtrar los datos concretos a extraer y procesar.

\subsection{Inferencia \textit{NLP}}

La inferencia mediante técnicas de procesamiento de lenguaje natural (\textit{NLP}) se ha realizado mediante la librería \textit{HuggingFace Transformers}. Esta librería permite la utilización de \textit{LLMs} (\textit{Large Language Models}) ya preentrenados para estas tareas \textit{NLP}.

Para añadir más tareas a la \textit{pipeline} de inferencia de datos, se han de modificar los siguientes archivos:

\begin{itemize}
    \item El archivo \texttt{connectors.py} si se van a añadir nuevos conectores para distintos orígenes o destinos de datos.
    \item El archivo \texttt{tasks.py} para añadir las clases correspondientes con todos los parámetros necesarios para configurar las conexiones de entrada, salida, nombre de las tablas de los datos, etc. En el mismo archivo se añadirán nuevos métodos \textit{task\_*}, uno por cada tarea o modelo \textit{NLP} a emplear.
    \item El archivo \texttt{pipeline.py} para añadir las nuevas opciones de ejecución para los nuevos orígenes o destinos de datos.
\end{itemize}

\subsection{Visualización de los datos}

La herramienta de visualización Apache Superset presenta una interfaz intuitiva y sencilla de utilizar para la agregación de nuevas fuentes de datos y la creación de gráficos interactivos.

En la pestaña \textit{Datasets} se pueden configurar conexiones a nuevos orígenes de datos. Posteriormente, desde la pestaña \textit{SQL Lab} se pueden realizar consultas en \textit{SQL} nativo sobre los \textit{datasets} importados.

A parte de los datos importados, los resultados de las consultas realizadas en \textit{SQL Lab} se pueden guardar en \textit{datasets} virtuales para poder utilizarlos posteriormente también como base para la creación de nuevos gráficos.

En la pestaña \textit{Charts} se pueden diseñar e implementar nuevas visualizaciones que añadir posteriormente a los \textit{dashboards} creados. El proceso presenta la selección de un conjunto de datos a utilizar para cada gráfico creado y un tipo de visualización que se puede configurar y personalizar posteriormente.

\subsection{Orquestación de los datos}

La orquestación de los datos se ha realizado mediante la herramienta Apache Airflow. Esta herramienta permite definir grafos acíclicos dirigidos (\textit{DAGs}) para diseñar los flujos de trabajo a ejecutar.

En la carpeta \texttt{dags} existente dentro del directorio respectivo de la herramienta se encuentran los \textit{DAGs} diseñados para la plataforma. Cada fuente de datos deberá tener su \textit{DAG} propio.

Estos grafos acíclicos dirigidos se pueden crear de manera estática (véase el archivo \texttt{twitter\_pipeline\_dag.py}) o de manera dinámica (véase el archivo \texttt{dataset\_pipeline\_dag.py}). De esta manera, al necesitar crear cierto número de flujos de datos que utilicen la misma configuración base pero cambiando solamente los parámetros de entrada, se puede realizar la configuración de manera dinámica para añadir más claridad al código y facilitar la comprensión del mismo.

Los \textit{DAGs} están formados por <<actuadores>>, que realizan las acciones, y por <<sensores>>, que esperan una confirmación para continuar con el flujo de trabajo. las \textit{data pipelines} diseñadas tienen la siguente estructura:

\begin{itemize}
    \item Actuador Airbyte: Ejecutar sincronización de los datos.
    \item Sensor Airbyte: Esperar la confirmación con el estado de la sincronización de los datos.
    \item Actuador \textit{Docker} (\textit{Spark}): Desplegar contenedor, ejecutar procesamiento de los datos y eliminar contenedor.
    \item Actuador \textit{Docker} (\textit{NLP}): Desplegar contenedor, ejecutar inferencia \textit{NLP} y eliminar el contenedor.
\end{itemize}

Cada tipo de tarea tiene su propia actuador o sensor. Airflow presenta actuadores listo para utilizar para algunas de las acciones más comunes como ejecutar comandos \textit{bash}, enviar correos electrónicos, realizar peticiones \textit{HTTP}, ejecutar código Python, etc.



\section{Compilación, instalación y ejecución del proyecto}

En los siguientes apartados se desarrollan los requisitos necesarios para la instalación y despliegue del proyecto.

\subsection{Especificaciones técnicas recomendadas}

Esta <<Plataforma \textit{Big data} para \textit{Sentiment Analysis}>> se ha desarrollado en un entorno local y empleando una sola máquina. No obstante, al ser una plataforma modular, escalable y distribuida, en un entorno de producción real se debería realizar el despliegue de cada componente en una máquina o clúster dedicado. De esta manera, se evitaría la sobrecarga del sistema, se aumentaría la seguridad de la plataforma y se eliminarían las limitaciones en el uso de recursos por cada componente.

En caso de realizar el despliegue en una sola máquina, y teniendo en cuenta las consideraciones anteriores, se recomiendan las siguientes especificaciones técnicas para la misma:

\begin{itemize}
    \item \textbf{Sistema operativo:} Windows 11 (22H2) / Ubuntu 22.04.2 LTS
    \item \textbf{CPU:} Intel(R) Core(TM) i5-13600K 3.50 GHz
    \item \textbf{Memoria RAM:} 32GB DDR4 3.2GHz
    \item \textbf{GPU:} NVIDIA GeForce RTX 3060 12GB
\end{itemize}

El punto más notable de esta plataforma sería los requisitos de memoria \textit{RAM} que presenta. La gran cantidad de componentes que se despliegan en contenedores \textit{Docker} establece el consumo en reposo de la plataforma en unos 13 GB (véase \autoref{fig:ram-usage}). Este consumo aumenta hasta los 18GB al momento de ejecutar la etapa de transformación del proceso \textit{ETL} (\textit{Extract, Transform, Load}), por lo que sería recomendable cumplir esta recomendación para evitar caídas del sistema ante altos niveles de carga.

\imagen{ram-usage}{Consumo de memoria RAM con la plataforma <<en reposo>>}

Respecto a la GPU, realmente no se necesita para la ejecución normal de la inferencia \textit{NLP} ya que actualmente dichos modelos se ejecutan en la CPU. No obstante, realizar estas tareas en una GPU de alta gama permitiría reducir notablemente los tiempos de ejecución de esta fase del proceso \textit{ETL}. También sería útil en gran medida en el caso de querer realizar el entrenamiento de modelos propios sobre cada flujo de datos.

\subsection{Dependencias \textit{software}}

A continuación, se describen las versiones de los componentes \textit{software} empleados para la realización del proyecto:

\begin{itemize}
    \item Airbyte 0.43.1
    \item Apache Airflow 2.6.2
    \item Apache Spark 3.2.3
    \item MongoDB 6.0
    \item ClickHouse 22.1.3.7
    \item Apache Superset 2.1.0
    \item HuggingFace Transformers 4.30.1
    \item Python 3.10.6
    \item Flask 2.3.2
\end{itemize}

\subsection{Instalación}

Para la instalación del proyecto tan solo es necesario clonar localmente el código del repositorio \textit{GitHub}: \url{https://github.com/liviuvj/sentiment-analysis-platform}

Posteriormente, es necesario instalar las dependencias que se encuentran en el archivo de requisitos. Para ello, es buena práctica crear primero un entorno virtual para evitar problemas de versiones posteriormente y realizar la instalación del fichero de requisitos.

\begin{verbatim}
    python3 -m venv venv
    source ./venv/bin/activate
    pip install -r requirements.txt
\end{verbatim}

\subsection{Compilación}

Este proyecto está compuesto en su mayor parte por contenedores \textit{docker} y código en \textit{scripts} de Python y Scala, por lo que no es necesaria la compilación del mismo. Al ser lenguajes de \textit{scripting} y ficheros de configuración, no resulta necesaria su compilación.

\subsection{Ejecución}

Para la ejecución del proyecto se ha creado un \textit{script bash} que se encarga de realizar la descarga de ficheros adicionales, clonación de repositorios, despliegue de contenedores \textit{Docker} y configuración de los componentes que forman parte del proyecto.

Por lo que tan solo será necesario ejecutar dicho \textit{script bash}:

\begin{verbatim}
    ./quickstart.sh
\end{verbatim}

No obstante, cada directorio principal correspondiente a las herramientas empleadas para la realización de este proyecto contiene un fichero \texttt{.env}. Dicho archivo contiene los parámetros de configuración por defecto, así como usuarios y contraseñas, de los distintos componentes de la plataforma. Por lo que sería recomendable modificar dichos valores en un entorno de producción.

Una vez levantados y configurados todos los componentes del proyecto, se puede acceder las distintas interfaces web disponibles para gestionar o monitorizar las herramientas. Las \textit{URLs} especificadas muestran los puertos asignados por defecto con la configuración realizada.

\begin{itemize}
    \item \textbf{Punto web de acceso centralizado.} Página web ligera que permite el acceso directo a todas las demás interfaces web que presentan los componentes de la plataforma. Accesible en \url{http://localhost:5000}

    \item \textbf{Apache Airbyte.} Plataforma desde la que se pueden crear, gestionar y monitorizar las conexiones de sincronización de datos entre origen y destino. Accesible en \url{http://localhost:8000}

    \item \textbf{Apache Spark.} Monitorización de los recursos utilizados por Spark. Accesible en \url{http://localhost:4040}

    \item \textbf{ClickHouse.} Sencilla interfaz web que permite la ejecución de consultas directas sobre la base de datos en lenguaje \textit{SQL}. Accesible en \url{http://localhost:8123/play}

    \item \textbf{Apache Superset.} Aplicación en la que se pueden diseñar, implementar, visualizar y compartir cuadros de mando interactivos. Accesible en \url{http://localhost:8088}

    \item \textbf{Apache Airflow.} Accesible en \url{http://localhost:8080}
\end{itemize}

\section{Pruebas del sistema}

A lo largo de este proyecto se han realizado diversas tareas de integración entre los distintos componentes que conforman la plataforma desarrollada. No obstante, no se han implementado unas pruebas unitarias o de integración de manera formal.

Esto se debe a que la mayoría de los componentes utilizados ya presentan sus propias pruebas específicas para la validación, integración y funcionamiento utilizados. Por ello, se ha decidido por la creación de un \textit{script bash} para realizar el despliegue y configuración automáticos de la plataforma. En caso de encontrarse algún fallo en alguna de las tecnologías empleadas, la propia herramienta será capaz de identificar la configuración errónea y notificar al usuario del fallo detectado.

\apendice{Documentación de usuario}

\section{Introducción}

En las secciones siguientes se incluye la documentación necesaria dar a los usuarios del proyecto las nociones necesarias y la manera de instalar, configurar, ejecutar y utilizar la plataforma desarrollada.

La definición de usuario final para esta plataforma cubre dos roles principales. El primero, un usuario sin conocimientos técnicos que solamente va a visualizar los datos finales a través del \textit{dashboard} diseñado. Y el segundo, un analista de datos que se va a encargar también de la exploración de los datos y de gestionar e implementar los cuadros de mando para la plataforma.

\section{Requisitos de usuarios}

Esta <<Plataforma \textit{Big data} para \textit{Sentiment Analysis}>> se ha desarrollado en un entorno local y empleando una sola máquina. No obstante, al ser una plataforma modular, escalable y distribuida, en un entorno de producción real se debería realizar el despliegue de cada componente en una máquina o clúster dedicado.

Para los usuarios finales, estos tendrían que acceder simplemente a través de las interfaces web correspondientes a los componentes necesarios. Por lo que los requisitos para estos usuarios serían los siguientes:

\vspace{2cm}

\begin{itemize}
    \item Disponer de una conexión con acceso a la red en la que se haya realizado el despliegue de esta <<Plataforma \textit{Big data} para \textit{Sentiment Analysis}>>.
    \item Tener asignados los roles o disponer de los permisos y credenciales necesarios para acceder a las interfaces web de las herramientas correspondientes (en este caso, Apache Superset).
    \item Disponer de los conocimientos necesarios para interpretar los datos visualizados. En caso de ser un analista de datos, necesita también conocer las técnicas de análisis de datos respectivas para llevar a cabo el diseño e implementación de \textit{dashboards}.
\end{itemize}


\section{Instalación}

Como se ha mencionado anteriormente, al tratarse de una plataforma \textit{Big Data}, esta herramienta idealmente se estaría ejecutando sobre un clúster dedicado para ello y no sobre la máquina del usuario final.

No obstante, gracias al \textit{script bash} que se ha creado para el despliegue y configuración de la plataforma, el usuario final puede instalarse el entorno en la máquina local para realizar las pruebas necesarias en caso de dispone de los requisitos técnicos necesarios. 

Para la instalación del proyecto tan solo es necesario clonar localmente el código del repositorio \textit{GitHub}: \url{https://github.com/liviuvj/sentiment-analysis-platform}

Posteriormente, es necesario instalar las dependencias que se encuentran en el archivo de requisitos. Para ello, es buena práctica crear primero un entorno virtual para evitar problemas de versiones posteriormente y realizar la instalación del fichero de requisitos.

\begin{verbatim}
    python3 -m venv venv
    source ./venv/bin/activate
    pip install -r requirements.txt
\end{verbatim}

Para la ejecución del proyecto se ha creado un \textit{script bash} que se encarga de realizar la descarga de ficheros adicionales, clonación de repositorios, despliegue de contenedores \textit{Docker} y configuración de los componentes que forman parte del proyecto.

\vspace{2cm}

Por lo que tan solo será necesario ejecutar dicho \textit{script bash}:

\begin{verbatim}
    ./quickstart.sh
\end{verbatim}

No obstante, cada directorio principal correspondiente a las herramientas empleadas para la realización de este proyecto contiene un fichero \texttt{.env}. Dicho archivo contiene los parámetros de configuración por defecto, así como usuarios y contraseñas, de los distintos componentes de la plataforma. Por lo que sería recomendable modificar dichos valores en un entorno de producción.



\section{Manual del usuario}

Las siguientes secciones que desarrollan los usos que pueden dar los usuarios finales a la plataforma se han distribuido según el propio tipo de usuario que la va a utilizar. 

\subsection{Usuario final básico}

El usuario final básico debería tener el acceso mínimo y necesario para su caso de uso. De esta manera, se tendría acceso a los siguientes componentes (cuyas \textit{URLs} muestran los puertos asignados por defecto con la configuración realizada):

\begin{itemize}
    \item \textbf{Punto web de acceso centralizado.} Página web ligera que permite el acceso directo a todas las demás interfaces web que presentan los componentes de la plataforma. Accesible en \url{http://localhost:5000}

    \item \textbf{Apache Superset.} Aplicación en la que se pueden diseñar, implementar, visualizar y compartir cuadros de mando interactivos. Accesible en \url{http://localhost:8088}

\end{itemize}

\subsubsection{Visualización de cuadros de mando}

Por las razones mencionadas anteriormente, este usuario podría únicamente visualizar los \textit{dashboards} implementados en la plataforma.

Para ello, es necesario acceder a la interfaz web de la herramienta Apache Superset y acceder a alguno de los cuadros de mando disponibles según los permisos asignados.

Posteriormente podrá navegar por las distintas vistas del \textit{dashboard} visualizando los gráficos y obtener los \textit{insights} que crea oportunos, además de realizar acciones como el filtrado de los datos con las opciones que disponga para ello en el menú lateral.


\subsection{Analista de datos}

El analista de datos debería tener los conocimientos y capacidades necesarias para realizar más que solamente visualizar los \textit{dashboards} ya implementados. Por ello, este usuario se entiende que tiene acceso a los siguients componentes (cuyas \textit{URLs} muestran los puertos asignados por defecto con la configuración realizada):

\begin{itemize}
    \item \textbf{Punto web de acceso centralizado.} Página web ligera que permite el acceso directo a todas las demás interfaces web que presentan los componentes de la plataforma. Accesible en \url{http://localhost:5000}

    \item \textbf{ClickHouse.} Sencilla interfaz web que permite la ejecución de consultas directas sobre la base de datos en lenguaje \textit{SQL}. Accesible en \url{http://localhost:8123/play}

    \item \textbf{Apache Superset.} Aplicación en la que se pueden diseñar, implementar, visualizar y compartir cuadros de mando interactivos. Accesible en \url{http://localhost:8088}
\end{itemize}

\subsubsection{Exploración de datos}

Este usuario debería tener acceso suficiente para realizar consultas en la base de datos utilizada para almacenar la información a visualizar, el sistema \textit{OLAP} ClickHouse.

Para ello, se facilita su acceso directamente desde la herramienta Apache Superset, en la pestaña dedicada para ello \textit{SQL Lab}. En esta vista se podrán seleccionar uno o varios conjuntos de datos sobre los que realizar consultas en lenguaje \textit{SQL} nativo.

Las consultas realizadas se guardan en un histórico para poder volver a ejecutarlas posteriormente en caso necesario. Además, los resultados de estas consultas se pueden exportar como nuevos \textit{datasets} virtuales que pueden utilizados posteriormente como base de nuevas consultas o gráficos.

\imagen{superset-lab}{Vista del \textit{SQL Lab} para la exploración de datos de Apache Superset}

\subsubsection{Creación o modificación de visualizaciones}

Para la creación o modificación de nuevas visualizaciones es necesario acceder a la pestaña \textit{Charts} de la herramienta Apache Superset. En esta vista se puede seleccionar alguno de los \textit{datasets} disponibles y un tipo de visualización de los más de 40 tipos distintos que tiene la herramienta.

A continuación, se accede a la vista de edición de la visualización. En esta pantalla se pueden seleccionar los campos deseados que se pueden añadir a la visualización, simplemente haciendo \textit{click} y arrastrándolos al elemento correspondiente.

\imagen{superset-chart-edit}{Vista de edición de visualizaciones de Apache Superset}

También se presentan en la misma vista opciones para cambiar el tipo de gráfico y personalizar los elementos visuales del mismo (paleta de colores, leyendas, títulos, etc.). Además, presenta también las opciones de configuración de analíticas avanzadas, como la previsión de valores.

\subsubsection{Integración de nuevos cuadros de mando}

Para la creación de nuevos cuadros de mando se ha de acceder a la pestaña \textit{Dashboards} de la herramienta Apache Superset. En esta vista se pueden modificar los \textit{dashboards} ya existentes o añadir nuevos.

Esta pantalla de edición cuenta con las siguientes partes principales:

\begin{itemize}
    \item El menú lateral izquierdo permite la definición y configuración de filtros para el cuadro de mando. Se permite la selección de atributos presentes entre los distintos \textit{datasets} que pueden estar formando el \textit{dashboard} y la elección de a qué gráficos afectan estos filtros.

    \item La vista central es la plantilla donde se van a colocar y organizar las distintas visualizaciones creadas anteriormente. Se puede ajustar tanto el tamaño como la posición que ocupan en la pantalla.

    \item El menú lateral derecho contiene las visualizaciones creadas hasta el momento y acceso directo a la creación de nuevas, estas se pueden colocar en la vista central simplemente arrastrándolas. También cuenta con unos elementos gráficos para la estructuración del \textit{dashboard} como separadores, pestañas, organizadores, etc.
\end{itemize}



\bibliographystyle{plain}
\bibliography{bibliografia}

\end{document}
