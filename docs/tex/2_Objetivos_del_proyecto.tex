\capitulo{2}{Objetivos del proyecto}

Este apartado explica de forma precisa y concisa los objetivos que se persiguen con la realización del proyecto. Se realiza una distinción entre los objetivos de carácter general, los de carácter técnico (propios del proyecto) y también los personales.

\section{Objetivos generales}

\begin{itemize}
    \item Realizar extracciones de datos de dominio público.

    \item Llevar a cabo tareas de procesamiento, limpieza y asegurar la calidad de los datos.

    \item Utilizar técnicas de procesamiento de lenguaje natural para enriquecer los datos.

    \item Diseñar e implementar cuadros de mando interactivos sobre la información obtenida.

    \item Aplicar procesos de Extracción, Transformación y Carga (\textit{ETL}).

    \item Diseñar y desarrollar una plataforma \textit{open-source} para análisis de sentimiento sobre \textit{Big Data}.
\end{itemize}

\vspace{2cm}

\section{Objetivos técnicos}

\begin{itemize}
    \item Utilizar herramientas y tecnologías de código abierto.

    \item Emplear tecnologías distribuidas y escalables capaces de soportar \textit{Big Data}.

    \item Estudiar los \textit{frameworks} y herramientas más convenientes a utilizar en cada etapa del proyecto.

    \item Diseñar e implementar \textit{data pipelines} para realizar el proceso \textit{ETL}.

    \item Permitir la programación temporal de la ejecución de las \textit{data pipelines}.

    \item Automatizar los procesos mediante herramientas de orquestación.

    \item Diseñar e implementar los \textit{dashboard} para la toma de decisiones siguiendo las buenas prácticas recomendadas.

    \item Mantener la calidad de los datos extraídos y procesados.

    \item Diseñar la plataforma con una arquitectura modular que permita el cambio por otras tecnologías en las distintas etapas del proyecto.

    \item Mantener la plataforma segura con distinción de usuarios y gestión de roles y permisos.

    \item Diseñar el esquema de datos a utilizar y las interacciones entre los componentes de la plataforma.

    \item Permitir la gestión o monitorización de la plataforma mediante interfaces web.

    \item Realizar el despliegue del proyecto mediante contenedores \textit{Docker}.
\end{itemize}

\vspace{5cm}

\section{Objetivos personales}

\begin{itemize}
    \item Poner en práctica la mayoría de conocimientos obtenidos a lo largo del máster sobre la consutrcción de infraestructuras para \textit{Big Data}, metodologías distribuidas y escalables de procesamiento de datos y \textit{business intelligence}. 

    \item Aprender a utilizar tecnologías modernas sobre extracción, transformación y carga de datos.

    \item Diseñar la arquitectura completa de una plataforma integral para ofrecer soluciones \textit{Big Data}. 

    \item Utilizar modelos del estado del arte sobre técnicas de procesamiento de lenguaje natural.

    \item Diseñar \textit{dashboards} interactivos y llamativos.
\end{itemize}

