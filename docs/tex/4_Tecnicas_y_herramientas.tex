\capitulo{4}{Técnicas y herramientas}

Esta parte de la memoria tiene como objetivo presentar las técnicas metodológicas y las herramientas de desarrollo que se han utilizado para llevar a cabo el proyecto. En el caso de algunas de estas herramientas se estudiarán diferentes alternativas, en las que se incluirán comparativas entre las distintas opciones y una justificación de la elección realizada.



\section{Técnicas}

En este apartado se hará una breve descripción sobre las técnicas empleadas a lo largo del proyecto.

\subsection{SCRUM}
Es un proceso de desarrollo software ubicado dentro de las metodologías ágiles. Consiste en segmentar un proyecto en varios requisitos que se han de cumplir y posteriormente subdividir estos en tareas. El desarrollo se realiza mediante \textit{sprints}, iteraciones incrementales de normalmente dos semanas de duración, en los que se planifican las tareas a realizar durante dicho periodo.

\subsection{Procesamiento de Lenguaje Natural}
El término \textit{NLP} (\textit{Natural Language Processing}) se refiere al conjunto de métodos, dentro de la inteligencia artificial, que trabajan con recursos textuales o sonoros. Se ponen en práctica metodologías de estadística, lingüística y \textit{machine learning} para permitir crear programas que puedan interpretar dicho tipo de información \cite{chowdhary2020natural}.

\subsection{Sentiment Analysis}
El análisis de sentimientos es una técnica en la que se busca identificar y extraer información subjetiva a partir de recursos textuales. Las principales maneras de realizar este tipo de análisis suelen seguir dos rutas.

La primera, utilizando reglas y diccionarios de palabras a las que se les asignan distintas puntuaciones según el sentimiento asociado a cada palabra. La segunda, y la que mejores resultados proporciona actualmente~\cite{mishev2020evaluation}, emplea técnicas de \textit{NLP} para extraer características de los datos y comprender el contexto de la información proporcionada. Esto permite realizar clasificaciones y predicciones más acertadas ya que el resultado no se limita simplemente a un subconjunto de palabras, sino al sentido que se les da a las mismas también.

\section{Herramientas}

Para llevar a cabo este proyecto, se ha utilizado el siguiente conjunto de herramientas.

\subsection{GitHub}

Para el \textit{hosting} del repositorio se ha utilizado \textit{GitHub}\footnote{\url{https://github.com/}}, puesto que ya se tenía experiencia en el uso de esta plataforma. Permite realizar la gestión del control de versiones a lo largo del desarrollo del software y simplifica el seguimiento de las tareas. Posee capacidades para creación de procesos de integración continua y despliegue continuo (\textit{CI/CD}), automatización de flujos de trabajo, seguimiento y gestión de proyectos.

\subsection{ZenHub}
Para facilitar el trabajo de la gestión del proyecto se ha utilizado \textit{ZenHub}\footnote{\url{https://www.zenhub.com/}}. Es una plataforma centrada en mejorar la productividad de los equipos de desarrollo, que permite llevar a cabo la planificación del proyecto, realizar un seguimiento del progreso y calcular métricas de productividad mediante gráficas. 

Se ha elegido esta herramienta ya que, además de permitir realizar toda la gestión del proyecto, cuenta con una extensión web desde la que se puede acceder al panel de control directamente desde el propio repositorio de GitHub. Por lo que todas las operaciones de planificación de tareas se llevan a cabo desde el mismo lugar y facilita el trabajo del desarrollador.

\subsection{Entorno de desarrollo integrado}

Un \textit{Integrated Development Environment} (\textit{IDE}) es, como indica el propio nombre, un conjunto de herramientas que componen un espacio de trabajo completo para desarrollar \textit{software}. Suele estar compuesto de las herramientas necesarias para editar, compilar, ejecutar y probar código, facilitando así la labor del desarrollador.

\subsubsection{Herramientas consideradas:}

\begin{itemize}
    \item \textbf{Spyder:} Entorno de desarrollo \textit{open-source} especializado en la exploración de datos y el análisis científico.
    \item \textbf{Visual Studio:} Herramienta que permite realizar todas las tareas de programación, depuración, pruebas y desarrollo de soluciones para cualquier plataforma.
    \item \textbf{Visual Studio Code:} Versión más ligera y personalizable de Visual Studio.
\end{itemize}

\subsubsection{Herramienta elegida:}

\begin{itemize}
    \item \textbf{Visual Studio Code}\footnote{\url{https://code.visualstudio.com/}}
\end{itemize}

Es el IDE elegido para llevar a cabo el desarrollo de proyecto. Como ventajas principales, presenta un tamaño reducido de instalación respecto a las otras opciones y permite la configuración y ejecución de tareas, además de la capacidad para instalar y personalizar nuevas funcionalidades mediante sus extensiones.

\subsubsection{Extensiones utilizadas}

Se han escogido una serie de extensiones del \textit{Marketplace} que presenta la herramienta para facilitar la calidad de vida al trabajar con este IDE.

\begin{itemize}
    \item \textbf{Python:} Extensión principal para dar soporte al lenguaje de programación Python para el correcto desarrollo de código (\textit{linting}, formato de código, exploración de variables, depuración, etc.).
    \item \textbf{Python Docstring Generator:} Facilita y asiste en la creación de comentarios tipo \textit{docstring} para funciones en Python.
    \item \textbf{Pylance:} Servidor de lenguae que añade soporte adicional a Python.
    \item \textbf{Trailing Whitespace:} Resalta y recorta los espacios en blanco sobrantes.
    \item \textbf{Visual Studio IntelliCode:} Emplea IA para añadir desarrollo predictivo y autocompletado de código.
    \item \textbf{Docker:} Facilita la creación y gestión de contenedores a través del IDE.
\end{itemize}

\subsection{Editor \LaTeX{}}

La elaboración de esta memoria está basada en la plantilla \LaTeX{} provista como ejemplo por los Coordinadores del Máster y disponible públicamente\footnote{\url{https://github.com/bbaruque/plantillaTFM_MUINBDES.git}}. Para facilitar la edición y gestión de esta plantilla, se ha decidido utilizar una herramienta adecuada para ello. 

\subsubsection{Herramientas consideradas:}

\begin{itemize}
    \item \textbf{MiK\TeX{} + \TeX{}:} Herramientas que realizan la traducción de \LaTeX{} a texto y permiten gestionar y editar este tipo de archivos, respectivamente.
    \item \textbf{Overleaf:} Plataforma en línea que facilita la gestión y edición de documentos con formato \LaTeX{}.
\end{itemize}

\subsubsection{Herramienta elegida:}

\begin{itemize}
    \item \textbf{Overleaf}
\end{itemize}

Overleaf es un editor en línea\footnote{\url{https://es.overleaf.com/}} de \LaTeX{}. Para utilizarlo no es necesario realizar la instalación de ningún componente, tiene documentación integrada para \LaTeX{} y permite la visualización de los cambios realizados en tiempo real, además de contar ya con los paquetes más utilizados.

También resulta más cómodo al tratarse de una plataforma \textit{online}, ya que tan solo hace falta disponer de un navegador y conexión a Internet para poder trabajar con ella desde cualquier equipo. Otra de las mejores funcionalidades que ofrece es la posibilidad de comprobar el histórico de los archivos modificados y realizar un \textit{rollback} de los mismos.

Se ha utilizado esta herramienta para elaborar la memoria y los anexos en \LaTeX{}.

\subsection{Joplin}
A lo largo de la duración del proyecto hará falta tomar notas de diversos temas. Para facilitar esta tarea, se ha utilizado \textit{Joplin}\footnote{\url{https://joplinapp.org/}}. Es una plataforma de código abierto que permite gestionar apuntes y notas en forma de \textit{notebooks}.

Entre las principales características que ofrece se encuentra la total privacidad de los datos, la sencilla interfaz que presenta, la facilidad de uso gracias al lenguaje \textit{Markdown} y la sincronización de contenido entre diversos equipos.

Se utilizará principalmente para dejar constancia de los temas comentados durante las reuniones y apuntar información relevante para el proyecto que se vaya encontrando a medida que se desarrolle este trabajo.

\subsection{Super Productivity}
La gestión del tiempo dedicado se ha llevado a cabo mediante la herramienta de código abierto \textit{Super Productivity}\footnote{\url{https://super-productivity.com/}}. Sus principales funciones consisten en realizar la planificación, seguimiento y gestión de tareas. Permite distribuir tareas a lo largo de diversos proyectos, la asignación de etiquetas personalizadas y tener constancia del tiempo estimado y dedicado para cada una. 

Presenta una interfaz sencilla de utilizar y amigable para el usuario que agiliza el trabajo gracias a la utilización de atajos de teclado. Otra de las características más importantes que tiene esta herramientas es la integración con varias plataformas para la importación de tareas. Por lo que la planificación realizada en GitHub y ZenHub se puede extraer a esta herramienta y realizar un mejor seguimiento del tiempo empleado en cada una de ellas.

\subsection{Postman}

Es una plataforma\footnote{\url{https://www.postman.com/}} para construir y utilizar \textit{APIs} que simplifica el desarrollo y la colaboración. Cuenta con una versión web y una aplicación de escritorio, además de un repositorio público de colecciones de \textit{APIs} y documentación sobre los posibles tipos de llamadas que se les puede realizar.

Esta herramienta será la utilizada para llevar a cabo una inspección inicial de los distintos \textit{endpoints} que presentan las \textit{APIs} investigadas en la sección anterior. Un ejemplo concreto de ello podría ser el siguiente \textit{workspace} de Twitter \cite{postmanTwitterAPI}, en el que se presentan documentadas las posibles llamadas a realizar.

\subsection{Herramienta de extracción de datos}

Para facilitar la labor de ejecución de consultas contra las \textit{APIs} de las distintas plataformas web, se ha decidido emplear librerías de código ya habilitadas para ello.

\subsubsection{\textit{API Wrappers}} \label{section:api_wrappers}

Las interfaces estudiadas en la sección anterior tienen un gran número de usuarios, lo que ha conllevado a la creación de distintas librerías o paquetes en algunos lenguajes de programación que ``envuelven'' y ``atacan'' los \textit{endpoints} de dichas \textit{APIs}. Estos paquetes tienen el objetivo de facilitar al usuario la tarea de crear las consultas y consumir los recursos provenientes de dichas peticiones.

Estas librerías iban a ser utilizadas inicialmente para construir un primer prototipo para esta etapa de extracción de datos. Más concretamente, se utilizarían las siguientes:

\begin{itemize}
    \item \textbf{\textit{Python Twitter Search API}.} Cliente en lenguaje \textit{Python} enfocado en utilizar los \textit{endpoints} de búsqueda de \textit{tweets} \cite{twitterSearchAPI}.
    \item \textbf{\textit{Python-Facebook}.} Una librería simple de \textit{Python} que simplifica el uso de la \textit{Graph API} de Meta, dando soporte tanto para Facebook como para Instagram \cite{facebookAPIsearch}.
    \item \textbf{\textit{PRAW: The Python Reddit API Wrapper}.} Paquete de \textit{Python} que facilita el acceso a la \textit{API} de Reddit \cite{redditPRAW}.
\end{itemize}

No obstante, tras comenzar a trabajar con ellas se observaron fallos de dependencias y partes deprecadas. Esto originó inconsistencias entre las versiones de las \textit{APIs} actuales y el código de dichas librerías, además de provocar contratiempos en la evaluación del \textit{sprint} correspondiente. Por estas razones, se terminó descartando esta opción y utilizando la que se propone a continuación. 

\subsubsection{Herramienta elegida: Airbyte}

Esta plataforma\footnote{\url{https://airbyte.com/}} permite realizar la creación de un \textit{pipeline} de extracción y guardado de datos de forma sencilla y rápida. Presenta una versión de código abierto y distribuida en contenedores \textit{docker}, junto a una interfaz web que simplifica el proceso de gestión.

Permite la creación, definición y configuración tanto de fuentes de datos como de destinos de los mismos. Actualmente cuenta con más de 300 conectores disponibles para distintas aplicaciones e interfaces web \cite{airbyteConnectors}.

Esta herramienta es la que se ha terminado utilizando en la fase de prototipado para la etapa de ingestión de datos de la herramienta final desarrollada.

\subsection{Herramienta de carga de datos}

Los datos adquiridos mediante la herramienta de extracción de datos han de ser cargados en alguna herramienta y poder ser consultados posteriormente cuando sea necesario. Además, el sistema seleccionado para esta tarea no puede utilizar un esquema de datos estricto ya que se está trabajando con datos no estructurados.

Para cumplir con estos requisitos, se ha investigado viabilidad de las siguientes opciones.

\subsubsection{Herramientas consideradas:}

\begin{itemize}
    \item \textbf{\textit{Apache HDFS} (\textit{Hadoop Distributed File System}).} Sistema de ficheros distribuido escalable horizontalmente, tolerante a fallos y de alto rendimiento para procesar grandes conjuntos de datos. Complejo de configurar correctamente para conseguir eficiencia óptima~\cite{hdfsArchitecture}.
    \item \textbf{\textit{Apache Haudi}.} Plataforma \textit{data lakehouse} \textit{open-source} de procesamiento de datos tanto en \textit{streaming} como en \textit{batch}. Permite ingestión de datos eficiente, versionado y actualización de datos, con soporte para transacciones \textit{ACID}. Actualmente aún presenta poca documentación y soporte, además de requerir bastantes recursos \textit{hardware} para conseguir un rendimiento óptimo~\cite{hudiOverview}.
    \item \textbf{\textit{Apache Cassandra}.} Base de datos \textit{NoSQL} columnar enfocada al procesamiento de grandes conjuntos de datos con alto rendimiento y tolerante a fallos, con gran eficiencia para escritura y replicación de datos. Presenta soporte limitado para realizar transacciones complejas y requiere definir previamente el esquema de los datos a cargar para conseguir el rendimiento óptimo~\cite{cassandraOverview}.
    \item \textbf{\textit{MongoDB}.} Base de datos \textit{NoSQL} orientada a documentos escalable y flexible que permite la ingestión y consulta eficientes de estructuras de datos complejas en formato \textit{JSON}. Presenta soporte limitado para modelado de datos relacional y para transacciones complejas~\cite{mongodbArchitecture}.
\end{itemize}

\subsubsection{Herramienta elegida:}

\begin{itemize}
    \item \textbf{\textit{MongoDB}}\footnote{\url{https://www.mongodb.com/}}
\end{itemize}

Se ha seleccionado esta herramienta por la sencilla integración y facilidad de uso que presenta para el caso de uso concreto de este proyecto. Los datos extraídos procedentes de las \textit{APIs} seleccionadas se presentan en su totalidad en formato \textit{JSON}, por lo que se pueden cargar directamente en esta base de datos sin realizar ninguna transformación intermedia. 

\subsection{Apache Spark}

Apache Spark\footnote{\url{https://spark.apache.org/}} es un sistema de procesamiento de datos distribuido y de código abierto. Spark es capaz de procesar grandes volúmenes de datos de manera rápida, eficiente y escalable.

Permite procesar datos en memoria, lo que lo hace más rápido que otros sistemas de procesamiento de datos como Hadoop MapReduce, que requieren que los datos se escriban en disco entre las operaciones. Sus principales características se centran en el rendimiento, escalabilidad, facilidad de uso y flexibilidad.

\subsection{Base de datos para procesamiento analítico de datos en línea}

Los datos extraídos y los inferidos a través de los modelos de procesamiento de lenguaje natural empleados han de ser persistidos posteriormente para su consiguiente explotación de manera visual. Por tanto, para llevar a cabo esta tarea de forma óptima, resulta necesario emplear una base de datos enfocada al procesamiento analítico en línea (\textit{OLAP}) de los datos, en lugar de emplear modelos enfocados al procesamiento de transacciones en línea (\textit{OLPT}) comunes.

\subsubsection{Herramientas consideradas:}

\begin{itemize}
    \item \textbf{\textit{ClickHouse}.} Sistema gestor de base de datos \textit{OLAP}\footnote{\url{https://clickhouse.com/}} basado en arquitectura \textit{share-nothing}, compresión de datos y posibilidad de consultas en lenguaje SQL nativo. Utiliza un concepto de vistas materializadas que se gestionan automáticamente para asegurar un rendimiento óptimo de las consultas en cualquier momento.
    
    \item \textbf{\textit{Apache Druid}.} Base de datos \textit{OLAP}\footnote{\url{https://druid.apache.org/}} distribuida, escalable y autobalanceada. Al contrario que ClickHouse, presenta una arquitectura modular en la que las consultas, datos y nodos de almacenamiento se encuentran separados unos de otros. A pesar de esto, realiza una gestión óptima de datos en \textit{streaming} y permite priorizar consultas.
\end{itemize}

\subsubsection{Herramienta elegida:}

\begin{itemize}
    \item \textbf{\textit{ClickHouse}}
\end{itemize}

La herramienta \textit{ClickHouse} ha elegido teniendo en cuenta los requisitos de este proyecto concreto. De esta manera, se permitirá una baja latencia y un buen rendimiento para las consultas realizadas. Además, la gestión y configuración no resultará tan complicada como la exigida por \textit{Apache Druid}, que necesita de varios servicios y componentes para funcionar correctamente. Por lo que el coste computacional y de mantenimiento sería mayor en comparación con la herramienta elegida.

\subsection{Herramienta analítica de visualización de datos}

Una vez que se tiene todos los datos extraídos, procesados, transformados e inferidos persistidos en el sistema \textit{OLAP}, se encuentran ya disponibles para su explotación visual. Para ello, es necesario utilizar una herramienta de visualización compatible con el sistema \textit{OLAP} empleado, capaz de explotar visualmente y de manera eficaz los datos obtenidos para asegurar un buen entendimiento de los mismos.

\subsubsection{Herramientas consideradas:}

\begin{itemize}
    \item \textbf{\textit{Metabase}.} Herramienta de exploración y visualización de datos\footnote{\url{https://www.metabase.com/}} rápida, ligera y con una interfaz sencilla. Permite la creación de gráficos simples a través de la formulación de preguntas interactivas. Posee numerosos recursos de visualización de datos y conectores para las principales bases de datos. 
    
    \item \textbf{\textit{Apache Superset}.} Plataforma de visualización y exploración de datos moderna\footnote{\url{https://superset.apache.org/}} e intuitiva que permite la gestión de \textit{dashboards}, roles y consultas asíncronas. Ofrece gran cantidad de recursos tanto para visualización de datos como conectores a bases de datos.
\end{itemize}

\subsubsection{Herramienta elegida:}

\begin{itemize}
    \item \textbf{\textit{Apache Superset}}
\end{itemize}

La herramienta \textit{Apache Superset} presenta todas las características necesarias para una correcta gestión y visualización de los datos. Permite la asignación de roles y permisos de usuario, gestión de \textit{dashboards} y gráficos individuales, construcción de consultas tanto en lenguaje \textit{SQL} nativo como de manera interactiva y visual, más de 40 tipos de visualizaciones distintas y más de 30 conectores para distintas bases de datos.
