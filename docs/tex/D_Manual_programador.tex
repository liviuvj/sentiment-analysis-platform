\apendice{Documentación técnica de programación}

\section{Introducción}

En esta sección se incluye la documentación técnica necesaria para entender la organización de directorios, la manera de instalar, configurar y ejecutar el proyecto, y los pasos a seguir para continuar con futuros desarrollos.


\section{Estructura de directorios}

\section{Manual del programador}

En esta sección se describirán todos los elementos necesarios y la metodología a seguir para realizar futuros desarrollos.

\subsection{Configuración de Airbyte}

En este apartado se detallarán los métodos de configuración de la herramienta de extracción de datos.

\subsubsection{Configuración del origen de datos}

A continuación, se va a crear un ejemplo de fichero de configuración para una de las fuentes de datos seleccionada.

\begin{itemize}
    
    \item \textbf{Utilizando la API.}
    
    \begin{enumerate}
        
        \item \textbf{Consultar la definición específica de la fuente de datos a configurar.} Para ello es necesario consultar el \textit{endpoint} \verb|/v1/source_definitions/get|, lo que resultaría en una respuesta como la siguiente en el caso de escoger Twitter como fuente de datos (abreviada debido a su extensión real):

        \begin{verbatim}
        {
          "sourceDefinitionId": "...",
          "documentationUrl": "...",
          "connectionSpecification": {
            "type": "object",
            "title": "Twitter Spec",
            "$schema": "...",
            "required": [
              "api_key",
              "query"
            ],
            "properties": {...},
            ...
          },
          "jobInfo": {...}
          }
        }
        \end{verbatim}
        
        \item \textbf{Enviar la petición de creación de fuente de datos.} Completando el \textit{payload} de la petición hacia \verb|/v1/sources/create| con las propiedades correspondientes obtenidas del paso anterior. 
        
        \item \textbf{Comprobar la conexión con la fuente de datos.} Enviando una petición a \verb|/v1/sources/check_connection|.
    
    \end{enumerate}
    
    \item \textbf{Utilizando la interfaz gráfica.} Navegando a la sección \textit{Sources} y seleccionando el tipo de fuente a configurar, se muestra un formulario de edición que permite realizar las mismas operaciones ejecutadas anteriormente mediante la \textit{API}. En la \autoref{fig:airbyte_source_config} se puede observar dicho formulario.

    \imagen{airbyte_source_config}{Configuración del origen de datos en Airbyte: Twitter}
    
\end{itemize}

\subsubsection{Configuración del destino de los datos}

A continuación, se va a crear un ejemplo de fichero de configuración para uno de los destinos de datos seleccionados.

\begin{itemize}
    
    \item \textbf{Utilizando la API.}
    
    \begin{enumerate}
        
        \item \textbf{Consultar la definición específica del destino de datos a configurar.} Para ello es necesario consultar el \textit{endpoint} \\ \verb|/v1/destination_definition_specifications/get|, lo que resultaría en una respuesta como la siguiente en el caso de escoger un fichero \textit{CSV} local como destino de datos (abreviada debido a su extensión real):

        \begin{verbatim}
        {
            "destinationDefinitionId": "...",
            "documentationUrl": "...",
            "connectionSpecification": {
                "type": "object",
                "title": "CSV Destination Spec",
                "$schema": "...",
                "required": [
                    "destination_path"
                ],
                "properties": {...},
                ...
            },
            "jobInfo": {...},
            "supportedDestinationSyncModes": [
                "overwrite",
                "append"
            ]
        }
        \end{verbatim}
        
        \item \textbf{Enviar la petición de creación de destino de datos.} Completando el \textit{payload} de la petición hacia \verb|/v1/destinations/create| con las propiedades correspondientes obtenidas del paso anterior. 
        
        \item \textbf{Comprobar la conexión con el destino de datos.} Enviando una petición a \verb|/v1/destinations/check_connection|.
    
    \end{enumerate}
    
    \item \textbf{Utilizando la interfaz gráfica.} Navegando a la sección \textit{Destinations} y seleccionando el tipo de fuente a configurar, se muestra un formulario de edición que permite realizar las mismas operaciones ejecutadas anteriormente mediante la \textit{API}. En la \autoref{fig:airbyte_destination_config} se puede observar dicho formulario.

    \imagen{airbyte_destination_config}{Configuración del destino de datos en Airbyte: CSV Local}
    
\end{itemize}

\subsubsection{Configuración de la conexión entre fuente y destino}

A continuación, se va a crear un ejemplo de fichero de configuración para una de las conexiones de datos seleccionada.

\begin{itemize}
    
    \item \textbf{Utilizando la API.}
    
    \begin{enumerate}
        
        \item \textbf{Crear la conexión entre fuente y destino.} Para ello es necesario mandar una petición al \textit{endpoint} \verb|/v1/connections/create| con un \textit{payload} como el siguiente (abreviado debido a que es bastante extenso):

        \begin{verbatim}
        {
          "name": "Twitter-API <> Local-CSV",
          "namespaceDefinition": "destination",
          "sourceId": "...",
          "destinationId": "...",
          "syncCatalog": {
            "streams": [
              {
                "stream": {
                  "name": "...",
                  "supportedSyncModes": [
                    "full_refresh", "incremental"
                  ]
                },
                "config": {
                  "syncMode": "full_refresh",
                  "destinationSyncMode": "append",
                  "aliasName": "...",
                  "selected": true
                }
              }
            ]
          },
          "scheduleType": "manual",
          "status": "active",
          "geography": "auto",
          "notifySchemaChanges": true,
          "nonBreakingChangesPreference": "ignore"
        }
        \end{verbatim}
        
        \item \textbf{Ejecutar una sincronización manual de la conexión.} Mandando una petición hacia \verb|/v1/connections/sync|.
    
    \end{enumerate}
    
    \item \textbf{Utilizando la interfaz gráfica.} Navegando a la sección \textit{Connections} y seleccionando las fuentes y destinos de datos para los que configurar la conexión. Se muestra un formulario de edición que permite realizar las mismas operaciones ejecutadas anteriormente mediante la \textit{API}. En la \autoref{fig:airbyte_connection_config} se puede observar dicho formulario.

    \imagen{airbyte_connection_config}{Configuración de la conexión entre origen y destino de los datos en Airbyte}
    
\end{itemize}

\section{Compilación, instalación y ejecución del proyecto}

\section{Pruebas del sistema}
