\capitulo{5}{Aspectos relevantes del desarrollo del proyecto} \label{section:relevant_aspects}

Este apartado pretende recoger los aspectos más interesantes del desarrollo del proyecto, además de la experiencia práctica adquirida durante la
realización del mismo con las diversas tecnologías empleadas.

\section{Extracción de datos} \label{section:data_extraction}

Comenzando con la primera etapa del proceso ETL, el objetivo de la extracción de datos consiste en investigar y explotar los posibles recursos disponibles para recoger toda la información necesaria para el proyecto. Consecuentemente, los requisitos fundamentales de esta etapa consistirán en localizar las fuentes de datos a utilizar y emplear las herramientas necesarias para extraer dichos datos.

\subsection{Fuentes de datos}

La información necesaria para conseguir el objetivo de este proyecto está formada por opiniones públicas de personas sobre algún tema o temas en concreto. La manera más sencilla de obtener estos datos es empleando recursos web como foros, blogs y redes sociales. Más concretamente, se ha optado por investigar la disponibilidad de \textit{APIs} públicas de los principales sitios web donde las personas publican sus opiniones.

A continuación, se realiza un pequeño resumen de la información de la que se dispone actualmente sobre las \textit{APIs} de cada plataforma.

\subsubsection{Twitter}

En 2006 se abrió al público la \textit{API REST} \cite{twitterGettingStarted} de Twitter, que actualmente se encuentra ya en su versión \textbf{v2}, aunque coexiste a su vez con algunas partes de la misma aún en la versión \textbf{v1.1} y otras de pago (\textit{Premium v1.1} o \textit{Enterprise}). Está basada en \textit{GraphQL}\footnote{Lenguaje de consultas para \textit{API}s que facilita la gestión de datos y peticiones (\url{https://graphql.org/}).} y devuelve los resultados en formato \textit{JSON}.

Los permisos que se deben asignar son solo de lectura o escritura de contenido. Mientras que el número de peticiones varía en función del \textit{endpoint}, la ventana temporal de restricción se limita a tan solo 15 minutos \cite{twitterRateLimits}. 

Ofrece acceso de lectura, escritura, modificación y borrado de una amplia variedad de recursos, como puede verse en la \autoref{tabla:recursosTwitter}.

\tablaSmallSinColoresConFuente{Recursos disponibles a través de la \textit{API} de Twitter}{\cite{twitterAPIreference}}{@{}p{.3 \textwidth} p{.1 \textwidth} p{.5 \textwidth}@{}}{recursosTwitter}
{\multicolumn{1}{l}{\textbf{Recurso}} & \multicolumn{1}{l}{\textbf{Versión}} & \multicolumn{1}{l}{\textbf{Descripción}}\\}{
\textbf{\textit{Tweets}} & v2 & Operaciones \textit{CRUD}. \\
& v1.1 & \\
& \textit{Premium} & \\
& \textit{Enterprise} & \\
\specialrule{.05em}{.05em}{0em}
\textbf{\textit{Users}} & v2 & Gestión y búsqueda de usuarios \\
& v1.1 & y relaciones entre los mismos. \\
& \textit{Premium} & \\
& \textit{Enterprise} & \\
\specialrule{.05em}{.05em}{0em}
\textbf{\textit{Spaces}} & v2 & Búsqueda de espacios y participantes. \\
\specialrule{.05em}{.05em}{0em}
\textbf{\textit{Direct Messages}} & v1.1 & Envío y respuesta a mensajes directos. \\
\specialrule{.05em}{.05em}{0em}
\textbf{\textit{Lists}} & v2 & Gestión de listas de contactos. \\
& v1.1 & \\ 
\specialrule{.05em}{.05em}{0em}
\textbf{\textit{Trends}} & v1.1 & Identificar tendencias por zonas geográficas. \\
\specialrule{.05em}{.05em}{0em}
\textbf{\textit{Media}} & v1.1 & Cargar archivos multimedia. \\
\specialrule{.05em}{.1em}{.1em}
\textbf{\textit{Places}} & v1.1 & Búsqueda de lugares. \\
}

\subsubsection{Facebook}

La \textit{API} de Facebook originalmente utilizaba \textit{FQL} (\textit{Facebook Query Language}) como lenguaje de consulta, parecido a \textit{SQL}. Sin embargo, en 2010 comenzó la migración hacia \textit{Graph API} \cite{facebookGraphAPI}, actualmente en su versión \textbf{v16.0}. Se organiza en función de colecciones, nodos y campos. Un nodo es un objeto único que representa una clase del diccionario de datos en concreto, mientras que los campos son atributos del mismo y una colección comprende un conjunto de nodos. Toda esta información es presentada en formato \textit{JSON}.

Presenta una gran cantidad de permisos \cite{facebookPermissions} que son requeridos para realizar las acciones de gestión, algunos de los cuales es necesario que sean aprobados por Facebook para su uso. Además, el número de peticiones que se pueden realizar se limita a 200 por hora por cada usuario \cite{facebookRateLimits}.

La lista de nodos que presenta la \textit{API} es muy extensa, aunque en la \autoref{tabla:nodosFacebook} se muestran algunos de ellos que podrían resultar útiles para el desarrollo de este proyecto.

\tablaSmallSinColoresConFuente{Muestra de nodos disponibles en \textit{Graph API} de Facebook}{\cite{facebookAPIreference}}{l l}{nodosFacebook}
{\multicolumn{1}{l}{\textbf{Nodo}} & \multicolumn{1}{l}{\textbf{Descripción}}\\}{
\textbf{\textit{Comment}} & Comentarios de los objetos. \\
\textbf{\textit{Link}} & Enlaces compartidos. \\
\textbf{\textit{Group}} & Objeto único de tipo grupo. \\
\textbf{\textit{Likes}} & Lista de personas que han dado \textit{like} a un objeto. \\
\textbf{\textit{Page}} & Información sobre páginas. \\
\textbf{\textit{User}} & Representación de un usuario. \\
}

\subsubsection{Instagram}

Lanzada originalmente en 2014 y actualmente integrada junto a la \textit{Graph API} de Facebook. Dispone de dos versiones, una más básica enfocada solamente al consumo de contenido, y la normal, que permite realizar diversos tipos de acciones sobre la cuenta y llevar a cabo su gestión.

Se dispone de un conjunto de permisos requeridos bastante más reducido que para la \textit{API} de Facebook, aunque el límite de peticiones es el mismo ya que funciona sobre la propia \textit{Graph API}.

Al estar basada también en la misma tecnología, la estructura consta de los mismos elementos mencionados en el apartado anterior. La diferencia serían los nodos principales en los que se distribuye su contenido, como se observa en la \autoref{tabla:nodosInstagram}.

\tablaSmallSinColoresConFuente{Muestra de nodos disponibles en \textit{Graph API} de Instagram}{\cite{instagramAPIreference}}{l l}{nodosInstagram}
{\multicolumn{1}{l}{\textbf{Nodo}} & \multicolumn{1}{l}{\textbf{Descripción}}\\}{
\textbf{\textit{Comment}} & Comentarios de los objetos. \\
\textbf{\textit{Hashtag}} & Representa un \textit{hashtag}. \\
\textbf{\textit{Multimedia}} & Referencia una foto, vídeo, historia o álbum. \\
\textbf{\textit{User}} & La cuenta de un usuario. \\
\textbf{\textit{Page}} & Información sobre páginas. \\
}

\subsubsection{YouTube}

Introducida en el año 2013, actualmente en su versión \textbf{v3}, y permite la integración de funcionalidades de la plataforma, búsqueda de contenido y análisis demográficos. Cada recurso se representa como un objeto \textit{JSON} sobre el que se pueden ejecutar varias acciones.

Respecto a permisos requeridos, no son tan estrictos como por parte de Meta. No obstante, el número de peticiones se calcula en función de las ``unidades'' que consume cada tipo de petición, teniendo un total básico de 10\,000 al día \cite{youtubeRateLimits}.

Cada recurso se representa como un objeto de datos con identificador único. Entre ellos, los más representativos para la realización de este proyecto podrían ser los expuestos en la \autoref{tabla:recursosYouTube}.

\tablaSmallSinColoresConFuente{Recursos disponibles a través de la \textit{API} de YouTube}{\cite{youtubeAPIreference}}{l l}{recursosYouTube}
{\multicolumn{1}{l}{\textbf{Recurso}} & \multicolumn{1}{l}{\textbf{Descripción}}\\}{
\textbf{\textit{Caption}} & Representa los subtítulos de un vídeo. \\
\textbf{\textit{Comment}} & Comentarios de los objetos. \\
\textbf{\textit{Playlist}} & Colección de vídeos accesibles de forma secuencial. \\
\textbf{\textit{Search result}} & Información de una búsqueda que apunta a un objeto. \\
\textbf{\textit{Video}} & Objeto representativo para un vídeo. \\
}

\subsubsection{Reddit}

Esta \textit{API} fue lanzada en 2011, proporcionando acceso y gestión sobre todas las acciones disponibles desde su interfaz web. También la menos restrictiva de las estudiadas en esta sección, aunque no por ello menos trabajada.

Como ventaja respecto al resto, permite realizar hasta 60 peticiones por minuto \cite{redditRateLimits} a través de todos sus \textit{endpoints}.

Los recursos se representan como objetos tipo \textit{JSON}. En la \autoref{tabla:recursosReddit} se pueden observar las principales estructuras de datos.

\tablaSmallSinColoresConFuente{Recursos disponibles a través de la \textit{API} de Reddit}{\cite{redditAPIreference}}{l l}{recursosReddit}
{\multicolumn{1}{l}{\textbf{Recurso}} & \multicolumn{1}{l}{\textbf{Descripción}}\\}{
\textbf{\textit{Comment}} & Comentario de las demás estructuras de datos. \\
\textbf{\textit{Subreddit}} & Representación de un subforo. \\
\textbf{\textit{Message}} & Información sobre mensajes. \\
\textbf{\textit{Account}} & Datos de la cuenta de un usuario. \\
}

\subsection{Método de extracción}

Para realizar la extracción de los datos se comenzó a realizar un prototipo inicial empleando los \textit{API wrappers} mencionados anteriormente (véase la \autoref{section:api_wrappers}). No obstante, debido a las razones ya explicadas en dicho apartado, finalmente se ha optado por utilizar la herramienta Airbyte para realizar esta tarea.

Esta plataforma de código abierto se ha instalado en la máquina local mediante contenedores \textit{docker}, lo que ha facilitado en gran medida su despliegue ya que está compuesta por una arquitectura compleja con varios servicios interconectados entre sí. Cuenta con una interfaz web sencilla que permite realizar la gestión y configuración de fuentes de datos, destinos de datos y conexiones. También presenta una \textit{API} propia \cite{airbyteAPI} desde la que es posible gestionar las configuraciones de dichos recursos sin necesidad de acceder a su interfaz web.

\subsubsection{Configuración de la fuente de datos}

A continuación, se va a crear un ejemplo de fichero de configuración para una de las fuentes de datos seleccionada.

\begin{itemize}
    
    \item \textbf{Utilizando la API.}
    
    \begin{enumerate}
        
        \item \textbf{Consultar la definición específica de la fuente de datos a configurar.} Para ello es necesario consultar el \textit{endpoint} \verb|/v1/source_definitions/get|, lo que resultaría en una respuesta como la siguiente en el caso de escoger Twitter como fuente de datos (abreviada debido a su extensión real):

        \begin{verbatim}
        {
          "sourceDefinitionId": "...",
          "documentationUrl": "...",
          "connectionSpecification": {
            "type": "object",
            "title": "Twitter Spec",
            "$schema": "...",
            "required": [
              "api_key",
              "query"
            ],
            "properties": {...},
            ...
          },
          "jobInfo": {...}
          }
        }
        \end{verbatim}
        
        \item \textbf{Mandar la petición de creación de fuente de datos.} Completando el \textit{payload} de la petición hacia \verb|/v1/sources/create| con las propiedades correspondientes obtenidas del paso anterior. 
        
        \item \textbf{Comprobar la conexión con la fuente de datos.} Mandando una petición a \verb|/v1/sources/check_connection|.
    
    \end{enumerate}
    
    \item \textbf{Utilizando la interfaz gráfica.} Navegando a la sección \textit{Sources} y seleccionando el tipo de fuente a configurar, se muestra un formulario de edición que permite realizar las mismas operaciones ejecutadas anteriormente mediante la \textit{API}. En la \autoref{fig:airbyte_source_config} se puede observar dicho formulario.

    \imagen{airbyte_source_config}{Configuración de la fuente de datos}
    
\end{itemize}

\subsubsection{Configuración del destino de los datos}

A continuación, se va a crear un ejemplo de fichero de configuración para uno de los destinos de datos seleccionados.

\begin{itemize}
    
    \item \textbf{Utilizando la API.}
    
    \begin{enumerate}
        
        \item \textbf{Consultar la definición específica del destino de datos a configurar.} Para ello es necesario consultar el \textit{endpoint} \\ \verb|/v1/destination_definition_specifications/get|, lo que resultaría en una respuesta como la siguiente en el caso de escoger un fichero \textit{CSV} local como destino de datos (abreviada debido a su extensión real):

        \begin{verbatim}
        {
            "destinationDefinitionId": "...",
            "documentationUrl": "...",
            "connectionSpecification": {
                "type": "object",
                "title": "CSV Destination Spec",
                "$schema": "...",
                "required": [
                    "destination_path"
                ],
                "properties": {...},
                ...
            },
            "jobInfo": {...},
            "supportedDestinationSyncModes": [
                "overwrite",
                "append"
            ]
        }
        \end{verbatim}
        
        \item \textbf{Mandar la petición de creación de destino de datos.} Completando el \textit{payload} de la petición hacia \verb|/v1/destinations/create| con las propiedades correspondientes obtenidas del paso anterior. 
        
        \item \textbf{Comprobar la conexión con el destino de datos.} Mandando una petición a \verb|/v1/destinations/check_connection|.
    
    \end{enumerate}
    
    \item \textbf{Utilizando la interfaz gráfica.} Navegando a la sección \textit{Destinations} y seleccionando el tipo de fuente a configurar, se muestra un formulario de edición que permite realizar las mismas operaciones ejecutadas anteriormente mediante la \textit{API}. En la \autoref{fig:airbyte_destination_config} se puede observar dicho formulario.

    \imagen{airbyte_destination_config}{Configuración de la fuente de datos}
    
\end{itemize}

\subsubsection{Configuración de la conexión entre fuente y destino}

A continuación, se va a crear un ejemplo de fichero de configuración para una de las conexiones de datos seleccionada.

\begin{itemize}
    
    \item \textbf{Utilizando la API.}
    
    \begin{enumerate}
        
        \item \textbf{Crear la conexión entre fuente y destino.} Para ello es necesario mandar una petición al \textit{endpoint} \verb|/v1/connections/create| con un \textit{payload} como el siguiente (abreviado debido a que es bastante extenso):

        \begin{verbatim}
        {
          "name": "Twitter-API <> Local-CSV",
          "namespaceDefinition": "destination",
          "sourceId": "...",
          "destinationId": "...",
          "syncCatalog": {
            "streams": [
              {
                "stream": {
                  "name": "...",
                  "supportedSyncModes": [
                    "full_refresh", "incremental"
                  ]
                },
                "config": {
                  "syncMode": "full_refresh",
                  "destinationSyncMode": "append",
                  "aliasName": "...",
                  "selected": true
                }
              }
            ]
          },
          "scheduleType": "manual",
          "status": "active",
          "geography": "auto",
          "notifySchemaChanges": true,
          "nonBreakingChangesPreference": "ignore"
        }
        \end{verbatim}
        
        \item \textbf{Ejecutar una sincronización manual de la conexión.} Mandando una petición hacia \verb|/v1/connections/sync|.
    
    \end{enumerate}
    
    \item \textbf{Utilizando la interfaz gráfica.} Navegando a la sección \textit{Connections} y seleccionando las fuentes y destinos de datos para los que configurar la conexión. Se muestra un formulario de edición que permite realizar las mismas operaciones ejecutadas anteriormente mediante la \textit{API}. En la \autoref{fig:airbyte_connection_config} se puede observar dicho formulario.

    \imagen{airbyte_connection_config}{Configuración de la fuente de datos}
    
\end{itemize}