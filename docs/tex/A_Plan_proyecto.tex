\apendice{Plan de Proyecto Software}

\section{Introducción}

En las siguientes secciones se realizará un estudio de la planificación temporal seguida durante el desarrollo de este proyecto, además de la viabilidad tanto económica como legal que podría llegar a suponer este trabajo.

Debido a la naturaleza inherente del proyecto, al no tratarse de un \textit{software} típicamente tradicional sino más bien centrado hacia la investigación e implementación de modelos de \textit{machine learning}, no ha resultado sencillo llevar a cabo algunas de las buenas prácticas y conceptos normales de acuerdo a un ``Plan de Proyecto Software'' tradicional.

\section{Planificación temporal}

La planificación del proyecto se ha llevado a cabo mediante la metodología de desarrollo ágil \textit{Scrum}. A continuación se realiza un desglose de los distintos \textit{Sprints} llevados a cabo.

Inicialmente, se presentan las tareas correspondientes a cada iteración del trabajo y su duración inicial estimada. Posteriormente, se realiza una comparación entre el tiempo total estimado y el real gastado mediante la ilustración de gráficos \textit{burn-down}.

\subsection{\textit{Sprint} 0 (01/02/2023 - 15/02/2023)}

Este Sprint inicial será dedicado a la preparación del entorno de trabajo para el proyecto. Se elegirán las herramientas con las que se trabajará en algunas de las etapas del proyecto, se investigarán técnicas y librerías a utilizar, se realizarán unas pruebas concepto iniciales y se comenzará la labor de documentación.

\begin{itemize}
    \item \textbf{Gestión del \textit{Sprint} (4h).}  Se realizará el planteamiento de las tareas a llevar a cabo a lo largo de este sprint y se documentarán en el apartado \textbf{A.2 - Planificación temporal} del \textbf{Apéndice A - Plan de Proyecto Software} de los anexos del proyecto.

    \item \textbf{Elegir IDE (2h).} Para la realización de este proyecto será necesaria la utilización de diversos lenguajes de programación, por lo que la elección de un entorno de desarrollo integrado adecuado resultará de gran ayuda.

    \item \textbf{Estudiar guía \LaTeX (2h).} Como objetivo para la generación de la memoria del proyecto, se va estudiar una guía sobre \LaTeX con el fin de recordar los conocimientos necesarios para poder crear la documentación correspondiente.
 
    \item \textbf{Documentación de la memoria - Técnicas y herramientas (4h).} Comenzar con la documentación de la memoria del proyecto, con la sección ``Técnicas y herramientas''. De manera inicial, se documentará lo siguiente:

    \begin{itemize}
        \item \textbf{Técnicas}
        \begin{itemize}
            \item \textit{Scrum}
            \item \textit{Natural Language Processing}
            \item \textit{Sentiment Analysis}
        \end{itemize}
        \item \textbf{Herramientas}
        \begin{itemize}
            \item GitHub
            \item ZenHub
            \item Overleaf
            \item Joplin
            \item Super Productivity
        \end{itemize}
    \end{itemize}
    
    \item \textbf{Documentación de la memoria - Trabajos relacionados (4h).} La siguiente parte de la memoria que se va a redactar será la sección de ``Trabajos relacionados''. En este apartado se describirán otras herramientas similares ya existentes que cumplen un propósito similar al planteado en este proyecto.
    
    También se escribirá sobre los principales artículos científicos que comprenden el \textit{state-of-the-art} relacionado con las técnicas de procesamiento de lenguaje natural que serán utilizadas.

    \item \textbf{Investigar y probar recursos NLP ya existentes (8h).} Ya que inicialmente no se prevé el desarrollo de un algoritmo NLP propio, se investigará el \textit{state-of-the-art} sobre análisis de sentimientos y se comprobará si existen recursos ya implementados para utilizar en el proyecto.

\end{itemize}


\section{Estudio de viabilidad}

\subsection{Viabilidad económica}

\subsection{Viabilidad legal}


