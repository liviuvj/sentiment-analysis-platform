\apendice{Plan de Proyecto Software} \label{section:project_plan}

\section{Introducción}

En las siguientes secciones se realizará un estudio de la planificación temporal seguida durante el desarrollo de este proyecto, además de la viabilidad tanto económica como legal que podría llegar a suponer este trabajo.

Debido a la naturaleza inherente del proyecto, al no tratarse de un \textit{software} típicamente tradicional sino más bien centrado hacia la investigación e implementación de modelos de \textit{machine learning}, no ha resultado sencillo llevar a cabo algunas de las buenas prácticas y conceptos normales de acuerdo a un <<Plan de Proyecto Software>> tradicional.

\section{Planificación temporal} \label{section:planification}

La planificación del proyecto se ha llevado a cabo mediante la metodología de desarrollo ágil \textit{Scrum}. A continuación se realiza un desglose de los distintos \textit{Sprints} llevados a cabo.

Inicialmente, se presentan las tareas correspondientes a cada iteración del trabajo y su duración inicial estimada. Posteriormente, se realiza una comparación entre el tiempo total estimado y el real empleado mediante la ilustración de gráficos \textit{burn-down}.

\subsection{\textit{Sprint} 0 (01/02/2023 -- 15/02/2023)}

Este \textit{Sprint} inicial se dedicará a la preparación del entorno de trabajo para el proyecto. Se elegirán las herramientas con las que se trabajará en algunas de las etapas del proyecto, se investigarán técnicas y librerías a utilizar, se realizarán unas pruebas concepto iniciales y se comenzará la labor de documentación.

\begin{itemize}
    \item \textbf{Gestión del \textit{Sprint} (4h).}  Se realizará el planteamiento de las tareas a llevar a cabo a lo largo de este sprint y se documentarán en la \autoref{section:planification} del \autoref{section:project_plan} de los anexos del proyecto.

    \item \textbf{Elegir IDE (2h).} Para la realización de este proyecto será necesaria la utilización de diversos lenguajes de programación, por lo que la elección de un entorno de desarrollo integrado adecuado resultará de gran ayuda.

    \item \textbf{Estudiar guía \LaTeX{} (2h).} Como objetivo para la generación de la memoria del proyecto, se va estudiar una guía sobre \LaTeX{} con el fin de recordar los conocimientos necesarios para poder crear la documentación correspondiente.
 
    \item \textbf{Documentación de la memoria - Técnicas y herramientas (4h).} Comenzar con la documentación de la memoria del proyecto, con la sección <<Técnicas y herramientas>>. De manera inicial, se documentará lo siguiente:

    \begin{itemize}
        \item \textbf{Técnicas}
        \begin{itemize}
            \item \textit{Scrum}
            \item \textit{Natural Language Processing}
            \item \textit{Sentiment Analysis}
        \end{itemize}
        \item \textbf{Herramientas}
        \begin{itemize}
            \item GitHub
            \item ZenHub
            \item Overleaf
            \item Joplin
            \item Super Productivity
        \end{itemize}
    \end{itemize}
    
    \item \textbf{Documentación de la memoria - Trabajos relacionados (4h).} La siguiente parte de la memoria que se va a redactar será el \autoref{section:related_works}. En este apartado se describirán otras herramientas similares ya existentes que cumplen un propósito similar al planteado en este proyecto.
    
    También se escribirá sobre los principales artículos científicos que comprenden el \textit{state-of-the-art} relacionado con las técnicas de procesamiento de lenguaje natural que serán utilizadas.

    \item \textbf{Investigar y probar recursos NLP ya existentes (8h).} Ya que inicialmente no se prevé el desarrollo de un algoritmo NLP propio, se investigará el \textit{state-of-the-art} sobre análisis de sentimientos y se comprobará si existen recursos ya implementados para utilizar en el proyecto.

\end{itemize}

\subsection{\textit{Sprint} 1 (15/02/2023 -- 01/03/2023)}

Durante la duración de este \textit{Sprint} se investigarán las \textit{APIs} de las posibles plataformas de las que se va a extraer la información textual y las herramientas disponibles para realizar la primera etapa del proceso \textit{ETL}. También se comenzará a realizar una primera prueba concepto utilizando los recursos investigados.

\begin{itemize}

    \item \textbf{Gestión del \textit{Sprint} (2h).} Se realizará el planteamiento de las tareas a llevar a cabo a lo largo de este sprint y se comenzará a documentar el \autoref{section:relevant_aspects}.

    \item \textbf{Elegir IDE (2h).} Para la realización de este proyecto será necesaria la utilización de diversos lenguajes de programación, por lo que la elección de un entorno de desarrollo integrado adecuado resultará de gran ayuda.

    \item \textbf{Investigar posibles APIs a utilizar (4h).} Como se ha especificado en el objetivo de este proyecto, se necesita información y opiniones públicas de las que poder obtener conocimiento sobre temas concretos. Para ello, se investigará la existencia de \textit{APIs} públicas de los principales sitios web en los que la gente suele expresar sus opiniones de manera general, siendo estos los foros, blogs y redes sociales.

    \item \textbf{Escoger tema inicial con alta polaridad (2h).} Para comprobar el correcto funcionamiento de los recursos \textit{NLP} investigados en el \textit{Sprint} anterior, se escogerá un tema con alta polaridad sobre el que se centrarán las pruebas de dichos recursos. De esta manera, será más sencillo de visualizar el correcto funcionamiento de estos y el análisis de sentimientos mediante ejemplos claros.

    \item \textbf{Investigar herramientas para realizar la extracción de datos (8h).} En este punto se comenzarán a investigar las posibles herramientas para realizar la etapa de extracción de datos del proyecto. Para ello, se compararán las principales alternativas disponibles para realizar la recogida de información de los recursos web descubiertos en este mismo \textit{Sprint}.

    \item \textbf{Crear prototipo inicial para la etapa de extracción de datos (8h).} Para realizar una mejor comparación de las herramientas investigadas en la tarea anterior, se creará un pequeño prototipo para esta primera etapa de extracción de datos sobre las \textit{APIs} seleccionadas, empleando para ello las tecnologías escogidas más relevantes.

    \item \textbf{Documentar los procesos del sprint actual (8h).} A lo largo de este sprint se va a realizar la investigación de varios recursos que deberán ser correctamente documentados, ya que las siguientes etapas del proyecto dependerán de la calidad de la información proporcionada inicialmente.

\end{itemize}

En este \textit{sprint} tuvo lugar un error de cálculo a la hora de planificar las tareas a realizar. La investigación inicial sobre los recursos de extracción de datos indicó como viable la utilización de los \textit{API wrappers} mencionados en la \autoref{section:data_extraction} cuando resultó no ser así. Por ello, la estimación inicial de crear un prototipo para la etapa de extracción de datos durante este \textit{sprint} se vio afectada, teniendo que completar su creación durante el siguiente \textit{sprint}.

\subsection{\textit{Sprint} 2 (01/03/2023 -- 15/03/2023)}

Durante la duración de este \textit{Sprint} se investigará la documentación de la herramienta de extracción de datos elegida y se terminará la creación del prototipo planteado inicialmente en el anterior \textit{sprint}.

\begin{itemize}

    \item \textbf{Gestión del \textit{Sprint} (2h).} Se realizará el planteamiento de las tareas a llevar a cabo a lo largo de este sprint y se comenzará a documentar el \autoref{section:relevant_aspects}.

    \item \textbf{Crear prototipo inicial para la etapa de extracción de datos (8h).} Para realizar una mejor comparación de las herramientas investigadas en la tarea anterior, se creará un pequeño prototipo para esta primera etapa de extracción de datos sobre las \textit{APIs} seleccionadas, empleando para ello las tecnologías escogidas más relevantes.

    \item \textbf{Crear cuenta de desarrollador para la API de Twitter (2h).} Para poder utilizar la \textit{API} de Twitter es necesario crear una cuenta de desarrollador para obtener los \textit{tokens} de acceso. Se creará una cuenta dedicada al proyecto para realizar las peticiones correspondientes.

    \item \textbf{Corregir memoria del proyecto (2h).} Se procederá a implementar las correcciones provistas a modo de \textit{feedback} por el tutor en los comentarios de las tareas.

    \item \textbf{Investigar documentación de la herramienta de extracción de datos elegida (4h).} Las herramientas de extracción de datos elegidas inicialmente para realizar esta labor no resultaron del todo óptimas como se ha mencionado. No obstante, otra de las alternativas que se planteaba utilizar más adelante parece resultar más adecuada. Por ello, se procederá a investigar la documentación disponible sobre la herramienta Airbyte \cite{airbyteOpenSource}.

    \item \textbf{Documentar prototipo de extracción de datos (4h).} Tras la creación del prototipo de extracción de datos, será necesario documentar el procedimiento también en la memoria del proyecto para que quede constancia del funcionamiento del mismo.

\end{itemize}

\subsection{\textit{Sprint} 3 (15/03/2023 -- 29/03/2023)}

Durante la duración de este \textit{Sprint} se investigarán las posibles herramientas a utilizar para la carga de los datos extraídos previamente en la anterior etapa y se continuará con el desarrollo del prototipo planteado.

\begin{itemize}

    \item \textbf{Gestión del \textit{Sprint} (2h).} Se realizará el planteamiento de las tareas a llevar a cabo a lo largo de este sprint.

    \item \textbf{Comentarios menores en la documentación (2h).} Se procederá a implementar el \textit{feedback} del tutor.

    \item \textbf{Investigar herramienta para realizar la carga de los datos (8h).} En este punto se comenzará a investigar las posibles herramientas para realizar la etapa de carga de datos del proyecto. Para ello, se compararán las principales alternativas disponibles para persistir la los datos extraídos.

    \item \textbf{Investigar documentación de la herramienta de carga de datos elegida (4h).} Tras la selección de la herramienta a utilizar para esta etapa del proyecto, se procederá a investigar su documentación para poder realizar un despliegue correcto de la misma e integrarla junto a los demás componentes del proyecto.

    \item \textbf{Desplegar e integrar la herramienta de carga de datos (8h).} Tras consultar la documentación necesaria, se procederá a realizar el despliegue y configuración de la herramienta para su correcta integración junto a los demás componentes del proyecto.

    \item \textbf{Documentar los procesos del \textit{sprint} actual (8h).} A lo largo de este \textit{sprint} se va a realizar la investigación de la herramienta a emplear para la carga de datos. Se procederá a documentar el despliegue e integración de dicha herramienta con los demás componentes del proyecto.

\end{itemize}

\subsection{\textit{Sprint} 4 (29/03/2023 -- 12/04/2023)}

Durante la duración de este \textit{Sprint} se realizará el procesamiento del conjunto de datos principalmente y se mejorará la extracción de datos del prototipo creado inicialmente.

\begin{itemize}

    \item \textbf{Gestión del \textit{Sprint} (2h).} Se realizará el planteamiento de las tareas a llevar a cabo a lo largo de este \textit{sprint}.

    \item \textbf{Modificar consulta para extracción de datos (4h).} El método actual para realizar la extracción de datos de la \textit{API} de Twitter presenta limitaciones en cuanto a la posibilidad de los parámetros a especificar. Se va a investigar cómo realizar dicha consulta de otra manera para poder recuperar la información adicional necesaria.

    \item \textbf{Modificación del conector base de Airbyte (8h).} El conector base utilizado para la extracción de datos de Twitter solamente permite realizar consultas simples a su \textit{API}. Para el desarrollo del proyecto y la información requerida en la consulta mejorada, es necesario desarrollar y mejorar el código fuente de este conector para permitir especificar los parámetros necesarios para las consultas correspondientes.

    \item \textbf{Procesar conjunto de datos (8h).} Tras la extracción y carga inicial de los datos, se va a proceder a realizar la limpieza correspondiente de los mismos con el objetivo de prepararlos para su futura explotación.

    \item \textbf{Documentar los procesos del \textit{sprint} actual (8h).} A lo largo de este \textit{sprint} se va a realizar la mejora del método de extracción de datos y el preprocesado de los mismos. Será necesaria la documentación de estos procesos para tener constancia de las modificaciones que se hayan realizado sobre el conjunto de datos extraído.

\end{itemize}

\subsection{\textit{Sprint} 5 (12/04/2023 -- 26/04/2023)}

Durante la duración de este \textit{Sprint} se concluirá la parte del procesamiento de datos y se comenzará la inferencia de los modelos de Procesamiento de Lenguaje Natural.

\begin{itemize}

    \item \textbf{Gestión del \textit{Sprint} (2h).} Se realizará el planteamiento de las tareas a llevar a cabo a lo largo de este \textit{sprint}.

    \item \textbf{Mejora del conector base de Airbyte (8h).} Previamente se modificó el conector base de Twitter para adaptarlo a la consulta realizada. Ahora se van a implementar unas mejoras adicionales que permitan mejor su integración con el código base de Airbyte.

    \item \textbf{Mejora del procesamiento del conjunto de datos (8h).} Previamente se comenzó con el procesamiento del conjunto de datos. Ahora se van a implementar mejoras sobre esta funcionalidad para permitir adaptarse a los nuevos cambios desarrollados para el conector.

    \item \textbf{Abrir PR al repositorio oficial de Airbyte con las mejoras desarrolladas (8h).} Las mejoras desarrolladas para el conector de Twitter pueden resultar de utilidad para los demás miembros de la comunidad que utilizan la herramienta Airbyte. Por ello, se va a crear un \textit{Pull Request} con los cambios realizados para que sean añadidos al repositorio oficial.

    \item \textbf{Comenzar con la inferencia de los modelos NLP (2h).} Una vez se han procesado y limpiado los datos correspondientes, se puede comenzar con la inferencia de los modelos NLP sobre los mismos.

\end{itemize}

\subsection{\textit{Sprint} 6 (26/04/2023 -- 10/05/2023)}

Durante la duración de este \textit{Sprint} se continuará con la inferencia de los modelos de Procesamiento de Lenguaje Natural.

\begin{itemize}

    \item \textbf{Gestión del \textit{Sprint} (2h).} Se realizará el planteamiento de las tareas a llevar a cabo a lo largo de este \textit{sprint}.

    \item \textbf{Continuación de la mejora del procesamiento de datos (4h).} Siguiendo con lo comentado en el \textit{sprint} anterior, se va a incluir el procesamiento de algunos datos adicionales.

    \item \textbf{Tarea NLP: \textit{Sentiment analysis} (4h).} Como continuación de lo comenzado en el \textit{sprint} anterior, se va a proceder con una de las tareas de NLP planteadas inicialmente: el análisis de sentimiento.

    \item \textbf{Tarea NLP: \textit{Topic classification} (4h).} Otra de las principales tareas NLP planteadas inicialmente será la tratada a continuación, la clasificación de temas.

    \item \textbf{Tarea NLP: \textit{Emotion classification} (4h).} Otra tarea NLP adicional que podría resultar interesante será la clasificación de emociones, que se diferencia de la clasificación de sentimientos en que la primera indica la emoción (alegre, triste, enfadado) de un texto y la segunda solamente la positividad o negatividad del texto.

    \item \textbf{Tarea NLP: \textit{Named entity recognition} (4h).} Otra tarea NLP que puede resultar de gran interés para el proyecto es el reconocimiento de entidades (\textit{NER}).

\end{itemize}

\subsection{\textit{Sprint} 7 (10/05/2023 -- 24/05/2023)}

Durante la duración de este \textit{Sprint} se mejorarán las tareas de inferencia de los modelos de Procesamiento de Lenguaje Natural y se investigará la manera óptima para persistir los datos obtenidos hasta el momento.

\begin{itemize}

    \item \textbf{Gestión del \textit{Sprint} (2h).} Se realizará el planteamiento de las tareas a llevar a cabo a lo largo de este \textit{sprint}.

    \item \textbf{Mejora de la etapa de transformación (8h).} Se van a desarrollar unas mejoras para la etapa de procesamiento de los datos que permitirán obtener una mayor calidad en la información final.

    \item \textbf{Mejora de tarea NLP: \textit{Sentiment analysis} (4h).} Se va a proceder a desarrollar mejoras para la tarea NLP que se encarga del análisis de sentimientos.

    \item \textbf{Mejora de tarea NLP: \textit{Topic classification} (4h).} Se va a proceder a desarrollar mejoras para la tarea NLP que se encarga de la clasificación de temas.

    \item \textbf{Mejora de tarea NLP: \textit{Emotion classification} (4h).} Se va a proceder a desarrollar mejoras para la tarea NLP que se encarga de la clasificación de emociones.

    \item \textbf{Mejora de tarea NLP: \textit{Named entity recognition} (4h).} Se va a proceder a desarrollar mejoras para la tarea NLP que se encarga del reconocimiento de entidades.

    \item \textbf{Investigar base de datos \textit{OLAP} (4h).} Los datos extraídos durante las fases anteriores y los inferidos a través de los modelos NLP empleados han de ser persistidos nuevamente para su posterior explotación de manera visual. Por tanto, para llevar a cabo esta tarea de forma óptima, será necesario investigar una base de datos enfocada al procesamiento analítico en línea (\textit{OLAP}) de los datos, en lugar de emplear modelos enfocados al procesamiento de transacciones en línea (\textit{OLPT}) comunes.

\end{itemize}

\subsection{\textit{Sprint} 8 (24/05/2023 -- 07/06/2023)}

Durante la duración de este \textit{Sprint}, el objetivo principal será investigar, desplegar e integrar el sistema óptimo a utilizar para persistir los datos obtenidos hasta el momento en una base de datos \textit{OLAP}. Se investigará también una herramienta de visualización compatible que utilizar posteriormente.

\begin{itemize}

    \item \textbf{Gestión del \textit{Sprint} (2h).} Se realizará el planteamiento de las tareas a llevar a cabo a lo largo de este \textit{sprint}.

    \item \textbf{Investigar documentación para el sistema \textit{OLAP} elegido (8h).} Tras la selección de la base de datos \textit{OLAP} a utilizar para las tareas de análisis visual, se procederá a investigar su documentación para poder realizar un despliegue correcto de este sistema e integrarlo junto a los demás componentes del proyecto.

    \item \textbf{Desplegar e integrar el sistema \textit{OLAP} (8h).} Tras consultar la documentación necesaria para poner en marcha el sistema \textit{OLAP}, se procederá a realizar el despliegue y configuración de la base de datos para su correcta integración junto a los demás componentes del proyecto.

    \item \textbf{Investigar herramienta de visualización (4h).} Una vez que se tienen todos los datos extraídos, procesados, transformados e inferidos persistidos en el sistema \textit{OLAP}, se encuentran listos para su puesta en explotación. Para ello, se ha de investigar una herramienta de visualización compatible con la tecnología empleada y acorde a los requisitos de uso, que pueda explotar visualmente y de manera eficaz los datos obtenidos.

    \item \textbf{Investigar documentación para la herramienta de visualización elegida (8h).} Tras la selección de una herramienta de visualización compatible con el sistema \textit{OLAP} elegido para las tareas de análisis visual, se procederá a investigar su documentación para poder realizar la integración y despliegue correctos, además de comprobar las posibilidades de configuración para las visualizaciones disponibles.

\end{itemize}

\subsection{\textit{Sprint} 9 (07/06/2023 -- 21/06/2023)}

Durante la duración de este \textit{Sprint}, el objetivo principal será desplegar e integrar la herramienta escogida para la visualización de los datos obtenidos hasta el momento, además de la realización de algunas mejoras para los procesos de las anteriores etapas.

\begin{itemize}

    \item \textbf{Gestión del \textit{Sprint} (2h).} Se realizará el planteamiento de las tareas a llevar a cabo a lo largo de este \textit{sprint}.

    \item \textbf{Aplicar correcciones sobre la memoria del proyecto (2h).} Se procederá a implementar las correcciones provistas a modo de \textit{feedback} por el tutor.

    \item \textbf{Mejorar integración del sistema \textit{OLAP} (4h).} Tras realizar la integración inicial con la herramienta \textit{OLAP ClickHouse}, se procederá a mejorar la configuración de las interacciones con esta base de datos para expandir sus posibilidades de integración con los demás componentes del proyecto.

    \item \textbf{Mejora y automatización de pipeline \textit{NLP} (8h).} Actualmente, el flujo de ejecución de las tareas \textit{NLP} se realiza de manera manual. Para poder ser utilizado de manera óptima, se va a proceder a mejorar este pipeline para permitir automatizar su ejecución. Además, el \textit{docker} correspondiente solamente necesita estar en ejecución mientras se ejecute esta parte del flujo, por lo que se va a asegurar que en cuanto se termine esta etapa se finalice también la ejecución del \textit{docker}.

    \item \textbf{Desplegar e integrar herramienta de visualización (8h).} Tras consultar la documentación necesaria para poner en marcha la herramienta de visualización seleccionada, se procederá a realizar el despliegue y configuración de \textit{Apache Superset} para su correcta integración junto a los demás componentes del proyecto.

    \item \textbf{Crear diseño inicial de los dashboards (8h).} Tras desplegar y configurar \textit{Apache Superset}, será necesario diseñar unos \textit{dashboards} para poder explotar los datos obtenidos. Para conseguir este objetivo, se comenzará a bocetar un diseño inicial de la forma que podrían adoptar los datos para interpretarlos y obtener información de calidad a partir de los mismos.

\end{itemize}


\subsection{\textit{Sprint} 10 (21/06/2023 -- 05/07/2023)}

Durante la duración de este \textit{Sprint}, el principal principal será el despliegue e integración de la herramienta \textit{Apache Airflow} para gestionar la orquestación de procesos. Además de la realización de algunas mejoras para los procesos de las anteriores etapas para facilitar la tarea anterior, también se integrarán nuevas fuentes de datos.

\begin{itemize}

    \item \textbf{Gestión del \textit{Sprint} (2h).} Se realizará el planteamiento de las tareas a llevar a cabo a lo largo de este \textit{sprint}.

    \item \textbf{Implementar \textit{dashboards} diseñados (4h).} Tras realizar el diseño de los \textit{dashboards} anteriormente, se procederá a realizar su implementación en la herramienta \textit{Apache Superset}. Al realizar este proceso, es posible que se realicen ciertas modificaciones sobre los diseños iniciales para adecuarse mejor a las posibilidades de la herramienta y de los datos.

    \item \textbf{Investigar herramienta de orquestación de procesos (4h).} Debido a la compleja arquitectura que presenta este proyecto, es necesario un método para gestionar todo el flujo de acciones que se lleva a cabo a través de los distintos componentes. Para ello, se investigará una herramienta que permita la orquestación de todos los procesos a ejecutar.

    \item \textbf{Investigar documentación para la herramienta de orquestación de procesos (8h).} Tras la selección de la herramienta de orquestación de procesos, se procederá a investigar su documentación para poder realizar la integración y despliegue correctos de la misma junto a los demás componentes del sistema.
    
    \item \textbf{Implementar ajustes en tareas \textit{Spark} (4h).} Para realizar la correcta integración y despliegue de \textit{Apache Airflow} es necesario realizar ciertas modificaciones en el funcionamiento del procesamiento de datos mediante \textit{Apache Spark}.
    
    \item \textbf{Implementar ajustes en tareas \textit{NLP} (4h).} Para realizar la correcta integración y despliegue de \textit{Apache Airflow} es necesario realizar ciertas modificaciones en el funcionamiento de la inferencia mediante \textit{NLP}.
    
    \item \textbf{Implementar ajustes en despliegue de \textit{MongoDB} y \textit{ClickHouse} (4h).} Para realizar la correcta integración y despliegue de \textit{Apache Airflow} es necesario realizar ciertas modificaciones en el despliegue de \textit{MongoDB} y del sistema \textit{OLAP ClickHouse}.
    
    \item \textbf{Despliegue e integración de la herramienta de orquestación de procesos (8h).} Tras consultar la documentación pertinente de \textit{Apache Airflow}, se procederá a realizar el despliegue y configuración del orquestador para su correcta integración junto a los demás componentes del proyecto.
    
    \item \textbf{Investigar nueva fuente de datos (4h).} Debido a los recientes cambios en las \textit{APIs} de Twitter y Reddit, su uso en el proyecto se ha vuelto poco viable. Por ello, se investigará otra fuente de datos como alternativa a las planteadas inicialmente.

    \item \textbf{Análisis exploratorio de nueva fuente de datos (4h).} Al haber seleccionado un conjunto de datos ya existente como la nueva fuente de datos, será necesario realizar un análisis exploratorio para ver las características de los datos disponibles y su usabilidad.

    \item \textbf{Aplicar correcciones sobre la memoria del proyecto (2h).} Se procederá a implementar las correcciones provistas a modo de \textit{feedback} por el tutor.

    \item \textbf{Integrar nueva fuente de datos (8h).} Tras investigar una nueva fuente de datos para el proyecto, será necesario proceder con su integración en las correspondientes etapas de extracción, procesamiento, inferencia, carga y visualización de los datos.

    \item \textbf{Documentar integración y despliegue de \textit{Apache Airflow} (2h).} Tras realizar la correcta integración y despliegue de esta herramienta de orquestación de procesos, se procederá a documentar los aspectos relevantes de lo trabajado en esta parte.

\end{itemize}


\section{Estudio de viabilidad}

\subsection{Viabilidad económica}

\subsection{Viabilidad legal}


