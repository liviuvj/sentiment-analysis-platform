\apendice{Plan de Proyecto Software} \label{section:project_plan}

\section{Introducción}

En las siguientes secciones se realizará un estudio de la planificación temporal seguida durante el desarrollo de este proyecto, además de la viabilidad tanto económica como legal que podría llegar a suponer este trabajo.

Debido a la naturaleza inherente del proyecto, al no tratarse de un \textit{software} típicamente tradicional sino más bien centrado hacia la investigación e implementación de modelos de \textit{machine learning}, no ha resultado sencillo llevar a cabo algunas de las buenas prácticas y conceptos normales de acuerdo a un ``Plan de Proyecto Software'' tradicional.

\section{Planificación temporal} \label{section:planification}

La planificación del proyecto se ha llevado a cabo mediante la metodología de desarrollo ágil \textit{Scrum}. A continuación se realiza un desglose de los distintos \textit{Sprints} llevados a cabo.

Inicialmente, se presentan las tareas correspondientes a cada iteración del trabajo y su duración inicial estimada. Posteriormente, se realiza una comparación entre el tiempo total estimado y el real empleado mediante la ilustración de gráficos \textit{burn-down}.

\subsection{\textit{Sprint} 0 (01/02/2023 -- 15/02/2023)}

Este \textit{Sprint} inicial se dedicará a la preparación del entorno de trabajo para el proyecto. Se elegirán las herramientas con las que se trabajará en algunas de las etapas del proyecto, se investigarán técnicas y librerías a utilizar, se realizarán unas pruebas concepto iniciales y se comenzará la labor de documentación.

\begin{itemize}
    \item \textbf{Gestión del \textit{Sprint} (4h).}  Se realizará el planteamiento de las tareas a llevar a cabo a lo largo de este sprint y se documentarán en la \autoref{section:planification} del \autoref{section:project_plan} de los anexos del proyecto.

    \item \textbf{Elegir IDE (2h).} Para la realización de este proyecto será necesaria la utilización de diversos lenguajes de programación, por lo que la elección de un entorno de desarrollo integrado adecuado resultará de gran ayuda.

    \item \textbf{Estudiar guía \LaTeX{} (2h).} Como objetivo para la generación de la memoria del proyecto, se va estudiar una guía sobre \LaTeX{} con el fin de recordar los conocimientos necesarios para poder crear la documentación correspondiente.
 
    \item \textbf{Documentación de la memoria - Técnicas y herramientas (4h).} Comenzar con la documentación de la memoria del proyecto, con la sección ``Técnicas y herramientas''. De manera inicial, se documentará lo siguiente:

    \begin{itemize}
        \item \textbf{Técnicas}
        \begin{itemize}
            \item \textit{Scrum}
            \item \textit{Natural Language Processing}
            \item \textit{Sentiment Analysis}
        \end{itemize}
        \item \textbf{Herramientas}
        \begin{itemize}
            \item GitHub
            \item ZenHub
            \item Overleaf
            \item Joplin
            \item Super Productivity
        \end{itemize}
    \end{itemize}
    
    \item \textbf{Documentación de la memoria - Trabajos relacionados (4h).} La siguiente parte de la memoria que se va a redactar será el \autoref{section:related_works}. En este apartado se describirán otras herramientas similares ya existentes que cumplen un propósito similar al planteado en este proyecto.
    
    También se escribirá sobre los principales artículos científicos que comprenden el \textit{state-of-the-art} relacionado con las técnicas de procesamiento de lenguaje natural que serán utilizadas.

    \item \textbf{Investigar y probar recursos NLP ya existentes (8h).} Ya que inicialmente no se prevé el desarrollo de un algoritmo NLP propio, se investigará el \textit{state-of-the-art} sobre análisis de sentimientos y se comprobará si existen recursos ya implementados para utilizar en el proyecto.

\end{itemize}

\imagen{sprint0}{Gráfico \textit{burn-down} del \textit{Sprint} 0}

\subsection{\textit{Sprint} 1 (15/02/2023 -- 01/03/2023)}

Durante la duración de este \textit{Sprint} se investigarán las \textit{APIs} de las posibles plataformas de las que se va a extraer la información textual y las herramientas disponibles para realizar la primera etapa del proceso \textit{ETL}. También se comenzará a realizar una primera prueba concepto utilizando los recursos investigados.

\begin{itemize}

    \item \textbf{Gestión del \textit{Sprint} (2h).} Se realizará el planteamiento de las tareas a llevar a cabo a lo largo de este sprint y se comenzará a documentar el \autoref{section:relevant_aspects}.

    \item \textbf{Elegir IDE (2h).} Para la realización de este proyecto será necesaria la utilización de diversos lenguajes de programación, por lo que la elección de un entorno de desarrollo integrado adecuado resultará de gran ayuda.

    \item \textbf{Investigar posibles APIs a utilizar (4h).} Como se ha especificado en el objetivo de este proyecto, se necesita información y opiniones públicas de las que poder obtener conocimiento sobre temas concretos. Para ello, se investigará la existencia de \textit{APIs} públicas de los principales sitios web en los que la gente suele expresar sus opiniones de manera general, siendo estos los foros, blogs y redes sociales.

    \item \textbf{Escoger tema inicial con alta polaridad (2h).} Para comprobar el correcto funcionamiento de los recursos \textit{NLP} investigados en el \textit{Sprint} anterior, se escogerá un tema con alta polaridad sobre el que se centrarán las pruebas de dichos recursos. De esta manera, será más sencillo de visualizar el correcto funcionamiento de estos y el análisis de sentimientos mediante ejemplos claros.

    \item \textbf{Investigar herramientas para realizar la extracción de datos (8h).} En este punto se comenzará a investigar las posibles herramientas para realizar la etapa de extracción de datos del proyecto. Para ello, se compararán las principales alternativas disponibles para realizar la recogida de información de los recursos web descubiertos en este mismo \textit{Sprint}.

    \item \textbf{Crear prototipo inicial para la etapa de extracción de datos (8h).} Para realizar una mejor comparación de las herramientas investigadas en la tarea anterior, se creará un pequeño prototipo para esta primera etapa de extracción de datos sobre las \textit{APIs} seleccionadas, empleando para ello las tecnologías escogidas más relevantes.

    \item \textbf{Documentar los procesos del sprint actual (8h).} A lo largo de este sprint se va a realizar la investigación de varios recursos que deberán ser correctamente documentados, ya que las siguientes etapas del proyecto dependerán de la calidad de la información proporcionada inicialmente.

\end{itemize}

En este \textit{sprint} tuvo lugar un error de cálculo a la hora de planificar las tareas a realizar. La investigación inicial sobre los recursos de extracción de datos indicó como viable la utilización de los \textit{API wrappers} mencionados en la \autoref{section:data_extraction} cuando resultó no ser así. Por ello, la estimación inicial de crear un prototipo para la etapa de extracción de datos durante este \textit{sprint} se vio afectada, teniendo que completar su creación durante el siguiente \textit{sprint}. Esto se puede ver reflejado en la \autoref{fig:sprint1}.

\imagen{sprint1}{Gráfico \textit{burn-down} del \textit{Sprint} 1}

\subsection{\textit{Sprint} 2 (01/03/2023 -- 15/03/2023)}

Durante la duración de este \textit{Sprint} se investigará la documentación de la herramienta de extracción de datos elegida y se terminará la creación del prototipo planteado inicialmente en el anterior \textit{sprint}.

\begin{itemize}

    \item \textbf{Gestión del \textit{Sprint} (2h).} Se realizará el planteamiento de las tareas a llevar a cabo a lo largo de este sprint y se comenzará a documentar el \autoref{section:relevant_aspects}.

    \item \textbf{Crear prototipo inicial para la etapa de extracción de datos (8h).} Para realizar una mejor comparación de las herramientas investigadas en la tarea anterior, se creará un pequeño prototipo para esta primera etapa de extracción de datos sobre las \textit{APIs} seleccionadas, empleando para ello las tecnologías escogidas más relevantes.

    \item \textbf{Crear cuenta de desarrollador para la API de Twitter (2h).} Para poder utilizar la \textit{API} de Twitter es necesario crear una cuenta de desarrollador para obtener los \textit{tokens} de acceso. Se creará una cuenta dedicada al proyecto para realizar las peticiones correspondientes.

    \item \textbf{Corregir memoria del proyecto (2h).} Se procederá a implementar las correcciones provistas a modo de \textit{feedback} por el tutor en los comentarios de las tareas.

    \item \textbf{Investigar documentación de la herramienta de extracción de datos elegida (4h).} Las herramientas de extracción de datos elegidas inicialmente para realizar esta labor no resultaron del todo óptimas como se ha mencionado. No obstante, otra de las alternativas que se planteaba utilizar más adelante parece resultar más adecuada. Por ello, se procederá a investigar la documentación disponible sobre la herramienta Airbyte \cite{airbyteOpenSource}.

    \item \textbf{Documentar prototipo de extracción de datos (4h).} Tras la creación del prototipo de extracción de datos, será necesario documentar el procedimiento también en la memoria del proyecto para que quede constancia del funcionamiento del mismo.

\end{itemize}

\imagen{sprint2}{Gráfico \textit{burn-down} del \textit{Sprint} 2}

\subsection{\textit{Sprint} 3 (15/03/2023 -- 29/03/2023)}

Durante la duración de este \textit{Sprint} se investigarán las posibles herramientas a utilizar para la carga de los datos extraídos previamente en la anterior etapa y se continuará con el desarrollo del prototipo planteado.

\begin{itemize}

    \item \textbf{Gestión del \textit{Sprint} (2h).} Se realizará el planteamiento de las tareas a llevar a cabo a lo largo de este sprint.

    \item \textbf{Comentarios menores en la documentación (2h).} Se procederá a implementar el \textit{feedback} del tutor.

    \item \textbf{Investigar herramienta para realizar la carga de los datos (8h).} En este punto se comenzará a investigar las posibles herramientas para realizar la etapa de carga de datos del proyecto. Para ello, se compararán las principales alternativas disponibles para persistir la los datos extraídos.

    \item \textbf{Investigar documentación de la herramienta de carga de datos elegida (4h).} Tras la selección de la herramienta a utilizar para esta etapa del proyecto, se procederá a investigar su documentación para poder realizar un despliegue correcto de la misma e integrarla junto a los demás componentes del proyecto.

    \item \textbf{Desplegar e integrar la herramienta de carga de datos (8h).} Tras consultar la documentación necesaria, se procederá a realizar el despliegue y configuración de la herramienta para su correcta integración junto a los demás componentes del proyecto.

    \item \textbf{Documentar los procesos del \textit{sprint} actual (8h).} A lo largo de este \textit{sprint} se va a realizar la investigación de la herramienta a emplear para la carga de datos. Se procederá a documentar el despliegue e integración de dicha herramienta con los demás componentes del proyecto.

\end{itemize}

\subsection{\textit{Sprint} 4 (29/03/2023 -- 12/04/2023)}

Durante la duración de este \textit{Sprint} se realizará el procesamiento del conjunto de datos principalmente y se mejorará la extracción de datos del prototipo creado inicialmente.

\begin{itemize}

    \item \textbf{Gestión del \textit{Sprint} (2h).} Se realizará el planteamiento de las tareas a llevar a cabo a lo largo de este \textit{sprint}.

    \item \textbf{Modificar consulta para extracción de datos (4h).} El método actual para realizar la extracción de datos de la \textit{API} de Twitter presenta limitaciones en cuanto a la posibilidad de los parámetros a especificar. Se va a investigar cómo realizar dicha consulta de otra manera para poder recuperar la información adicional necesaria.

    \item \textbf{Modificación del conector base de Airbyte (8h).} El conector base utilizado para la extracción de datos de Twitter solamente permite realizar consultas simples a su \textit{API}. Para el desarrollo del proyecto y la información requerida en la consulta mejorada, es necesario desarrollar y mejorar el código fuente de este conector para permitir especificar los parámetros necesarios para las consultas correspondientes.

    \item \textbf{Procesar conjunto de datos (8h).} Tras la extracción y carga inicial de los datos, se va a proceder a realizar la limpieza correspondiente de los mismos con el objetivo de prepararlos para su futura explotación.

    \item \textbf{Documentar los procesos del \textit{sprint} actual (8h).} A lo largo de este \textit{sprint} se va a realizar la mejora del método de extracción de datos y el preprocesado de los mismos. Será necesaria la documentación de estos procesos para tener constancia de las modificaciones que se hayan realizado sobre el conjunto de datos extraído.

\end{itemize}

\section{Estudio de viabilidad}

\subsection{Viabilidad económica}

\subsection{Viabilidad legal}


