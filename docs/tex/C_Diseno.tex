\apendice{Especificación de diseño}

\section{Introducción}

\section{Diseño de datos}

En los siguientes apartados se detallará el diseño de los modelos de datos utilizados en cada parte del proyecto y el proceso seguido para elaborar los cuadros de mando en los que se visualiza la información obtenida finalmente.

\subsection{Diseño de los modelos de datos}

En esta sección se explicarán los modelos de datos empleados entre los distintos componentes del proyecto.

\subsection{Diseño de los cuadros de mando}

A continuación, se detalla la evolución seguida para el diseño de los cuadros de mando implementados en la herramienta \textit{Apache Superset}.

\subsubsection{Primera iteración}

Los siguientes bocetos capturan la primera iteración del diseño inicial del \textit{dashboard} junto a la interfaz implementada con la herramienta \textit{Apache Superset}.

\imagen{dashboard-mockup-1}{Diseño inicial de la pestaña \textit{Raw data}}

\imagen{dashboard-1-iter-1}{Primera iteración de la pestaña \textit{Raw data}}

\imagen{dashboard-mockup-2}{Diseño inicial de la pestaña \textit{Sentiment overview}}

\imagen{dashboard-2-iter-1}{Primera iteración de la pestaña \textit{Sentiment overview}}

\imagen{dashboard-mockup-3}{Diseño inicial de la pestaña \textit{Sentiment analysis}}

\imagen{dashboard-3-iter-1}{Primera iteración de la pestaña \textit{Sentiment analysis}}

\section{Diseño procedimental}

\section{Diseño arquitectónico}


